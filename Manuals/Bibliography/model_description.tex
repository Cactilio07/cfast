

\section{Model Type}

CFAST is a two-zone fire model that predicts the thermal environment caused by a fire within a compartmented structure. Each compartment is divided into an upper and lower gas layer (zone in the term zone fire model refers to the layers being modeled). The fire drives combustion products from the lower to the upper layer via the plume. The temperature within each layer is uniform, and its evolution in time is described by a set of ordinary differential equations derived from the fundamental laws of mass and energy conservation. The transport of smoke and heat from zone to zone is dictated by empirical correlations. Because the governing equations are relatively simple, CFAST simulations typically require a few tens of seconds of CPU time on typical personal computers.


\section{Model Version}

The first public release of CFAST was version 1.0 in June, 1990. This version was restructured from FAST~\cite{Models:FAST} to incorporate the ``lessons learned'' from the zone model CCFM developed by Cooper and Forney~\cite{Models:CCFM}. Version~2 was released as a component of Hazard~1.2 in 1994~\cite{Models:HAZARDI, Models:HAZARDI_12}. The first of the 3.x series was released in 1995 and included a vertical flame spread algorithm, ceiling jets and non-uniform heat loss to the ceiling, spot targets, and heating and burning of multiple objects (ignition by heat flux, temperature or time) in addition to multiple prescribed fires. As it evolved over the next five years, version~3 included smoke and heat detectors, suppression through heat release reduction, better characterization of flow through doors and windows, vertical heat conduction through ceiling/floor boundaries, and non-rectangular compartments. In 2000, version~4 was released and included horizontal heat conduction through walls, and horizontal smoke flow in corridors. Version~5 improved the combustion chemistry. Version 6, released in July, 2005, incorporates a more consistent implementation of vents, fire objects, and event processing and includes a graphical user interface which substantially improves its usability. Beginning with version 6, a formal revision management system was implemented to track changes to the CFAST source code. The open-source program development tools are provided by GitHub: \href{https://github.com}{\textct{https://github.com}}.

The current version of CFAST, version 7, was released in 2015.

A Wiki file, located on the CFAST web site, maintained by the developers, details the changes made for each release: \href{https://github.com/firemodels/cfast/wiki/Release-Notes}{\textct{https://github.com/firemodels/cfast/wiki/Release-Notes}}.


\section{Model Developers}

CFAST was developed and is maintained by the Fire Research Division of the National Institute of Standards and Technology. The developers are Richard Peacock, Glenn Forney, and Paul Reneke. Kevin McGrattan has participated in the changes leading to CFAST, version 7.

\section{Relevant Publications}

The manuals for CFAST consist of this Technical Reference Guide, a User's Guide~\cite{CFAST_Users_Guide_7}, a Verification and Validation Guide~\cite{CFAST_Valid_Guide_7}, and a Configuration Management Guide~\cite{CFAST_Config_Guide_7}.  The Technical Reference Guide describes the underlying physical principles. The User's Guide describes how to use the model. The Verification and Validation Guide documents sensitivity analyses, model verification, model validation, and model limitations consistent with ASTM~E1355~\cite{CFAST:ASTM:E1355}. The Configuration Management Guide documents the processes used during the development and validation of the model.

The U.S. Nuclear Regulatory Commission has published a verification and validation study of five selected fire models commonly used in support of risk-informed and performance-based fire protection at nuclear power plants~\cite{NRCNUREG1824}. In addition to an extensive study of the CFAST model, the report compares the output of several other models ranging from simple hand calculations to more complex computational fluid dynamics codes such as the Fire Dynamics Simulator (FDS) developed by the National Institute of Standards and Technology (NIST) \cite{FDS_Tech_Guide_6}.


\section{Governing Equations and Assumptions}

The governing equations of CFAST are for conservation of mass and energy within the lower and upper layers of connected compartments within a building. The momentum within any one zone is assumed to be zero.  Momentum between zones in adjacent compartments is accounted for in terms of horizontal or vent flow equations (Bernoulli's law). Other features of CFAST include:
\begin{itemize}
\item Compartment geometry: CFAST is generally limited to fire scenarios where the compartment volumes are strongly stratified. The empirical correlations contained in CFAST were developed for relatively uncluttered, flat ceilings in compartments that can be characterized as ``rooms'' as opposed to corridors or vertical shafts. There are no hard limits on what kind of compartment can or cannot be modeled in CFAST. The CFAST Validation Guide indicates the accuracy of its predictions for compartments of various aspect ratios.
\item Heat Release Rate: CFAST does not predict fire growth on burning objects. The heat release rate is specified by the user for one or more fires. There is a simple sub-model to limit the heat release based on available oxygen.
\item Radiation from fires is modeled with a simple point source approximation.  This limits the accuracy of the model within a few diameters of the fire. Calculation of radiative exchange between compartments is not modeled.
\item Mechanical ventilation is modeled by specifying volumetric flow rates into or out of compartments. The overall HVAC (heating, ventilation, air conditioning) system is not modeled.
\item Natural Ventilation and Leakage: The flow through vertical openings, like doors and windows, is modeled using the Bernoulli equation for the pressure difference between two compartments. Horizontal openings, like hatches, are treated with a single empirical correlation based on pressure and density differences between upper and lower compartments. Leakage is modeled by explicitly creating a small vertical or horizontal opening.
\item Suppression: CFAST predicts sprinkler activation based on an empirical ceiling jet correlation and activation model. A simple suppression model decreases the specified heat release rate.
\end{itemize}
The technical approach and assumptions of the model have been presented in the peer-reviewed scientific literature~\cite{Jones:1993a, Jones:1985, Jones:1984} and conference proceedings~\cite{Jones:1991}. CFAST has been reviewed and included in industry-standard handbooks such as the SFPE Handbook~\cite{Walton:2003} and referenced in specific standards, including NFPA~805~\cite{NFPA805:2004} and NFPA~551~\cite{NFPA551:2004}.

Also, all documents released by NIST are required to go through an internal editorial review and approval process. This process is designed to ensure compliance with the technical requirements, policy, and editorial quality required by NIST. The technical review includes a critical evaluation of the technical content and methodology, statistical treatment of data, uncertainty analysis, use of appropriate reference data and units, and bibliographic references. CFAST manuals are always first reviewed by a member of the Fire Research Division, then by the immediate supervisor of the author of the document, then by the chief of the Fire Research Division, and finally by a reader from outside the division. These reviewers are technical experts in the field. Once the document has been reviewed, it is then brought before the Editorial Review Board (ERB), a body of representatives from all the NIST laboratories. At least one reader is designated by the Board for each document that it accepts for review. This last reader is selected based on technical competence and impartiality. The reader is usually from outside the division producing the document and is responsible for checking that the document conforms with NIST policy on units, uncertainty and scope. This reader does not need to be a technical expert in fire or combustion.

Besides formal internal and peer review, CFAST is subjected to continuous scrutiny because it is available to the general public and is used internationally by those involved in fire safety design and postfire reconstruction. The source code for CFAST is also released publicly, and has been used at various universities worldwide, both in the classroom as a teaching tool as well as for research. As a result, flaws in the theoretical development and the computer program itself have been identified and fixed. The user base continues to serve as a means to evaluate the model, which is as important to its development as the formal internal and external peer review processes.

For each major release of CFAST, NIST has maintained a history of the source code which goes back to March 1989.  While it is not practical to reconstruct the programs for each release for use with modern software tools and computer operating systems, the source code history allows the developers to examine what changes were made at each release point. This provides detailed documentation of the history of model development and is often useful to understand the impact of changes to sub-models as the model continues to evolve.




\section{Input Data Required to Run the Model}

All of the data required to run the CFAST model reside in a single input file that the user generates. The file consists of the following information:
\begin{itemize}
\item material properties (e.g., thermal conductivity, specific heat, density, thickness, heat of combustion)
\item compartment dimensions (height, width, length)
\item construction materials of the compartment (e.g., concrete, gypsum)
\item dimensions and positions of horizontal and vertical flow openings such as doors, windows, and vents
\item mechanical ventilation specifications
\item fire properties (e.g., heat release rate, lower oxygen limit, and species production rates as a function of time)
\item sprinkler and detector specifications
\item positions, sizes, and characteristics of targets
\item specifications for visual output from the model
\end{itemize}
The input files are provided for the validation exercises described in the Validation Guide~\cite{CFAST_Valid_Guide_7}. A complete description of the input parameters can be found in the CFAST User's Guide~\cite{CFAST_Users_Guide_7}.

A comprehensive assessment of the numerical parameters (such as default time step or solution convergence criteria) and physical parameters (such as empirical constants for convective heat transfer or plume entrainment) used in CFAST is not available in one document. Instead, specific parameters have been tested in various verification and validation studies performed at NIST and elsewhere. Numerical parameters are described in this Technical Reference Guide and are subject to the internal review process at NIST, but many physical parameters are extracted from the literature and do not undergo a formal review. The model user is expected to assess the appropriateness of default values provided by CFAST and make changes to the default values, if needed.



\section{Model Results}

The output of CFAST are the sensible variables that are needed for assessing the environment in a building subjected to a fire. Once the simulation is complete, CFAST produces an output file containing all of the solution variables.  Typical outputs include (but are not limited to) the following:
\begin{itemize}
\item environmental conditions in the room (such as hot gas layer temperature; plume centerline temperature; oxygen and smoke concentration; and ceiling, wall, and floor temperatures)
\item heat transfer-related outputs to walls and targets (such as incident convective, radiative, and total heat fluxes)
\item fire intensity and flame height
\item flow velocities through vents and openings
\item detector and sprinkler activation times
\end{itemize}


\section{Model Scenarios}

While the governing transport equations are based on the fundamental conservation laws of mass and energy, the fire-specific algorithms within CFAST are based on empirical correlations. These correlations include fire plume and ceiling jet temperatures and velocities, vent flow rates, sprinkler activation, and so on. These sub-models were developed independently of each other under ideal conditions. CFAST combines these sub-models in such a way that there are no hard limits on when a particular sub-model is appropriate or not. The decision as to whether CFAST is appropriate for a given fire scenario is based primarily on the hundreds of experiments and thousands of point-to-point comparisons between CFAST and measured quantities that are included in the CFAST Validation Guide~\cite{CFAST_Valid_Guide_7}. This document includes a list of the experiments and their important physical attributes such as the nature of the fire, the aspect ratio of the compartment, the ventilation rate, and the relative location of targets. For each quantity of interest, such as upper layer temperature or target heat flux, there is a calculated bias factor and standard deviation that indicates the accuracy of the model for the particular quantity of interest which is based on measurement uncertainty. Thus, the CFAST Validation Guide indicates what fire scenarios are appropriate for CFAST, and the degree of accuracy that can be expected for a particular type of prediction.

In addition to what is included in the CFAST Validation Guide, validation studies have been performed by NIST grantees, students at universities, and engineering firms using the model.  Because each organization has its  own reasons for  validating the model, the  referenced papers and reports do not follow any particular guidelines. Some of the works only provide  a qualitative assessment  of the model,  concluding that the  model  agreement with  a  particular  experiment  is ``good''  or ``reasonable.'' Sometimes, the conclusion is that the model works well in certain cases, not as well in others. These studies are included in the survey because the references  are useful to other model users who may have a similar application  and are interested in qualitative assessment. It is important to note  that some of the papers point out flaws in early releases of CFAST that have been corrected or improved in more recent  releases. Some of  the issues raised, however,  are still subjects of  active research. Continued updates for CFAST  are greatly influenced by the feedback provided by users, often through publication of validation efforts. 