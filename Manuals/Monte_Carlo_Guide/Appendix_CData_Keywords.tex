\chapter{CData/CFAST Text-based Input File}

The operation of CData and CFAST is based on a single ASCII\footnote{ASCII -- American Standard Code for Information Interchange. There are 256 characters that make up the standard ASCII text.} text file containing parameters organized into {\em namelist}\footnote{A {\em namelist} is a Fortran input record.} groups.
The input file provides CFAST  with all of the necessary information to describe the scenario. The graphical user interface, CEdit, writes this file. This appendix details all the parameters, which are organized into groups that roughly coincide with the tabs in the graphical user interface.

\section{Naming the Input File}

The input file is saved with a name such as {\ct job\_name.in}, where {\ct job\_name} is any character string that helps to identify the simulation. All of the output files associated with the calculation will then have this common prefix name.

There should be no blank spaces in the job name. Instead use the underscore character to represent a space.

Be aware that CData and CFAST will simply over-write the output files of a given case if its assigned name is the same. This is convenient when developing an input file because you save on disk space. Just be careful not to overwrite a calculation that you want to keep.

\section{Namelist Formatting}

Parameters are specified within the input file by using {\em namelist} formatted records. Each namelist record begins with the ampersand character, {\ct \&}, followed immediately by the name of the namelist group, then a comma-delimited list of the input parameters, and finally a forward slash, {\ct /}. For example, the line
\begin{lstlisting}
&TIME  SIMULATION = 3600., PRINT = 50., SMOKEVIEW = 50., SPREADSHEET = 50. /
\end{lstlisting}
sets various values of parameters contained in the {\ct TIME} namelist group. The meanings of these various parameters is explained in this guide. The namelist records can span multiple lines in the input file, but just be sure to end the record with a slash or else the data will not be understood. Do not add anything to a namelist line other than the parameters and values appropriate for that group. Otherwise, CFAST will stop immediately upon execution.

Parameters within a namelist record can be separated by either commas, spaces, or line breaks. It is recommended that you use commas or line breaks, and never use tab stops because they are not explicitly defined in the namelist data structure. CFAST and CEdit expect the first character of the file to be an ampersand, {\ct \&}, and by convention the first namelist is the {\ct HEAD} namelist but any namelist can be the first. Comments and notes can be written into the file between namelists so long as nothing comes before the ampersand except a space and nothing comes between the ampersand and the slash except appropriate parameters corresponding to that particular namelist group. However, it is important to note that any comments in an  input file that is opened by CEdit and saved will be lost.

The parameters in the input file can be integers, reals, character strings, or logical parameters. A logical parameter is either {\ct .TRUE.} or {\ct .FALSE.} -- the periods are a Fortran convention. Character strings that are listed in this User's Guide must be copied exactly as written -- the code is case sensitive and underscores {\em do} matter. The maximum length of most character input parameters is 60.

Most of the input parameters are simply real or integer scalars, like {\ct PRINT = 50.}, but sometimes the inputs can be arrays.

Note that character strings can be enclosed either by single or double quotation marks, however CEdit only recognizes the single quotation mark. Be careful not to create the input file by pasting text from something other than a simple text editor, in which case the punctuation marks may not transfer properly into the text file. Some text file encodings may not work on all systems. If file reading errors occur and no typographical errors can be found in the input file, try saving the input file using a different encoding. For example, the text file editor Notepad works fine on a Windows PC, but a file edited in Notepad may not work on Linux or Mac OS~X because of the difference in line endings between Windows and Unix/Linux operating systems. The editor Wordpad typically works better, but try a simple case first.


\section{Input File Structure}

In general, the namelist records can be entered in any order in the input file, but it is a good idea to organize them in some systematic way. Typically, general information is listed near the top of the input file, and detailed information, like obstructions, devices, and so on, are listed below. CFAST scans the entire input file each time it processes a particular namelist group. With some text editors, it has been noticed that the last line of the file is often not read by CFAST because of the presence of an ``end of file'' character. To ensure that CFAST reads the entire input file, add
\begin{lstlisting}
&TAIL /
\end{lstlisting}
as the last line at the end of the input file. This completes the file from {\ct \&HEAD} to {\ct \&TAIL}. CFAST does not even look for this last line. It just forces the ``end of file'' character past relevant input.

The general structure of an input file is shown below, with many lines of the original input file\footnote{The actual input file, Users\_Guide\_Example.in, is part of the CFAST software distribution} removed for clarity.

%\clearpage
\begin{lstlisting}[basicstyle=\tiny]
&HEAD VERSION = 7600, TITLE = 'CFAST Simulation' /
&MHDR NUMBER_OF_CASES = 3 WRITE_RANDOM_SEEDS = .TRUE. /

!! Scenario Configuration
&TIME SIMULATION = 3600 PRINT = 60 SMOKEVIEW = 15 SPREADSHEET = 15 /
&INIT PRESSURE = 101325 RELATIVE_HUMIDITY = 50 INTERIOR_TEMPERATURE = 20 EXTERIOR_TEMPERATURE = 20 /

!! Material Properties
&MATL ID = 'CONCRETE' MATERIAL = 'Concrete Normal Weight (6 in)',
  CONDUCTIVITY = 1.75 DENSITY = 2200 SPECIFIC_HEAT = 1, THICKNESS = 0.15 EMISSIVITY = 0.94 /

!! Compartments
&COMP ID = 'Comp 1'
  DEPTH = 3 HEIGHT = 3 WIDTH = 3
  CEILING_MATL_ID = 'CONCRETE' CEILING_THICKNESS = 0.15 WALL_MATL_ID = 'CONCRETE' WALL_THICKNESS = 0.15
  ORIGIN = 0, 0, 0 GRID = 50, 50, 50 /

!! Wall Vents
&MRND ID = 'Vent Width Generator' DISTRIBUTION_TYPE = 'UNIFORM' VALUE_TYPE = 'REAL'
  MINIMUM = 0.25 MAXIMUM = 2.0/
&MFLD ID = 'Wall Vent Width' FIELD_TYPE = 'VALUE' RAND_ID = 'Vent Width Generator' FIELD = 'Wall Vent' 'WIDTH' /

&MRND ID = 'Vent Height Generator' DISTRIBUTION_TYPE = 'UNIFORM' VALUE_TYPE = 'REAL'
  MINIMUM = 1.5 MAXIMUM = 2.5/
&MFLD ID = 'Wall Vent Height' FIELD_TYPE = 'VALUE' RAND_ID = 'Vent Height Generator' FIELD = 'Wall Vent' 'TOP' /

&VENT TYPE = 'WALL' ID = 'Wall Vent' COMP_IDS = 'Comp 1' 'OUTSIDE' , BOTTOM = 0 TOP = 2, WIDTH = 1
  FACE = 'FRONT' OFFSET = 1 /

!! Fires

&MRND ID = 'Peak HRR', DISTRIBUTION_TYPE = 'UNIFORM' VALUE_TYPE = 'REAL' MINIMUM = 500000 MAXIMUM = 3000000 /
&MRND ID = 'End of Fire HRR' DISTRIBUTION_TYPE = 'CONSTANT' VALUE_TYPE = 'REAL' REAL_CONSTANT_VALUE = 0 /
&MRND ID = 'Peak HRR Time Interval' DISTRIBUTION_TYPE = 'CONSTANT' VALUE_TYPE = 'REAL' REAL_CONSTANT_VALUE = 900 /
&MRND ID = 'Fire Time Interval' DISTRIBUTION_TYPE = 'CONSTANT' VALUE_TYPE = 'REAL' REAL_CONSTANT_VALUE = 10 /

&MFIR ID = 'Fire_generator' FIRE_ID = 'Fire' FIRE_TIME_GENERATORS = 'Fire Time Interval'
    'Peak HRR Time Interval' 'Fire Time Interval' FIRE_HRR_GENERATORS = 'Peak HRR' 'Peak HRR' 'End of Fire HRR' /

&FIRE ID = 'Fire'  COMP_ID = 'Comp 1', FIRE_ID = 'Constant Fire'  LOCATION = 1.5, 1.5 /
&CHEM ID = 'Constant Fire' CARBON = 1 CHLORINE = 0 HYDROGEN = 4 NITROGEN = 0 OXYGEN = 0
   HEAT_OF_COMBUSTION = 50000 RADIATIVE_FRACTION = 0.35 /
&TABL ID = 'Constant Fire' LABELS = 'TIME', 'HRR' , 'HEIGHT' , 'AREA' , 'CO_YIELD' , 'SOOT_YIELD' , 'HCN_YIELD' , 'HCL_YIELD' , 'TRACE_YIELD'  /
&TABL ID = 'Constant Fire', DATA = 0, 0, 0, 0, 0, 0, 0, 0, 0 /
&TABL ID = 'Constant Fire', DATA = 10, 100, 0, 0.113798159261744, 0, 0, 0, 0, 0 /
&TABL ID = 'Constant Fire', DATA = 990, 100, 0, 0.113798159261744, 0, 0, 0, 0, 0 /
&TABL ID = 'Constant Fire', DATA = 1000, 0, 0, 0, 0, 0, 0, 0, 0 /

!! Special Outputs

&DUMP ID = 'Time to Upper Layer 600 C' FILE = 'COMPARTMENTS' TYPE = 'TRIGGER_GREATER'
    FIRST_FIELD = 'Time' 'Simulation Time' SECOND_FIELD = 'Comp 1' 'Upper Layer Temperature' CRITERION = 600 /
&DUMP ID = 'Actual HRR at Upper Layer 600 C' FILE = 'COMPARTMENTS' TYPE = 'TRIGGER_GREATER'
    FIRST_FIELD = 'Fire' 'HRR Actual' SECOND_FIELD = 'Comp 1' 'Upper Layer Temperature' CRITERION = 600 /
&DUMP ID = 'Time to Layer Height 1.5 m' FILE = 'COMPARTMENTS' TYPE = 'TRIGGER_LESSER'
    FIRST_FIELD = 'Time' 'Simulation Time' SECOND_FIELD = 'Comp 1' 'Layer Height' CRITERION = 1.5 /
&DUMP ID = 'Maximum Upper Layer Temp' FILE = 'COMPARTMENTS' TYPE = 'MAXIMUM'
    FIRST_FIELD = 'Comp 1' 'Upper Layer Temperature' /
&DUMP ID = 'Minimum Layer Height' FILE = 'COMPARTMENTS' TYPE = 'MINIMUM'
    FIRST_FIELD = 'Comp 1' 'Layer Height' /
&DUMP ID = 'Maximum Actual HRR' FILE = 'COMPARTMENTS' TYPE = 'MAXIMUM'
    FIRST_FIELD = 'Fire' 'HRR Actual' /

&MSTT ID = 'Width of Vent' FYI = 'I am a firm believer in checking the inputs to make sure that you are getting what you expect'
    ANALYSIS_TYPE = 'HISTOGRAM' OUTPUT_FILENAME = 'simple_width.jpg' COLUMN_TITLE = 'Wall Vent_WIDTH' /
&MSTT ID = 'Top of Vent' FYI = 'I am a firm believer in checking the inputs to make sure that you are getting what you expect'
    ANALYSIS_TYPE = 'HISTOGRAM' OUTPUT_FILENAME = 'simple_top.jpg' COLUMN_TITLE = 'Wall Vent_TOP' /
&MSTT ID = 'Peak HRR' FYI = 'I am a firm believer in checking the inputs to make sure that you are getting what you expect'
    ANALYSIS_TYPE = 'HISTOGRAM' OUTPUT_FILENAME = 'simple_peak_hrr.jpg' COLUMN_TITLE = 'Fire_HRR_PT  2' /
&MSTT ID = 'Time to FO' FYI = 'I am a firm believer in checking the inputs to make sure that you are getting what you expect'
    ANALYSIS_TYPE = 'HISTOGRAM' OUTPUT_FILENAME = 'Simple_Time_to_FO.jpg' COLUMN_TITLE = 'Time to Upper Layer 600 C' /
&MSTT ID = 'Max Upper Temp' FYI = 'I am a firm believer in checking the inputs to make sure that you are getting what you expect'
    ANALYSIS_TYPE = 'HISTOGRAM' OUTPUT_FILENAME = 'Simple_MaxUpperTemp.jpg' COLUMN_TITLE = 'Maximum Upper Layer Temp' /
&MSTT ID = 'Convergence of Layer Height Reaching 1.5' OUTPUT_FILENAME = 'Simple_time_to_1p5.jpg' ANALYSIS_TYPE = 'CONVERGENCE_OF_MEAN'
    COLUMN_TITLE = 'Time to Layer Height 1.5 m' /
&MSTT ID = 'Convergence of Max Temp' OUTPUT_FILENAME = 'Simple_max_temp.jpg' ANALYSIS_TYPE = 'CONVERGENCE_OF_MEAN'
    COLUMN_TITLE = 'Maximum Upper Layer Temp' /
&MSTT ID = 'Correlation Tree on Temp' OUTPUT_FILENAME = 'Simple_tree_temp.jpg' ANALYSIS_TYPE = 'CORRELATION_TREES'
    COLUMN_TITLE = 'Maximum Upper Layer Temp' /

&TAIL /
\end{lstlisting}
It is recommended that when looking at a new scenario, first select a pre-written input file that resembles the case, make the necessary changes, then run the case to determine if the geometry is set up correctly. It is best to start off with a relatively simple file that captures the main features of the problem without getting tied down with too much detail that might mask a fundamental flaw in the calculation. As you learn how to write input files, you will continually run and re-run your case as you add in complexity.

Table~\ref{tbl:namelistgroups} provides a quick reference to all the namelist parameters used by CData and where you can find the reference to where it is introduced in the document and the table containing all of the keywords for each group.


\begin{table}[ht]
\begin{center}
\caption{CFAST Input File Keywords}
\label{tbl:namelistgroups}
\begin{tabular}{|c|l|c|c|}
\hline
Group Name   & Namelist Group Description       & Reference Section & Parameter Table  \\ \hline
{\ct MFIR}   & Monte Carlo fire definitions     & \ref{info:MFIR}   & \ref{tbl:MFIR}   \\ \hline
{\ct MFLD}   & Monte Carlo field definitions    & \ref{info:MFLD}   & \ref{tbl:MFLD}   \\ \hline
{\ct MHDR}   & Monte Carlo header               & \ref{info:MHDR}   & \ref{tbl:MHDR}   \\ \hline
{\ct MRND}   & Monte Carlo random distributions & \ref{info:MRND}   & \ref{tbl:MRND}   \\ \hline
{\ct MSTT}   & Monte Carlo statistical outputs  & \ref{info:MSTT}   & \ref{tbl:MSTT}   \\ \hline
\end{tabular}
\end{center}
\end{table}

Examples of each of the Monte Carlo inputs are included in the sections that follow. A complete list of all CFAST inputs are included in the CFAST User's Guide.  All examples are taken from the sample input file {\ct Simple.in} listed above. Following are some general rules about the CFAST input file:

\begin{itemize}
\item The {\ct HEAD} input identifies the version of CFAST for which the input file was created and is typically the first line in the input file. Use {\ct \&TAIL} as the last line at the end of the input file. This completes the file from {\ct \&HEAD} to {\ct \&TAIL}. CFAST does not even look for this last line. It just forces the “end of file” character past relevant input.
\item Many of the listed keywords are mutually exclusive. Repeated entry of some keywords can cause the program to either fail or run in an unpredictable manner.
\item Use of some keywords triggers the code to operate in a certain mode/condition. For example, specifying {\ct ADIABATIC} to be {\ct .TRUE.} triggers the code to treat all compartment surfaces to be perfectly insulated.
\item Multiple inputs are required whenever the keyword is in plural form --- keywords ending with an \textbf{s}. For example, the keyword parameter, {\ct TEMPERATURES}, within the namelist group, {\ct INIT}, requires two temperature values (in this case, one for exterior ambient temperature and one for interior ambient temperature). In the case of missing inputs, an error message will be generated to assist users in troubleshooting any errors.
\item Default values to inputs are assigned to some of the keywords to facilitate the set up of an input file. Users should review the applicability of any default values for their simulation.
\end{itemize}


\clearpage

\section{Monte Carlo Header, Namelist Group \texorpdfstring{{\tt MHDR}}{MHDR}}

\noindent
\renewcommand{\tabcolsep}{.1in}
\begin{longtable}{|l|l|l|l|l|}
\caption[Monte Carlo Header Parameters ({\ct MHDR} namelist group)]{For more information see Section~\ref{info:MHDR}.}
\label{tbl:MHDR} \\
\hline
\multicolumn{5}{|c|}{{\ct MHDR} (Monte Carlo Header Parameters)} \\
\hline \hline
\endfirsthead
\caption[]{Continued} \\
\hline
\multicolumn{5}{|c|}{{\ct MHDR} (Monte Carlo Header Parameters)} \\
\hline \hline
\endhead
\small
\parbox{1.5in}{\bf Parameter}    & \parbox{1in}{\bf Type}  & \parbox{1in}{\bf Reference}  & \parbox{1in}{\bf Units}  & \parbox{1in}{\bf Default Value} \\ \hline
{\ct NUMBER\_OF\_CASES}          & Integer        & Section \ref{info:MHDR}   &     &                          \\ \hline
{\ct OUTPUT\_DIRECTORY}          & Character      & Section \ref{info:MHDR}   &     & Local directory          \\ \hline
{\ct PARAMETER\_FILENAME}        & Character      & Section \ref{info:MHDR}   &     & 'filename'\_parameters   \\ \hline
{\ct SEEDS}                      & Integer Pair   & Section \ref{info:MHDR}   &     &                          \\ \hline
{\ct WORK\_DIRECTORY}            & Character      & Section \ref{info:MHDR}   &     & Local directory          \\ \hline
{\ct WRITE\_SEEDS}               & Logical        & Section \ref{info:MHDR}   &     & {\ct .TRUE.}             \\ \hline
\end{longtable}

\noindent Examples:
\begin{lstlisting}
&MHDR NUMBER_OF_CASES = 3 WRITE_RANDOM_SEEDS = .TRUE. /
\end{lstlisting}


\clearpage

\section{Monte Carlo Random Input Generators, Namelist Group \texorpdfstring{{\tt MRND}}{MRND}}

\noindent
\renewcommand{\tabcolsep}{.1in}
\begin{longtable}{|l|l|l|l|l|}
\caption[Monte Carlo Random Generators ({\ct MRND} namelist group)]{For more information see Section~\ref{info:MRND}.}
\label{tbl:MRND} \\
\hline
\multicolumn{5}{|c|}{{\ct MRND} (Monte Carlo Random Generators)} \\
\hline \hline
\endfirsthead
\caption[]{Continued} \\
\hline
\multicolumn{5}{|c|}{{\ct MRND} (Monte Carlo Random Generators)} \\
\hline \hline
\endhead
\small
\parbox{1.5in}{\bf Parameter}    & \parbox{1in}{\bf Type}  & \parbox{1in}{\bf Reference}  & \parbox{1in}{\bf Units}  & \parbox{1in}{\bf Default Value} \\ \hline
{\ct ID}                & Character         & Section \ref{info:MRND}   &      &     \\ \hline
{\ct FYI}               & Character         & Section \ref{info:MRND}   &      &     \\ \hline
{\ct TYPE}              & Selection List    & Section \ref{info:MRND}   &      &     \\ \hline
{\ct RANDOM\_SEEDS}     & Integer Pair      & Section \ref{info:MRND}   &      &     \\ \hline
{\ct VALUES}            & Character array   & Section \ref{info:MRND}   &      &     \\ \hline
{\ct PROBABILITIES}     & Real array        & Section \ref{info:MRND}   &      &     \\ \hline
{\ct MINIMUM}           & Real              & Section \ref{info:MRND}   &      &     \\ \hline
{\ct MAXIMUM}           & Real              & Section \ref{info:MRND}   &      &     \\ \hline
{\ct PEAK}              & Real              & Section \ref{info:MRND}   &      &     \\ \hline
{\ct MEAN}              & Real              & Section \ref{info:MRND}   &      &     \\ \hline
{\ct STDEV}             & Real              & Section \ref{info:MRND}   &      &     \\ \hline
{\ct ALPHA}             & Real              & Section \ref{info:MRND}   &      &     \\ \hline
{\ct BETA}              & Real              & Section \ref{info:MRND}   &      &     \\ \hline
{\ct MINIMUM\_FIELD}    & Character Pair    & Section \ref{info:MRND}   &      &     \\ \hline
{\ct MAXIMUM\_FIELD}    & Character Pair    & Section \ref{info:MRND}   &      &     \\ \hline

\end{longtable}

\noindent Examples:
\begin{lstlisting}
&MRND ID = 'Vent Width Generator' DISTRIBUTION_TYPE = 'UNIFORM' VALUE_TYPE = 'REAL'
  MINIMUM = 0.25 MAXIMUM = 2.0/

&MRND ID = 'Fire Time Interval' DISTRIBUTION_TYPE = 'CONSTANT' VALUE_TYPE = 'REAL'
  REAL_CONSTANT_VALUE = 10 /
\end{lstlisting}


\clearpage

\section{Monte Carlo Input Field Generators, Namelist Group \texorpdfstring{{\tt MFLD}}{MFLD}}

\noindent
\renewcommand{\tabcolsep}{.1in}
\begin{longtable}{|l|l|l|l|l|}
\caption[Monte Carlo Input Field Generators ({\ct MFLD} namelist group)]{For more information see Section~\ref{info:MFLD}.}
\label{tbl:MFLD} \\
\hline
\multicolumn{5}{|c|}{{\ct MFLD} (Monte Carlo Input Field Generators)} \\
\hline \hline
\endfirsthead
\caption[]{Continued} \\
\hline
\multicolumn{5}{|c|}{{\ct MFLD} (Monte Carlo Input Field Generators)} \\
\hline \hline
\endhead
\small
\parbox{1.5in}{\bf Parameter}    & \parbox{1in}{\bf Type}  & \parbox{1in}{\bf Reference}  & \parbox{1in}{\bf Units}  & \parbox{1in}{\bf Default Value} \\ \hline
{\ct ID}                         & Character         & Section \ref{info:MFLD}   &      &     \\ \hline
{\ct FYI}                        & Character         & Section \ref{info:MFLD}   &      &     \\ \hline
{\ct TYPE}                       & Selection List    & Section \ref{info:MFLD}   &      &     \\ \hline
{\ct FIELD}                      & Character Pair    & Section \ref{info:MFLD}   &      &     \\ \hline
{\ct FIELD\_LABELS}              & Character         & Section \ref{info:MFLD}   &      &     \\ \hline
{\ct RAND\_ID}                   & Character         & Section \ref{info:MFLD}   &      &     \\ \hline
{\ct ADD\_TO\_PARAMETERS}        & Logical           & Section \ref{info:MFLD}   &      &     \\ \hline
{\ct PARAMETER\_COLUMN\_LABEL}   & Character         & Section \ref{info:MFLD}   &      &     \\ \hline
{\ct VALUES}                     & Character array   & Section \ref{info:MFLD}   &      &     \\ \hline
{\ct BASE\_SCALING\_VALUE}       & Real              & Section \ref{info:MFLD}   &      &     \\ \hline

\end{longtable}

\noindent Examples:
\begin{lstlisting}
&MFLD ID = 'Wall Vent Width' FIELD_TYPE = 'VALUE' RAND_ID = 'Vent Width Generator'
  FIELD = 'Wall Vent' 'WIDTH' /
\end{lstlisting}


\clearpage

\section{Monte Carlo Input Fire Generators, Namelist Group \texorpdfstring{{\tt MFIR}}{MFIR}}

\begin{landscape}
\noindent
\renewcommand{\tabcolsep}{.1in}
\begin{longtable}{|l|l|l|l|l|}
\caption[Monte Carlo Fire Generators ({\ct MFIR} namelist group)]{For more information see Section~\ref{info:MFIR}.}
\label{tbl:MFIR} \\
\hline
\multicolumn{5}{|c|}{{\ct MFIR} (Monte Carlo Fire Generators)} \\
\hline \hline
\endfirsthead
\caption[]{Continued} \\
\hline
\multicolumn{5}{|c|}{{\ct MFIR} (Monte Carlo Fire Generators)} \\
\hline \hline
\endhead
\parbox{1.5in}{\bf Parameter}    & \parbox{1in}{\bf Type}  & \parbox{1in}{\bf Reference}  & \parbox{1in}{\bf Units}  & \parbox{1in}{\bf Default Value} \\ \hline
{\ct ID}                                                    & Character         & Section \ref{info:MFIR}   &      &     \\ \hline
{\ct FYI}                                                   & Character         & Section \ref{info:MFIR}   &      &     \\ \hline
{\ct FIRE\_ID}                                              & character         & Section \ref{info:MFIR}   &      &     \\ \hline
{\ct MODIFY\_FIRE\_AREA\_TO\_MATCH\_HRR}                    & Logical           & Section \ref{info:MFIR}   &      &     \\ \hline
{\ct FIRE\_COMPARTMENT\_RANDOM\_GENERATOR\_ID}              & Character         & Section \ref{info:MFIR}   &      &     \\ \hline
{\ct FIRE\_COMPARTMENT\_IDS}                                & Character Array   & Section \ref{info:MFIR}   &      &     \\ \hline
{\ct ADD\_FIRE\_COMPARTMENT\_TO\_PARAMETERS}                & Logical           & Section \ref{info:MFIR}   &      &     \\ \hline
{\ct FIRE\_COMPARTMENT\_PARAMETER\_COLUMN\_LABEL}           & Character         & Section \ref{info:MFIR}   &      &     \\ \hline \hline
{\ct INCIPIENT\_TYPE\_RANDOM\_GENERATOR\_ID}                & Character         & Section \ref{info:MFIR}   &      &     \\ \hline
{\ct INCIPIENT\_FIRE\_TYPES}                                & Character array   & Section \ref{info:MFIR}   &      &     \\ \hline
{\ct INCIPIENT\_GROWTH\_TYPE}                               & Character         & Section \ref{info:MFIR}   &      &     \\ \hline
{\ct FLAMING\_INCIPIENT\_DELAY\_RANDOM\_GENERATOR\_ID}      & Character         & Section \ref{info:MFIR}   &      &     \\ \hline
{\ct FLAMING\_PEAK\_INCIPIENT\_RANDOM\_GENERATOR\_ID}       & Character         & Section \ref{info:MFIR}   &      &     \\ \hline
{\ct SMOLDERING\_INCIPIENT\_DELAY\_RANDOM\_GENERATOR\_ID}   & Character         & Section \ref{info:MFIR}   &      &     \\ \hline
{\ct SMOLDERING\_PEAK\_INCIPIENT\_RANDOM\_GENERATOR\_ID}    & Character         & Section \ref{info:MFIR}   &      &     \\ \hline
{\ct ADD\_INCIPIENT\_TYPE\_TO\_PARAMETERS}                  & Logical           & Section \ref{info:MFIR}   &      &     \\ \hline
{\ct ADD\_INCIPIENT\_TIME\_TO\_PARAMETERS}                  & Logical           & Section \ref{info:MFIR}   &      &     \\ \hline
{\ct ADD\_INCIPIENT\_PEAK\_TO\_PARAMETERS}                  & Logical           & Section \ref{info:MFIR}   &      &     \\ \hline \hline
{\ct BASE\_FIRE\_ID}                                        & Character         & Section \ref{info:MFIR}   &      &     \\ \hline
{\ct SCALING\_FIRE\_HRR\_RANDOM\_GENERATOR\_ID}             & Character         & Section \ref{info:MFIR}   &      &     \\ \hline
{\ct SCALING\_FIRE\_TIME\_RANDOM\_GENERATOR\_ID}            & Character         & Section \ref{info:MFIR}   &      &     \\ \hline
{\ct ADD\_HRR\_SCALE\_TO\_PARAMETERS}                       & Logical           & Section \ref{info:MFIR}   &      &     \\ \hline
{\ct HRR\_SCALE\_COLUMN\_LABEL}                             & Character         & Section \ref{info:MFIR}   &      &     \\ \hline
{\ct ADD\_TIME\_SCALE\_TO\_PARAMETERS}                      & Logical           & Section \ref{info:MFIR}   &      &     \\ \hline
{\ct TIME\_SCALE\_COLUMN\_LABEL}                            & Character         & Section \ref{info:MFIR}   &      &     \\ \hline \hline
{\ct FIRE\_TIME\_GENERATORS}                                & Character array   & Section \ref{info:MFIR}   &      &     \\ \hline
{\ct FIRE\_HRR\_GENERATORS}                                 & Character array   & Section \ref{info:MFIR}   &      &     \\ \hline
{\ct NUMBER\_OF\_GROWTH\_POINTS}                            & Integer           & Section \ref{info:MFIR}   &      &     \\ \hline
{\ct NUMBER\_OF\_DECAY\_POINTS}                             & Integer           & Section \ref{info:MFIR}   &      &     \\ \hline
{\ct GROWTH\_EXPONENT}                                      & Real              & Section \ref{info:MFIR}   &      &     \\ \hline
{\ct DECAY\_EXPONENT}                                       & Real              & Section \ref{info:MFIR}   &      &     \\ \hline
{\ct ADD\_HRR\_TO\_PARAMETERS}                              & Logical           & Section \ref{info:MFIR}   &      &     \\ \hline
{\ct HRR\_COLUMN\_LABEL}                                    & Character         & Section \ref{info:MFIR}   &      &     \\ \hline
{\ct ADD\_TIME\_TO\_PARAMETERS}                             & Logical           & Section \ref{info:MFIR}   &      &     \\ \hline
{\ct TIME\_COLUMN\_LABEL}                                   & Character         & Section \ref{info:MFIR}   &      &     \\ \hline


\end{longtable}
\end{landscape}
\noindent Examples:
\begin{lstlisting}
&MFLD ID = 'Wall Vent Width' FIELD_TYPE = 'VALUE' RAND_ID = 'Vent Width Generator'
  FIELD = 'Wall Vent' 'WIDTH' /
\end{lstlisting}


\clearpage



\chapter{Running CData from a Command Prompt}

CData is run from a Windows command prompt from the folder where the base input file is located.

\begin{lstlisting}
[drive1:\][folder1\]cdata project.in
\end{lstlisting}

The project name will have extensions appended as needed (see below). For example, to run a test case when the CFAST executable is located in c:$\backslash$firemodels$\backslash$cfast7 and the input data file is located in c:$\backslash$data, the following command could be used:

\begin{lstlisting}
c:\firemodels\cfast7\cfast c:\data\testfire0   <<< note that the extension is optional.
\end{lstlisting}

One or more command line options can follow the name of the file to be run as follows:

\begin{itemize}
\item -k - no interactive keyboard access
\item -i - initialization only
\item -c - compact output
\item -f - full output (c and f are exclusive). Note the interaction of the f and c option. The default for the console output is -c. The default for the file output is /f. This default action can be overwritten by explicitly including the /f or /c option.
\item -n - net heat flux option
\item -s - limit spreadsheet output to the files specified: C for \_compartments.csv, D for \_devices.csv, M for \_masses.csv, V for \_vents.csv, and W for \_walls.csv. Default is -S:CDMVW for output of all spreadsheet files.
\item -v - validation output (outputs a modified set of spreadsheet files with different column headers designed to facilitate automated analysis of the output)
\end{itemize}


\label{last_page}




