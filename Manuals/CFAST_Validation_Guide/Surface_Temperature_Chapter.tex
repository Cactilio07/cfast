\chapter{Surface Temperature}

Solid surfaces  in CFAST are treated as one-dimensional transient conduction problems. This chapter divides solid surfaces into two major categories -- compartment linings (i.e., walls, ceiling, floor) and targets (i.e., anything that is not a wall, ceiling, or floor). Heat transfer to the inside surface of compartment linings and the front and rear faces (as specified by the user) of targets consists of convection (through the use of empirical correlations) and radiation (calculated by the model using view factors for the fire, gas layers, and compartment surfaces). Heat conduction into a solid surface is calculated via a one-dimensional solution of the heat equation in cartesian or cylindrical coordinates.  The latter is particularly useful for predicting the thermal response of electrical cables.

For compartment linings, the ``outside'' surface is, by default, exposed to the exterior ambient temperature with convection and radiation calculated in a similar manner to the inside surface.  The ``outside'' boundary condition can also be specified as a constant temperature (i.e., the outside surface can be at ambient temperature) or can be connected to the ``outside'' surface of part or all of a second compartment.  For targets, the back surface is simply pointed in a direction opposite that of the front surface with convection and radiation calculated in a similar manner to the front surface.

\section{ Compartment Ceiling, Wall, and Floor Temperature}

In the NIST/NRC and WTC tests, thermocouples and heat flux gauges were positioned at various locations on all four walls of the test compartments, plus on the ceilings and floors. Over the course of 15 experiments, a number of the thermocouples and gauges failed, but because over half of the measurement points were in roughly the same relative location to the fire, the faulty data was discarded based on examining replicate experiments or locations on the opposite wall. Table~\ref{NIST_NRC_Wall_Coords} lists the measurement locations for each test. For each test, eight locations are used for comparison, two on the long (mainly north) wall, two on the short (east) wall, two on the floor, and two on the ceiling.  Of the two locations for each panel, one is considered in the far-field, relatively remote from the fire; one is in the near-field, relatively close to the fire.  How close or far varied from test to test, depending on the availability of working flux gauges. The two short wall locations are equally remote from the fire; thus, one location is in the lower layer, one in the upper.

\begin{table}[p]
\caption[Wall measurement positions for the NIST/NRC series]
{Wall thermocouple and heat flux gauge positions for the NIST/NRC series. The origin of the coordinate system lies on the floor in the southwest
corner of the compartment. The designation ``U'' and ``C'' is irrelevant, and the last digit ``2'' indicates that the thermocouple is measuring the wall temperature rather than the heat flux gauge temperature.}
\begin{center}
\begin{tabular}{|l|c|c|c||l|c|c|c|}
\hline
Name              & $x$   & $y$  & $z$      & Name              & $x$   & $y$   & $z$       \\ \hline \hline
TC North U-1-2    & 3.85  & 7.04 & 1.49     & TC South U-1-2    & 3.86  & 0     & 1.49      \\ \hline
TC North U-2-2    & 3.86  & 7.04 & 3.71     & TC South U-2-2    & 3.86  & 0     & 3.82      \\ \hline
TC North U-3-2    & 9.48  & 7.04 & 1.86     & TC South U-3-2    & 9.54  & 0     & 1.86      \\ \hline
TC North U-4-2    & 12.07 & 7.04 & 1.88     & TC South U-4-2    & 12.08 & 0     & 1.86      \\ \hline
TC North U-5-2    & 17.69 & 7.04 & 1.49     & TC South U-5-2    & 17.69 & 0     & 1.50      \\ \hline
TC North U-6-2    & 17.69 & 7.04 & 3.69     & TC South U-6-2    & 17.74 & 0     & 3.70      \\ \hline \hline
TC East U-1-2     & 21.66 & 1.52 & 1.12     & TC West U-1-2     & 0     & 1.59  & 1.12      \\ \hline
TC East U-2-2     & 21.66 & 1.52 & 2.40     & TC West U-2-2     & 0     & 1.59  & 2.42      \\ \hline
TC East U-3-2     & 21.66 & 5.68 & 1.13     & TC West U-3-2     & 0     & 5.70  & 1.12      \\ \hline
TC East U-4-2     & 21.66 & 5.70 & 2.42     & TC West U-4-2     & 0     & 5.70  & 2.42      \\ \hline \hline
TC Floor U-1-2    & 3.08  & 3.51 & 0        & TC Ceiling U-1-2  & 3.04  & 3.60  & 3.82      \\ \hline
TC Floor U-2-2    & 9.08  & 1.94 & 0        & TC Ceiling C-2-2  & 8.99  & 2.00  & 3.82      \\ \hline
TC Floor U-3-2    & 9.06  & 5.97 & 0        & TC Ceiling C-3-2  & 9.03  & 5.97  & 3.82      \\ \hline
TC Floor U-4-2    & 10.86 & 2.38 & 0        & TC Ceiling C-4-2  & 10.79 & 2.38  & 3.82      \\ \hline
TC Floor C-5-2    & 10.93 & 5.20 & 0        & TC Ceiling C-5-2  & 10.79 & 5.20  & 3.82      \\ \hline
TC Floor U-6-2    & 13.13 & 1.99 & 0        & TC Ceiling C-6-2  & 13.00 & 2.07  & 3.82      \\ \hline
TC Floor U-7-2    & 13.00 & 5.92 & 0        & TC Ceiling C-7-2  & 12.84 & 5.98  & 3.82      \\ \hline
TC Floor U-8-2    & 18.63 & 3.54 & 0        & TC Ceiling U-8-2  & 18.71 & 3.54  & 3.82      \\ \hline
\end{tabular}
\end{center}
\label{NIST_NRC_Wall_Coords}
\end{table}

The WTC test measured ceiling temperatures, both at the surface and beneath a layer of marinite board. Table~\ref{WTC_Ceiling_Coords} below lists the coordinates of the measurement locations relative to the center of the fire pan. Names with ``IN'' appended are measurements made between two marinite boards which lined the compartment surfaces.

\begin{table}[h!]
\caption[Ceiling surface measurement locations for the WTC series]{Locations of ceiling surface temperature measurements relative to the fire pan in the WTC series.}
\begin{center}
\begin{tabular}{|l|c|c|c|}
\hline
Name                & $x$ (m)   & $y$ (m)   & $z$ (m)   \\ \hline \hline
TCC                 & 0.62      & 0.07      & 3.82      \\ \hline
TCN3                & 0.62      & 0.67      & 3.82      \\ \hline
TCS3                & 0.62      & -0.53     & 3.82      \\ \hline
TCE7                & 2.18      & 0.07      & 3.82      \\ \hline
TCW7                & -1.15     & 0.07      & 3.82      \\ \hline \hline
TCCIN               & 0.62      & 0.07      & 3.83      \\ \hline
TCN3IN              & 0.62      & 0.67      & 3.83      \\ \hline
TCS3IN              & 0.62      & -0.53     & 3.83      \\ \hline
TCE4IN              & 1.28      & 0.07      & 3.83      \\ \hline
TCW4IN              & 0.05      & 0.07      & 3.83      \\ \hline
\end{tabular}
\end{center}
\label{WTC_Ceiling_Coords}
\end{table}

Figure \ref{fig:Surface_Temperature_Scatter} shows a comparison of predicted and measured values for total heat flux. Appendix B provides comparisons of heat flux and surface temperature on cable and surface targets.
\label{Surface Temperature}
\label{Wall Temperature}
\label{Ceiling Temperature}
\label{Floor Temperature}

\begin{figure}
\begin{center}
\includegraphics[height=4in]{SCRIPT_FIGURES/ScatterPlots/Surface_Temperature}
\end{center}
\caption[Summary, Compartment Surface Temperature]{Summary, Compartment Surface Temperature.} \label{fig:Surface_Temperature_Scatter}
\end{figure}


\section{Target Temperature}

Target temperature and heat flux data are available from the NIST/NRC test series.  In the NIST/NRC tests, the targets are different types of cables in various configurations: horizontal, vertical, in trays, or free-hanging. Since these tests are intended to represent electrical cables, they are modeled as cylindrical targets. Targets in the SP AST and WTC tests, intended to represent various components of the building structure, are modeled as normal thermally-thick targets in CFAST.

The SP Adiabatic Surface Temperature Experiments included measurements of gas, plate thermometer, and steel temperatures for compartment and pool fire experiments conducted at SP, Sweden. Only the compartment fire experiments are included in the CFAST comparisons. Three additional experiments were conducted at SP, Sweden, in 2011, in which a 6~m long, 20~cm diameter vertical column was positioned in the middle of 1.1~m and 1.9~m diesel and 1.1~m heptane pool fires~\cite{Sjostrom:AST}. Gas, plate, and steel surface temperature measurements were made at heights of 1~m, 2~m, 3~m, 4~m, and 5~m above the pool surface. At heights of 1~m, 3~m, and 5~m, these measurements were made at only one angular position. However, at 2~m and 4~m, the measurements were made at four positions. At these heights, two conventional plates thermometers were positioned approximately 10~cm from the column surface, along with two special plate thermometers (SPT) that were installed flush with the column surface. At each height, comparable predictions were made with CFAST using normal, thermally-thick targets.

The compartment for the WTC experiments contained a hollow box column roughly 0.5~m from the fire pan, two trusses over the top of the pan, and one or two steel bars resting on the lower truss flanges. In Tests 1, 2 and 3, the steel was bare, and in Tests 4, 5 and 6, the steel was coated with various thicknesses of sprayed fire-resistive materials. The column was instrumented near its base (about 0.5~m from the floor, middle (1.5~m), and upper (2.5~m). Four measurements of steel (and insulation) temperatures were made at each location, for each of its four sides. These elements were modeled using thin sheet obstructions with a resolution of 10~cm. In addition to the steel structural elements, five cylinders (``slugs'') of nickel~200 ($\ge$ 99~\% nickel), 25.4~cm long and 10.2~cm in diameter, were positioned 50~cm north of the centerline in the WTC experiments. Slugs 1 through 5 were located 2.92~m, 1.82~m, 0.57~m, 0.05~m, and 1.56~m, respectively, from the longitudinal axis of the fire pan. All the slugs were 50~cm north of the lateral axis. The fire pan measured 2~m by 1~m. Four thermocouples were inserted into each slug at various locations. All four temperatures for each slug were virtually indistinguishable.Targets were modeled with normal thermally-thick targets in CFAST.

Figure \ref{fig:Target_Temperature_Scatter} shows a comparison of predicted and measured values for total heat flux. Appendix B provides comparisons of heat flux and surface temperature on cable and surface targets.
\label{Target Temperature}

\begin{figure}
\begin{center}
\includegraphics[height=4in]{SCRIPT_FIGURES/ScatterPlots/Target_Temperature}
\end{center}
\caption[Summary, Target Temperature]{Summary, Target Temperature.} \label{fig:Target_Temperature_Scatter}
\end{figure}



