\chapter{Structure of the CFAST Input File} \label{sec:CFAST_Keywords}



The purpose of this appendix is to describe the format of the text-based input file that is generated by the CEdit graphical user interface. A simple input file is listed below. The input values may be integers, real numbers, or text. The input file is a comma-separated ASCII text file and may be edited with a spreadsheet program or any text editor. It is possible to use a word processor but it is important to save the file in ASCII text format and not in a word processing format. Note that some word processors will save punctuation and other characters incorrectly for the simple ASCII text file used by CFAST. It is recommended that the input files be created with the input editor, CEdit, provided as part of the CFAST distribution.  In addition to checking the input data for errors, it includes typical ranges for input values to assist in appropriate use of the model.

Each line of the input file consists of a label followed by one or more alphanumeric parameters associated with that input label, separated by commas.  The label must always begin in the first space of the line and be in capital letters.  Following the label, the values may start in any column, and all values must be separated by a comma.  Values may contain decimal points if needed or desired.  They are not required.

Inputs are in standard SI units.  The maximum line length is 1024 characters, so all data for each keyword must fit in this number of characters.

\begin{lstlisting}
VERSN,7,Default example fire for user guide
!!
!!Scenario Configuration Keywords
!!
TIMES,1800,-120,10,30
EAMB,293.15,101300,0
TAMB,293.15,101300,0,50
!!
!!Material Properties
!!
MATL,CONCRETE,1.75,1000,2200,0.15,0.94,"Concrete, Normal Weight (6 in)"
MATL,GLASSFB3,0.036,795,105,0.013,0.9,"Glass Fiber, Organic Bonded (1/2 in)"
!!
!!Compartment keywords
!!
COMPA,Comp 1,9.1,5,4.6,0,0,0,GLASSFB3,CONCRETE,CONCRETE,50,50,50
!!
!!Vent keywords
!!
HVENT,1,2,1,1,2.4,0,0,1,1
!!
!!Fire keywords
!!
GLOBA,10,393.15
!!burner
FIRE,1,4.55,2.5,0,1,1,0,0,0,1,burner
CHEMI,1,4,0,0,0,0.33,5E+07,METHANE
TIME,0,60,120,180,240,300,360,420,480,540,1800
HRR,0,100000,150000,200000,150000,125000,100000,90000,80000,75000,75000
SOOT,0,0,0,0,0,0,0,0,0,0,0
CO,0.001,0.001,0.001,0.001,0.001,0.001,0.001,0.001,0.001,0.001,0.001
TRACE,0,0,0,0,0,0,0,0,0,0,0
AREA,0.2,0.2,0.2,0.2,0.2,0.2,0.2,0.2,0.2,0.2,0.2
HEIGH,0,0,0,0,0,0,0,0,0,0,0
MATL,METHANE,0.07,1090,930,0.0127,0.04,"Methane, a transparent gas (CH4)"
!!wood_wall
FIRE,1,9.1,2.5,0,1,1,0,0,0,1,wood_wall
CHEMI,6,10,5,0,0,0.33,1.81E+07,HARDWOOD
TIME,0,8000
HRR,0,1000000
SOOT,0.015,0.015
CO,0.006171682,0.006171682
TRACE,0,0
AREA,0.05,9
HEIGH,0,3
MATL,HARDWOOD,0.16,1255,720,0.019,0.9,"Wood, Hardwoods (oak, maple) (3/4 in)"
!!
!!Target and detector keywords
!!
TARGET,1,2.2,1.88,2.34,0,0,1,CONCRETE,IMPLICIT,PDE,0.5
!!
!!visualizations
!!
SLCF,2-D,Y,2.5,1
SLCF,2-D,Z,4.554,1
\end{lstlisting}


\section{COMPA}

\begin{lstlisting}
COMPA, NAME, WIDTH, DEPTH, INTERNAL_HEIGHT, ABSOLUTE_X_POSITION, ABSOLUTE_Y_POSITION, FLOOR_HEIGHT, CEILING_MATERIAL_NAME,  FLOOR_MATERIAL_NAME, WALL_MATERIAL_NAME, X_GRID_SPACING, Y_GRID_SPACING, Z_GRID_SPACING
\end{lstlisting}
The compartments are numbered internally as they are read in. The other key words which refer to compartment numbers then refer to these ordinals. Compartments must be defined before they are referenced by other commands. Example:
\begin{lstlisting}
COMPA,hallway,9.1,5.0,4.6,0.,0.,0.,CONCRETE,CONCRETE,CONCRETE,50,50,50
\end{lstlisting}

\section{DEADROOM}

\begin{lstlisting}
DEADROOM, DEAD_ROOM_NUMBER, CONNECTED_ROOM_NUMBER
\end{lstlisting}
The DEADROOM command specifies a compartment that is only connected to another compartment that in turn is not connected to the outside.  Compartment pressure is not calculated for the ``dead'' room. Pressure for this room is assumed to be the same as that of the connected room.

\section{DETEC}

\begin{lstlisting}
DETEC, TYPE, COMPARTMENT, ACTIVATION_TEMPERATURE,E DEPTH, WIDTH, HEIGHT, RTI, SUPPRESSION, SPRAY_DENSITY
\end{lstlisting}
The DETEC keyword is used for both detectors and sprinklers. Sprinklers and detectors are both considered detection devices and are handled using the same input keywords.  Detection is based upon heat transfer to the detector. Fire suppression by a user-specified water spray begins once the associated detection device is activated.

For the type of detector, use 1 for smoke detector and 2 for heat detector or sprinklers. If suppression is set to a value of 1, a sprinkler will quench the fire with the specified spray density of water. If turned off (a value of 0), the device is handled as a heat or smoke detector only - values entered for activation temperature, RTI, and spray density are ignored.

The spray density is the amount of water dispersed by a water sprinkler.  The units for spray density are length/time.  These units are derived by dividing the volumetric rate of water flow by the area protected by the water spray. The suppression calculation is based upon an experimental correlation by Evans, and depends upon the RTI, activation temperature, and spray density to determine the behavior of the sprinkler.

Example:

\begin{lstlisting}
DETECT,2,1,344.2,1.5,1.5,2.29,98,1,7.00E-05
\end{lstlisting}

\section{DJIGN}

\begin{lstlisting}
DJIGN, 373.15
\end{lstlisting}
This entry sets the ignition temperature for door jet fires. Please read the technical reference manual for the meaning and implication of modifying these two parameters \cite{CFAST_Tech_Guide_6}.

This entry is superseded by the two-entry key word GLOBA which replaces both LIMO2 and DJIGN.

Example:

\begin{lstlisting}
DJIGN,488
\end{lstlisting}

\section{DTCHE}

\begin{lstlisting}
DTCHE, MINIMUM_TIME, COUNT
\end{lstlisting}
The purpose of DTCHE is to prevent excessive computation with a very small time step. This often appears to users as though the program has frozen, when it is simply the set of equations has reached a point that requires very small time steps for the solver to converge. A negative entry on DTCHE turns off the time step checking algorithm. A positive value for the minimum time input sets the software for a maximum number of iterations (COUNT) below a minimum time step (MINIMUN\_TIME) before producing an error and stopping.

Example:

\begin{lstlisting}
DTCHECK,1.E-9,100
\end{lstlisting}

\section{EAMB and TAMB}

\begin{lstlisting}
TAMB, AMBIENT_TEMPERATURE, AMBIENT_PRESSURE, STATION_ELEVATION, RELATIVE_HUMIDITY
EAMB, AMBIENT_TEMPERATURE, AMBIENT_PRESSURE, STATION_ELEVATION, RELATIVE_HUMIDITY
\end{lstlisting}

This keyword sets ambient conditions, TAMB for the internal and EAMB for outside the building. For the internal ambient, relative humidity sets the initial water concentration in the ambient air.  For the external ambient, it sets the water content for air flowing into compartments through vents connecting to the exterior. Temperatures are in Kelvin, pressure in Pascals, and relative humidity in percent. Default values for temperature, pressure, and relative humidity are 20~\degc, 101300~Pa, and 50~\%, respectively.

Example:

\begin{lstlisting}
EAMB,300,101300,0,50
TAMB,300,101300,0,50
\end{lstlisting}

\section{EVENT}

\begin{lstlisting}
EVENT, TYPE, FIRST_COMPARTMENT, SECOND_COMPARTMENT, VENT_NUMBER, TIME, FINAL_FRACTION
\end{lstlisting}

Type indicates the vent type associated with this EVENT action. ``H'' indicates a horizontal flow event that changes the vent opening, ``V'' a vertical flow event, ``M'' a mechanical flow event, and ``F'' for filtering of mechanical ventilation flow.  Final\_Fraction is the percent of the full opening width expressed as a fraction for vents and fraction of trace species and soot removed for filters.

EVENT is used to open or close a vent or to change filtering. This replaces the earlier CVENT and applies to all vents for vertical flow (V), horizontal flow (H) mechanical flow (M), and filtering of mechanical ventilation (F). The intent is to allow these events to be triggered by time, temperature or flux as is done with detectors. However, at the moment, time is the only option.

The form for EVENT is

\begin{lstlisting}
EVENT, H, FIRST_COMPARTMENT, SECOND_COMPARTMENT, VENT_NUMBER, TIME, FINAL_FRACTION, DECAY_TIME
EVENT, V, FIRST_COMPARTMENT, SECOND_COMPARTMENT, V_ID, TIME, FINAL_FRACTION, DECAY_TIME
EVENT, M, FIRST_COMPARTMENT, SECOND_COMPARTMENT, MVENT_ID, TIME, FINAL_FRACTION, DECAY_TIME
EVENT, F, FIRST_COMPARTMENT, SECOND_COMPARTMENT, MVENT_ID, TIME, FINAL_FRACTION, DECAY_TIME
\end{lstlisting}

Decay time is the duration of the event and the units are seconds.

Example:

\begin{lstlisting}
EVENT,H,1,2,1,10.,0.3,1
\end{lstlisting}

The convention for vent fractions is the 1 is 100 \% open, and 0 is closed. Filtering applies only to trace species and particulate (soot) and only to mechanical ventilation vents. For filtering, 0 indicates no filtering and 1 indicates that all soot and trace species flowing through the fan is removed.

\section{FIRE}

\begin{lstlisting}
FIRE, COMPARTMENT, X_POSITION, Y_POSITION, Z_POSITION, 1, IGNITION_TYPE, IGNITION_CRITERION, ASSOCIATED_IGNITION_TARGET, 0, 0, NAME
\end{lstlisting}

This key word places a fire into a compartment. The associated keywords CHEMI, TIME, HRR, SOOT, CO, TRACE, AREA, and HEIGH completely specify the fire for the current simulation. IGNITION\_TYPE can be TIME, FLUX, or TEMP.  IF FLUX or TEMP is specified, an existing, user-specified target specified in the ASSOCIATED\_IGNITION\_TARGET field is used to determine the ignition time.

Example:

\begin{lstlisting}
!!bunsen
FIRE,1,4.55,2.5,0,1,TIME,0,0,0,0,burner
CHEMI,1,4,0,0,0,0.33,5E+07
TIME,0,60,120,180,240,300,360,420,480,540,1800
HRR,0,100000,150000,200000,150000,125000,100000,90000,80000,75000,75000
SOOT,0,0,0,0,0,0,0,0,0,0,0
CO,0.001047221,0.001047221,0.001047221,0.001047221,0.001047221,0.001047221,0.001047221,0.001047221,0.001047221,0.001047221,0.001047221
TRACE,0,0,0,0,0,0,0,0,0,0,0
AREA,0.1,0.1,0.1,0.1,0.1,0.1,0.1,0.1,0.1,0.1,0.1
HEIGH,0,0,0,0,0,0,0,0,0,0,0
\end{lstlisting}

\subsection{CHEMI}

\begin{lstlisting}
CHEMI, FORMULA_C, FORMULA_H, FORMULA_O, FORMULA_N, FORMULA_Cl,  RADIATIVE_FRACTION, HEAT_OF_COMBUSTION
\end{lstlisting}

The CHEMI input defines the chemical formula of the burning material (FORMULA\_C, \_H, \_O, \_N, and \_Cl), its heat of combustion, and fraction of its head given off by radiation.

\subsection{TIME}

\begin{lstlisting}
TIME, T_1, T_2, T_3, ... , T_N-1, T_N
\end{lstlisting}

The TIME input defines a series of time points which correspond to entries on the other fire inputs HRR, SOOT, CO, TRACE, AREA, and HEIGH.  These define the time-based variation of the fire size, fire area, and species yields for the specified fire.

\subsection{HRR}

\begin{lstlisting}
HRR, Q_1, Q_2, Q_3, ... , Q_N-1, Q_N
\end{lstlisting}

The HRR input defines a series of heat release rates which correspond to entries on the TIME input.  These define the time-based variation of the fire size for the specified fire.

\subsection{SOOT, CO, TRACE}

\begin{lstlisting}
SOOT, S_1, S_2, S_3, ... , S_N-1, S_N
CO, C_1, C_2, C_3, ... , C_N-1, C_N
TRACE, TR_1, TR_2, TR_3, ... , TR_N-1, TR_N
\end{lstlisting}

These three inputs define a series of species yields which correspond to entries on the TIME input.  These define the time-based variation of the soot, carbon monoxide, and trace species yields for the specified fire.

\subsection{AREA}

\begin{lstlisting}
AREA, A_1, A_2, A_3, ... , A_N-1, A_N
\end{lstlisting}

The AREA input defines a series of areas which correspond to entries on the TIME input.  These define the time-based variation of the cross-sectional area of the base of the fire for the specified fire.

\subsection{HEIGH}

\begin{lstlisting}
HEIGH, H_1, H_2, H_3, ... , H_N-1, H_N
\end{lstlisting}

The HEIGH input defines a series of height values which correspond to entries on the TIME input.  These define the time-based variation of the vertical position of the base of the fire (measured from the floor of the current compartment) for the specified fire.

\section{GLOBA}

\begin{lstlisting}
GLOBA, LOWER_OXYGEN_LIMIT,  IGNITION_TEMPERATURE
\end{lstlisting}

This parameter is global and applies to all fires. Here, with two parameters, the command sets the lower oxygen limit and the ignition temperature for door jet fires. The first entry sets the lower oxygen limit for combustion in a layer. The second entry sets the ignition temperature for door jet fires. Please read the technical reference manual for the meaning and implication of modifying these two parameters \cite{CFAST_Tech_Guide_7}.

This two-entry key word replaces LIMO2 and DJIGN.

Example:

\begin{lstlisting}
GLOBA,10,488
\end{lstlisting}

This sets the limiting oxygen index to 10 \% and the ignition temperature to 488 K. These are the default values.

\section{HALL}

\begin{lstlisting}
HALL, COMPARTMENT
\end{lstlisting}

This command invokes the corridor flow ceiling jet algorithm for the chosen compartment.

Example:

\begin{lstlisting}
HALL, 3
\end{lstlisting}

\section{HHEAT}

\begin{lstlisting}
HHEAT, FIRST_COMPARTMENT, NUMBER_OF_PARTS, N PAIRS OF {SECOND_COMPARTMENT, FRACTION}
\end{lstlisting}

Used to allow heat conduction between pairs of compartments which have a contiguous vertical partition between them.  There are two forms of this command. The first form is to use only a compartment number. In this case, CFAST will calculate the conductive heat transfer to all compartments connected to this compartment by horizontal convective flow. The second form specifies the compartments to be connected and what fraction of the compartment is connected to an adjacent compartment. This latter is particularly useful for rooms which are connected to adjacent rooms as well as hallways. The user of the model is responsible for the consistency of these pairings.  The model does not check to insure that the specified compartment pairs are located next to one another.

Example:

\begin{lstlisting}
HHEAT,1,1,2,0.5
\end{lstlisting}
specifies that compartment one has one connection to compartment two and the fraction of wall surface through which heat is transferred is one half of the wall surface of compartment one.

\section{HVENT}

\begin{lstlisting}
HVENT, FIRST_COMPARTMENT, SECOND_COMPARTMENT, VENT_NUMBER, WIDTH, SOFFIT, SILL, FIRST_COMPARTMENT_OFFSET, SECOND_COMPARTMENT_OFFSET, FACE, OPENING_TRIGGER_TYPE, OPENING_TRIGGER_CRITERION, ASSOCIATED_OPENING_TRIGGER_TARGET, INITIAL_TIME, INITIAL_FRACTION, FINAL_TIME, FINAL_FRACTION
\end{lstlisting}

Vent which allows horizontal flow of gases through vents such as doors and windows. Compartment offsets are triplets from the compartment origin.  FACE is an integer from 1 to 4 counterclockwise from the origin defining which wall face to place the vent on when visualizing with Smokeview. It doesn't affect the dynamics of the calculation, just the way it looks. The size of the opening can be modified by EVENT. This changes the Opening\_Fraction from the initial value (set above) to some other value. Typical use is to start with the door open (Initial\_Opening\_Fraction~=~1) and use EVENT to close the door (Final\_Fraction~=~0).

Example:

\begin{lstlisting}
HVENT,1,2,1,1,2,0,2.5,0,1,TIME,,,0,1,0,1
\end{lstlisting}

\section{INTER}

\begin{lstlisting}
INTER INITIAL_INTERFACE_HEIGHT_1 INITIAL_INTERFACE_HEIGHT_2 ... INITIAL_INTERFACE_HEIGHT_N
\end{lstlisting}

This is used to set the initial interface height below the top of the compartment. A great deal of care is needed to use this, as the model has only rudimentary checks for the limits imposed (for example, the initial value must specify a height not greater than the compartment height. This does change the nature of a zone in the context of a zone model.

Example:

\begin{lstlisting}
INTER 2   1.2     3   2.0
\end{lstlisting}

\section{ISOF}

\begin{lstlisting}
ISOF, VALUE, COMPARTMENT(S)
\end{lstlisting}

The ISOF command creates 3-dimensional animated contours of gas temperature in one or more compartments.  For example, a 300 \degc temperature isosurface is a 3-D surface on which the gas temperature is 300 \degc. The output frequency of the slice files is controlled by the SMOKEVIEW\_OUTPUT\_INTERVAL input on the TIMES input line. Note that 3-D isosurface files can be quite large if the output interval is small. Computational time can also increase significantly with specification of isosurface files.

\begin{lstlisting}
ISOF, 300, 1, 2
ISOF, 600
\end{lstlisting}

The first example specifies a 300 \degc temperature isosurface in compartments 1 and 2.  The last example specifies 600 \degc isosurfaces in all compartments.

\section{LIMO2}

\begin{lstlisting}
LIMO2, LOWER_OXYGEN_INDEX
\end{lstlisting}

This parameter is global and applies to all fires. The single entry sets the lower oxygen limit for combustion.  Please read the technical reference manual for the meaning and implication of modifying these parameters \cite{CFAST_Tech_Guide_7}.

This entry is superceded by the two-entry key word GLOBA which replaces both LIMO2 and DJIGN.

\section{MATL}

\begin{lstlisting}
MATL, SHORT_NAME, CONDUCTIVITY, SPECIFIC_HEAT, DENSITY, THICKNESS, EMISSIVITY, LONG_NAME
\end{lstlisting}

This command defines the thickness and thermal properties for a single material that may be referenced as a compartment surface material, target material, or fire object. Each name must be unique within a single input file.

\begin{lstlisting}
MATL,CONCRETE,1.75,1000,2200,0.15,0.94,"Concrete, Normal Weight (6 in)"
MATL,GLASSFB3,0.036,795,105,0.013,0.9,"Glass Fiber, Organic Bonded (1/2 in)"
\end{lstlisting}

\section{MVENT}

\begin{lstlisting}
MVENT, FROM_COMPARTMENT, TO_COMPARTMENT, ID_NUMBER, FROM_OPENING_ORIENTATION, FROM_CENTER_HEIGHT, FROM_OPENING_AREA, TO_OPENINGORIENTATION, TO_CENTER_HEIGHT, TO_OPENING_AREA, FLOW, BEGIN_DROPOFF_PRESSURE, ZERO_FLOW_PRESSURE, OPENING_TRIGGER_TYPE, OPENING_TRIGGER_CRITERION, ASSOCIATED_OPENING_TRIGGER_TARGET, INITIAL_TIME, INITIAL_FRACTION, FINAL_TIME, FINAL_FRACTION, X_OFFSET, Y_OFFSET
\end{lstlisting}

This replaces the more complex mechanical ventilation commands with a constant flow fan connection.  The original commands MVOPN MVFAN MVDCT and INELV are not supported in this version. The command specifies a pair of openings connected by a constant volume flow fan. 

Example:

\begin{lstlisting}
MVENT,2,1,1,V,1,0.25,V,1,0.25,0.02,200,300,FLUX,6,Targ1,,0,,1,2,2
\end{lstlisting}

\section{ONEZ}

\begin{lstlisting}
ONEZ, COMPARTMENT
\end{lstlisting}

For tall compartments or those removed from the room of fire origin, the compartment may be modeled as a single, well-mixed zone rather than the default two-zone assumption. A single zone approximation is appropriate for smoke flow far from a fire source where the two-zone layer stratification is less pronounced than in compartments near the fire. This is used in situations where the stratification does not occur. Examples are elevators, shafts, complex stairwells, and compartments far from the fire.

Example:

\begin{lstlisting}
ONEZ,2
\end{lstlisting}

\section{ROOMA and ROOMH}

\begin{lstlisting}
ROOMA AND ROOMH, COMPARTMENT, NUMBER_OF_VALUES, AREA(OR HEIGHT) VALUES
\end{lstlisting}

These key words allow the user to define non-rectangular rooms by specifying cross-sectional area as a function of height. One set of values is included for each compartment that has a variable cross-sectional area. The format in both cases is the key followed by an index of the number of values. These key words must be used in pairs.

Example:

\begin{lstlisting}
ROOMA 1    3     10.0  5.0  3.0
ROOMH 1    3     0.0   1.0  2.0
\end{lstlisting}

The example specifies that compartment 1 has a cross-sectional area of 10 m$^2$, 5 m$^2$ and 3 m$^2$ at elevations 0.0 m, 1.0 m and 2.0 m respectively.

\section{SLCF}

\begin{lstlisting}
SLCF, DOMAIN, LOCATION, COMPARTMENT(S)
\end{lstlisting}

With Smokeview, CFAST can display animations of various gas phase quantities in a plane or volume in one or more compartments, depending on the inputs. For 2-D slices, the DOMAIN input is 2-D, followed by the selected plane (either X, Y, or Z), and a position within the compartment along the respective axis. For 3-D slices, the DOMAIN input 3-D is selected. One, several, or all compartments can be specified. As many SLCF commands as needed can be used in a CFAST input file.  The output frequency of the slice files is controlled by the SMOKEVIEW\_OUTPUT\_INTERVAL input on the TIMES input line. Note that 3-D slice files can be quite large if the output interval is small. Computational time can also increase significantly with specification of 3-D slice files.

Example:

\begin{lstlisting}
SLCF, 3-D, 1, 2, 3
SLCF, 2-D, X, 2.5, 3, 5
SLCF, 3-D
SLCF, 2-D
\\end{lstlisting}

The first SLCF example specifies a 3-D slice file output for compartments 1, 2, and 3.  The second SLCF example specifies a 2-d slice file output parallel to the X (depth) axis, 2.5 m from the axis origin in compartments 3 and 5. The last two examples specify default 3-D and 2-D slice file output for all compartments.

\section{STPMAX}

\begin{lstlisting}
STPMAX, MAXIMUM_TIME_STEP
\end{lstlisting}

This specifies the largest time step that the model will take. The default value is one second. In most cases, the numerical routines adjust the time step appropriately; however, for long simulation times and slowly varying conditions, a larger value is appropriate. In cases where the fire height and vent soffits interact, the time step may need to be smaller.

Example:

\begin{lstlisting}
STPMAX,0.2
\end{lstlisting}

\section{TARGE}

\begin{lstlisting}
TARGE, COMPARTMENT, WIDTH, DEPTH, HEIGHT, NORMAL_DEPTH, NORMAL_BREADTH, NORMAL_HEIGHT, MATERIAL, METHOD, EQUATION_TYPE, INTERNAL_LOCATION
\end{lstlisting}

CFAST can track and report calculations of the net heat flux striking arbitrarily positioned and oriented targets and the temperature of these targets. A non-zero normal vector must be specified as must a material from the thermophysical database. METHOD can be set to STEADY for steady state solution, XPLICIT for explicit solution, and MPLICIT for an implicit solution. Equation\_Type can be PDE for cartesian coordinates or CYL for cylindrical coordinates.

Example:

\begin{lstlisting}
TARGE,1,2.20,1.88,2.34,0.00,0.00,1.00,CONCRETE,IMPLICIT,PDE
\end{lstlisting}

\section{TIMES}

\begin{lstlisting}
TIMES, SIMULATION_TIME, PRINT_INTERVAL, SPREADSHEET_OUTPUT_INTERVAL, SMOKEVIEW_OUTPUT_INTERVAL
\end{lstlisting}

Example:

\begin{lstlisting}
TIMES,360,-120,130,140,150,
\end{lstlisting}

Printed output will be on the screen or in a file named project.out. The four spreadsheet listings are in project.(project\_n.csv, project\_w.csv, project\_f.csv and project\_s.csv). The Smokeview output interval are for graphical output using the companion program Smokeview.

\section{THRMF}

\begin{lstlisting}
THRMF, THERMOPHYSICAL_PROPERTIES_FILE
\end{lstlisting}

By default, thermophysical properties are obtained from the thermal.csv which is located in the directory where the model executables reside. This allows for another file to be used.

Example:

\begin{lstlisting}
THRMF,NEWTHERMALFILE
\end{lstlisting}

\section{VERSN}

\begin{lstlisting}
VERSN,version number, Title
\end{lstlisting}

This input must be the first line in a CFAST input and largely specifies a title for the simulation. The major version number from the file must match the major version number kept internally (7 at the moment).

Example

\begin{lstlisting}
VERSN,6,"Simple test of the object file input"
\end{lstlisting}

\section{VHEAT}

\begin{lstlisting}
VHEAT, FIRST_COMPARTMENT, SECOND_COMPARTMENT
\end{lstlisting}

Heat transfer between the ceiling and floor of specified compartments can be incorporated with the VHEAT key word. Ceiling to floor heat transfer occurs between interior compartments of the structure or between an interior compartment and the outdoors. The model checks to make sure that the ceiling and floor are reasonably contiguous (within 0.01 m), and the assumption is made that this is true for the entire ceiling and floor.

Example:

\begin{lstlisting}
VHEAT,1,2
\end{lstlisting}

The floor properties of the top compartment (1 in this case) and the ceiling properties of the bottom compartment (2 in this case) must be defined by COMPA and included in the thermophysical file.

\section{VVENT}

\begin{lstlisting}
VVENT, TOP_COMPARTMENT, BOTTOM_COMPARTMENT, VENT_NUMBER, AREA, SHAPE, OPENING_TRIGGER_TYPE, OPENING_TRIGGER_CRITERION, ASSOCIATED_OPENING_TRIGGER_TARGET, INITIAL_TIME, INITIAL_FRACTION, FINAL_TIME, FINAL_FRACTION, X_OFFSET, Y_OFFSET
\end{lstlisting}

This key word describes horizontal openings between compartments connected with a vent in the floor/ceiling between the compartments. Each VVENT line in the input file describes one floor/ceiling vent.

Example:

\begin{lstlisting}
VVENT,2,1,1,0.5,2,TEMP,573.15,Targ1,,0,,1,2.5,2.5
VVENT,2,1,1,0.5,2,TIME,,,0,1,0,1,2,2
\end{lstlisting}



\chapter{Running CFAST from a Command Prompt}

The model CFAST can also be run from a Windows command prompt.  CFAST can be run from any folder, and refer to a data file in any other folder. The fires and thermophysical properties have to be in either the data folder, or the executable folder. The data folder is checked first and then the executable folder.

\begin{lstlisting}
[drive1:\][folder1\]cfast [drive2:\][folder2\]project
\end{lstlisting}

The project name will have extensions appended as needed (see below). For example, to run a test case when the CFAST executable is located in c:$\backslash$firemodels$\backslash$cfast7 and the input data file is located in c:$\backslash$data, the following command could be used:

\begin{lstlisting}
c:\firemodels\cfast7\cfast c:\data\testfire0   <<< note that the extension is optional.
\end{lstlisting}

Command line options

\begin{itemize}
\item -k - no interactive keyboard access
\item -i - initialization only
\item -c - compact output
\item -f - full output (c and f are exclusive). Note the interaction of the f and c option. The default for the console output is /c. The default for the file output is /f. This default action can be overwritten by explicitly including the /f or /c option.
\item -n - net heat flux option
\item -v - validation output (outputs a modified set of spreadsheet files with different column headers designed to facilitate automated analysis of the output)
\end{itemize}


\label{last_page}
