\chapter{CFAST Text-based Input File}

The operation of CFAST is based on a single ASCII\footnote{ASCII -- American Standard Code for Information Interchange. There are 256 characters that make up the standard ASCII text.} text file containing parameters organized into {\em namelist}\footnote{A {\em namelist} is a Fortran input record.} groups.
The input file provides CFAST  with all of the necessary information to describe the scenario. The graphical user interface, CEdit, writes this file. This appendix details all the parameters, which are organized into groups that roughly coincide with the tabs in the graphical user interface.

\section{Naming the Input File}

The input file is saved with a name such as {\ct job\_name.in}, where {\ct job\_name} is any character string that helps to identify the simulation. All of the output files associated with the calculation will then have this common prefix name.

There should be no blank spaces in the job name. Instead use the underscore character to represent a space.

Be aware that CFAST will simply over-write the output files of a given case if its assigned name is the same. This is convenient when developing an input file because you save on disk space. Just be careful not to overwrite a calculation that you want to keep.

\section{Namelist Formatting}

Parameters are specified within the input file by using {\em namelist} formatted records. Each namelist record begins with the ampersand character, {\ct \&}, followed immediately by the name of the namelist group, then a comma-delimited list of the input parameters, and finally a forward slash, {\ct /}. For example, the line
\begin{lstlisting}
&TIME  SIMULATION = 3600., PRINT = 50., SMOKEVIEW = 50., SPREADSHEET = 50. /
\end{lstlisting}
sets various values of parameters contained in the {\ct TIME} namelist group. The meanings of these various parameters is explained in this guide. The namelist records can span multiple lines in the input file, but just be sure to end the record with a slash or else the data will not be understood. Do not add anything to a namelist line other than the parameters and values appropriate for that group. Otherwise, CFAST will stop immediately upon execution.

Parameters within a namelist record can be separated by either commas, spaces, or line breaks. It is recommended that you use commas or line breaks, and never use tab stops because they are not explicitly defined in the namelist data structure. Comments and notes can be written into the file so long as nothing comes before the ampersand except a space and nothing comes between the ampersand and the slash except appropriate parameters corresponding to that particular namelist group.

The parameters in the input file can be integers, reals, character strings, or logical parameters. A logical parameter is either {\ct .TRUE.} or {\ct .FALSE.} -- the periods are a Fortran convention. Character strings that are listed in this User's Guide must be copied exactly as written -- the code is case sensitive and underscores {\em do} matter. The maximum length of most character input parameters is 60.

Most of the input parameters are simply real or integer scalars, like {\ct PRINT = 50.}, but sometimes the inputs can be arrays.

Note that character strings can be enclosed either by single or double quotation marks. Be careful not to create the input file by pasting text from something other than a simple text editor, in which case the punctuation marks may not transfer properly into the text file. Some text file encodings may not work on all systems. If file reading errors occur and no typographical errors can be found in the input file, try saving the input file using a different encoding. For example, the text file editor Notepad works fine on a Windows PC, but a file edited in Notepad may not work on Linux or Mac OS~X because of the difference in line endings between Windows and Unix/Linux operating systems. The editor Wordpad typically works better, but try a simple case first.


\section{Input File Structure}

In general, the namelist records can be entered in any order in the input file, but it is a good idea to organize them in some systematic way. Typically, general information is listed near the top of the input file, and detailed information, like obstructions, devices, and so on, are listed below. CFAST scans the entire input file each time it processes a particular namelist group. With some text editors, it has been noticed that the last line of the file is often not read by CFAST because of the presence of an ``end of file'' character. To ensure that CFAST reads the entire input file, add
\begin{lstlisting}
&TAIL /
\end{lstlisting}
as the last line at the end of the input file. This completes the file from {\ct \&HEAD} to {\ct \&TAIL}. CFAST does not even look for this last line. It just forces the ``end of file'' character past relevant input.

The general structure of an input file is shown below, with many lines of the original input file\footnote{The actual input file, Users\_Guide\_Example.in, is part of the CFAST software distribution} removed for clarity.

\begin{lstlisting}
&HEAD  VERSION = 7300, TITLE = 'Users Guide Example Case' /
&TIME  SIMULATION = 3600., PRINT = 50., SMOKEVIEW = 50., SPREADSHEET = 50. /
&INIT  PRESSURE = 101325., RELATIVE_HUMIDITY = 50., TEMPERATURES = 20., 20. /
&MISC  LOWER_OXYGEN_LIMIT = 0.10 /
&MATL ...
&COMP ID = 'Comp 1', WIDTH = 5., DEPTH = 5., HEIGHT = 3. ... /
&RAMP ... / for compartment cross-sectional area if required
&VENT ID = 'VVENT 1', TYPE = 'CEILING', COMP_IDS = 'Comp 3', 'Comp 2',
      AREA = 1., SHAPE = 'ROUND',
&RAMP ... / for vent opening and closing if required
&FIRE ID = 'Wood_Wall', COMP_ID = 'Comp 2', LOCATION = 2.5, 5., 0.,
      CARBON = 6, HYDROGEN = 10, OXYGEN = 5,
      HEAT_OF_COMBUSTION = 1.81e4, RADIATIVE_FRACTION = 0.33,
      SOOT_YIELD = 0.015, CO_YIELD = 0.006171682,
      AREA = 4.5, HEIGHT = 1.5,
      HRR_RAMP_ID = 'Wood Wall Fire' /
&RAMP ID = 'Wood Wall Fire', TYPE = 'HEAT_RELEASE_RATE',
      T = 0., 8000., F = 0., 1000000. / for the fire
&DEVC ID = 'Smoke Detector 1', COMP_ID = 'Comp 1',
      TYPE = 'SMOKE', LOCATION = 3., 3., 2.97, SETPOINT = 23.93346 /
&CONN TYPE = 'FLOOR', COMP_ID = 'Comp 3', COMP_IDS = 'Comp 2' /
&SLCF DOMAIN = '2-D', PLANE = 'X', POSITION = 2.5 /
&TAIL / End of file.
\end{lstlisting}
It is recommended that when looking at a new scenario, first select a pre-written input file that resembles the case, make the necessary changes, then run the case to determine if the geometry is set up correctly. It is best to start off with a relatively simple file that captures the main features of the problem without getting tied down with too much detail that might mask a fundamental flaw in the calculation. As you learn how to write input files, you will continually run and re-run your case as you add in complexity.

Table~\ref{tbl:namelistgroups} provides a quick reference to all the namelist parameters and where you can find the reference to where it is introduced in the document and the table containing all of the keywords for each group.


\begin{table}[ht]
\begin{center}
\caption{CFAST Input File Keywords}
\label{tbl:namelistgroups}
\begin{tabular}{|c|l|c|c|}
\hline
Group Name   & Namelist Group Description     & Reference Section & Parameter Table  \\ \hline
{\ct COMP}   & Compartments                   & \ref{info:COMP}   & \ref{tbl:COMP}   \\ \hline
{\ct CONN}   & Surface Connections            & \ref{info:CONN}   & \ref{tbl:CONN}   \\ \hline
{\ct DEVC}   & Devices                        & \ref{info:DEVC}   & \ref{tbl:DEVC}   \\ \hline
{\ct FIRE}   & Fires                          & \ref{info:FIRE}   & \ref{tbl:FIRE}   \\ \hline
{\ct HEAD}   & Input File Header              & \ref{info:HEAD}   & \ref{tbl:HEAD}   \\ \hline
{\ct INIT}   & Initial Conditions             & \ref{info:INIT}   & \ref{tbl:INIT}   \\ \hline
{\ct ISOF}   & Isosurface File Outputs        & \ref{info:ISOF}   & \ref{tbl:ISOF}   \\ \hline
{\ct MATL}   & Material Properties            & \ref{info:MATL}   & \ref{tbl:MATL}   \\ \hline
{\ct MISC}   & Miscellaneous                  & \ref{info:MISC}   & \ref{tbl:MISC}   \\ \hline
{\ct RAMP}   & Ramp Profiles                  & \ref{info:RAMP}   & \ref{tbl:RAMP}   \\ \hline
{\ct SLCF}   & Slice File Outputs             & \ref{info:SLCF}   & \ref{tbl:SLCF}   \\ \hline
{\ct TAIL}   & End of Input File Indicator    &                   &                  \\ \hline
{\ct TIME}   & Simulation Time                & \ref{info:TIME}   & \ref{tbl:TIME}   \\ \hline
{\ct VENT}   & Vents                          & \ref{info:VENT}   & \ref{tbl:VENT}   \\ \hline
\end{tabular}
\end{center}
\end{table}

Examples of each of the inputs are included in the sections that follow.  All examples are taken from the sample input file {\ct Users\_Guide\_Example.in} included with the CFAST distribution. Following are some general rules about the CFAST input file:

\begin{itemize}
\item The {\ct HEAD} input identifies the version of CFAST for which the input file was created and is typically the first line in the input file. Use {\ct \&TAIL} as the last line at the end of the input file. This completes the file from {\ct \&HEAD} to {\ct \&TAIL}. CFAST does not even look for this last line. It just forces the “end of file” character past relevant input.
\item Many of the listed keywords are mutually exclusive. Repeated entry of some keywords can cause the program to either fail or run in an unpredictable manner.
\item Use of some keywords triggers the code to operate in a certain mode/condition. For example, specifying {\ct ADIABATIC} to be {\ct TRUE} triggers the code to treat all compartment surfaces to be perfectly insulated.
\item Multiple inputs are required whenever the keyword is in plural form --- keywords ending with an \textbf{s}. For example, the keyword parameter, {\ct TEMPERATURES}, within the namelist group, {\ct INIT}, requires two temperature values (in this case, one for exterior ambient temperature and one for interior ambient temperature). In the case of missing inputs, an error message will be generated to assist users in troubleshooting any errors.
\item Default values to inputs are assigned to some of the keywords to facilitate the set up of an input file. For instance, Table \ref{tbl:MISC} shows that the {\ct LOWER\_OXYGEN\_LIMIT} has a default value of {\ct 0.15}. This value is taken from the SFPE handbook \cite{SFPE:2003} and implies that the burning rate will be reduced when the oxygen level is below 15\%. Users should review the applicability of any default values for their simulation.
\end{itemize}


\clearpage

\section{Simulation Environment, Namelist Groups \texorpdfstring{{\tt HEAD}}{HEAD}, \texorpdfstring{{\tt TIME}}{TIME}, \texorpdfstring{{\tt INIT}}{INIT}, and  \texorpdfstring{{\tt MISC}}{MISC}}

\renewcommand{\tabcolsep}{.1in}
\begin{longtable}{|l|l|l|l|l|}
\caption[Header Parameters ({\ct HEAD} namelist group)]{For more information see Section~\ref{info:HEAD}.}
\label{tbl:HEAD} \\
\hline
\multicolumn{5}{|c|}{{\ct HEAD} (Header Parameters)} \\
\hline \hline
\endfirsthead
\caption[]{Continued} \\
\hline
\multicolumn{5}{|c|}{{\ct HEAD} (Header Parameters)} \\
\hline \hline
\endhead
\parbox{1.5in}{\bf Parameter}    & \parbox{1in}{\bf Type}  & \parbox{1in}{\bf Reference}  & \parbox{1in}{\bf Units}  & \parbox{1in}{\bf Default Value} \\ \hline
{\ct VERSION}         & Integer                 & Section \ref{info:HEAD}      &         	     &                   \\ \hline
{\ct TITLE}           & Character               & Section \ref{info:HEAD}      &       		     &                 	 \\ \hline
\end{longtable}



\begin{longtable}{@{\extracolsep{\fill}}|l|l|l|l|l|}
\caption[Time Parameters ({\ct TIME} namelist group)]{For more information see Section~\ref{info:TIME}.}
\label{tbl:TIME} \\
\hline
\multicolumn{5}{|c|}{{\ct TIME} (Time Parameters)} \\
\hline \hline
\endfirsthead
\caption[]{Continued} \\
\hline
\multicolumn{5}{|c|}{{\ct TIME} (Time Parameters)} \\
\hline \hline
\endhead
\parbox{1.5in}{\bf Parameter}    & \parbox{1in}{\bf Type}  & \parbox{1in}{\bf Reference}  & \parbox{1in}{\bf Units}  & \parbox{1in}{\bf Default Value} \\ \hline
{\ct PRINT}             & Integer   & Section \ref{info:TIME}                 & s         & 60              \\ \hline
{\ct SIMULATION}        & Integer   & Section \ref{info:TIME}                 & s         & 3600            \\ \hline
{\ct SMOKEVIEW}         & Integer   & Section \ref{info:TIME}                 & s         & 15              \\ \hline
{\ct SPREADSHEET}       & Integer   & Section \ref{info:TIME}                 & s         & 15              \\ \hline
\end{longtable}



\begin{longtable}{@{\extracolsep{\fill}}|l|l|l|l|l|}
\caption[Initial Conditions ({\ct INIT} namelist group)]{For more information see Section~\ref{info:INIT}.}
\label{tbl:INIT} \\
\hline
\multicolumn{5}{|c|}{{\ct INIT} (Initial Conditions)} \\
\hline \hline
\endfirsthead
\caption[]{Continued} \\
\hline
\multicolumn{5}{|c|}{{\ct INIT} (Initial Conditions)} \\
\hline \hline
\endhead
\parbox{1.5in}{\bf Parameter}    & \parbox{1in}{\bf Type}  & \parbox{1in}{\bf Reference}  & \parbox{1in}{\bf Units}  & \parbox{1in}{\bf Default Value} \\ \hline
{\ct PRESSURE}                   & Real   		   & Section \ref{info:INIT}      & Pa       		     & 101325     		       \\ \hline
{\ct RELATIVE\_HUMIDITY}         & Real   		   & Section \ref{info:INIT}      & \%      		     & 50       		       \\ \hline
{\ct INTERIOR\_TEMPERATURE}       & Real  	       & Section \ref{info:INIT}      & $^\circ$C		     & 20      	   \\ \hline
{\ct EXTERIOR\_TEMPERATURE}       & Real            & Section \ref{info:INIT}      & $^\circ$C		     & 20      	   \\ \hline
\end{longtable}



\begin{longtable}{@{\extracolsep{\fill}}|l|l|l|l|l|}
\caption[Miscellaneous Parameters ({\ct MISC} namelist group)]{For more information see Section~\ref{info:MISC}.}
\label{tbl:MISC} \\
\hline
\multicolumn{5}{|c|}{{\ct MISC} (Miscellaneous Parameters)} \\
\hline \hline
\endfirsthead
\caption[]{Continued} \\
\hline
\multicolumn{5}{|c|}{{\ct MISC} (Miscellaneous Parameters)} \\
\hline \hline
\endhead
\parbox{1.5in}{\bf Parameter}    & \parbox{1in}{\bf Type}  & \parbox{1in}{\bf Reference}  & \parbox{1in}{\bf Units}  & \parbox{1in}{\bf Default Value} \\ \hline
{\ct ADIABATIC}            & Logical     & Section \ref{info:MISC}                 &           & {\ct .FALSE.}   \\ \hline
{\ct MAX\_TIME\_STEP}      & Real        & Section \ref{info:MISC}                 & s         &   2             \\ \hline
{\ct LOWER\_OXYGEN\_LIMIT} & Real        & Section \ref{info:MISC}                 &           &   0.15          \\ \hline
\end{longtable}

\noindent Examples:
\begin{lstlisting}
&HEAD  VERSION = 7300, TITLE = 'Users Guide Example Case' /
&TIME  SIMULATION = 3600., PRINT = 50., SMOKEVIEW = 50., SPREADSHEET = 50. /
&INIT  PRESSURE = 101325., RELATIVE_HUMIDITY = 50.,
       INTERIOR_TEMPERATURE = 20., EXTERIOR_TEMPERATURE = 20. /
&MISC  LOWER_OXYGEN_LIMIT = 0.10 /
\end{lstlisting}


\clearpage

\section{Thermal Properties, Namelist Group \texorpdfstring{{\tt MATL}}{MATL}}

\begin{minipage}{6.5in}
\renewcommand\footnoterule{}
\begin{longtable}{@{\extracolsep{\fill}}|l|l|l|l|l|}
\caption[Thermal Properties ({\ct MATL} namelist group)]{For more information see Section~\ref{info:MATL}.}
\label{tbl:MATL} \\
\hline
\multicolumn{5}{|c|}{{\ct MATL} (Material Properties)} \\
\hline \hline
\endfirsthead
\caption[]{Continued} \\
\hline
\multicolumn{5}{|c|}{{\ct MATL} (Material Properties)} \\
\hline \hline
\endhead
\parbox{1.5in}{\bf Parameter}    & \parbox{1in}{\bf Type}  & \parbox{1in}{\bf Reference}  & \parbox{1in}{\bf Units}  & \parbox{1in}{\bf Default Value} \\ \hline
{\ct CONDUCTIVITY}*\footnote{ * indicates a required input for each {\ct MATL} input included in the input file.}       & Real 	 & Section \ref{info:MATL}                 & kW/(m$\cdot$K)  	&                 \\ \hline
{\ct DENSITY}*            & Real 	 & Section \ref{info:MATL}                 & kg/m$^3$ 		&                 \\ \hline
{\ct EMISSIVITY}          & Real	 & Section \ref{info:MATL}                 &         		&   0.9           \\ \hline
{\ct SPECIFIC\_HEAT}*     & Real	 & Section \ref{info:MATL}                 & kJ/(kg$\cdot$K)    &                 \\ \hline
{\ct ID}*                 & Character    & Section \ref{info:MATL}                 &                    &                 \\ \hline
{\ct THICKNESS}*          & Real  	 & Section \ref{info:MATL}                 & m     	        &                 \\ \hline
\end{longtable}
\end{minipage}

\vspace{\baselineskip}

\noindent Example:
\begin{lstlisting}
&MATL ID = 'CONCRETE', CONDUCTIVITY = 1.75, SPECIFIC_HEAT = 1.,
      DENSITY = 2200., EMISSIVITY = 0.94, THICKNESS = 0.15 /
\end{lstlisting}


\clearpage
\section{Compartments, Namelist Group \texorpdfstring{{\tt COMP}}{COMP}}

\begin{minipage}{6.5in}
\renewcommand\footnoterule{}
%\begin{longtable}{@{\extracolsep{\fill}}|l|l|l|l|l|}
\begin{longtable}{|l|l|l|l|l@{\extracolsep{\fill}}|}
\caption[Compartment parameters ({\ct COMP} namelist group)]{For more information see Section~\ref{info:COMP}.}
\label{tbl:COMP} \\
\hline
\multicolumn{5}{|c|}{{\ct COMP} (Compartment Parameters)} \\
\hline \hline
\endfirsthead
\caption[]{Continued} \\
\hline
\multicolumn{5}{|c|}{{\ct COMP} (Compartment Parameters)} \\
\hline \hline
\endhead
\parbox{1.5in}{\bf Parameter}    & \parbox{1in}{\bf Type}  & \parbox{1in}{\bf Reference}  & \parbox{1in}{\bf Units}  & \parbox{1in}{\bf Default Value} \\ \hline
{\ct CEILING\_MATL\_ID}     & Character		 & Section \ref{info:COMP2}               &           &                 \\ \hline
{\ct DEPTH}*\footnote{ * indicates a required input for each {\ct COMP} input included in the input file. At least on {\ct COMP} input must be included in an input file.}                 & Real     		 & Section \ref{info:COMP}                & m         &  		\\ \hline
{\ct FLOOR\_MATL\_ID}       & Character		 & Section \ref{info:COMP2}               &           &                 \\ \hline
{\ct GRID}             & Integer Triplet& Section \ref{info:SLCF2}               &           & 50,50,50                \\ \hline
{\ct HALL}                  & Logical  		 & Section \ref{info:COMP3}               &           & {\ct .FALSE.}   \\ \hline
{\ct HEIGHT}*                & Real     		 & Section \ref{info:COMP}   	          & m         &                 \\ \hline
{\ct ID}*                    & Character		 & Section \ref{info:COMP}                &           &          \\ \hline
{\ct ORIGIN}           & Real Triplet   & Section \ref{info:COMP}                & m         & 0,0,0                \\ \hline
{\ct ROOM\_AREA\_RAMP}
\footnote{For compartments where the cross-sectional area varies with height, the namelist group, {\ct RAMP} can be used. Keywords associated with this feature can be found in Section~\ref{tbl:RAMP}.}
                            & Character          & Section \ref{info:COMP4}                 &           &                 \\ \hline
{\ct SHAFT}                 & Logical  		 & Section \ref{info:COMP3}                 &           & {\ct .FALSE.}   \\ \hline
{\ct WALL\_MATL\_ID}        & Character		 & Section \ref{info:COMP2}                 &           &                 \\ \hline
{\ct WIDTH}*                 & Real               & Section \ref{info:COMP}                  & m         &                 \\ \hline
\end{longtable}
\end{minipage}

\vspace{\baselineskip}
\noindent Example:
\begin{lstlisting}
&COMP ID = 'Comp 1', WIDTH = 5., DEPTH = 5., HEIGHT = 3.,
      ORIGIN = 0., 0., 0., CEILING_MATL_ID = 'CONCRETE',
      WALL_MATL_ID = 'CONCRETE', FLOOR_MATL_ID = 'CONCRETE',
      GRID = 50, 50, 50 /
\end{lstlisting}




\clearpage
\section{Vents, Namelist Group \texorpdfstring{{\tt VENT}}{VENT}}
\label{info:VENT6}

\subsection{Wall Vents\texorpdfstring{{\tt ,TYPE='WALL'}}{, TYPE='WALL'}}

\begin{minipage}{6.5in}
\renewcommand\footnoterule{}
\begin{longtable}{@{\extracolsep{\fill}}|l|l|l|l|l|}
\caption[Wall Vent Parameters ({\ct VENT} namelist group)]{For more information see Section~\ref{info:VENT}.}
\label{tbl:VENT} \\
\hline
\multicolumn{5}{|c|}{{\ct VENT, TYPE='WALL'} (Wall Vent Parameters)} \\
\hline \hline
\endfirsthead
\caption[]{Continued} \\
\hline
\multicolumn{5}{|c|}{{\ct VENT, TYPE='WALL'} (Wall Vent Parameters)} \\
\hline \hline
\endhead
\parbox{1.5in}{\bf Parameter}    & \parbox{1in}{\bf Type}  & \parbox{1in}{\bf Reference}  & \parbox{1in}{\bf Units}  & \parbox{1in}{\bf Default Value} \\ \hline
{\ct BOTTOM}\footnote{ * indicates a required input for each wall {\ct VENT} input included in the input file.} *               & Real                   & Section \ref{info:VENT}      & m                        &                 \\ \hline
{\ct COMP\_IDS}*     	   			         & Character Doublet        & Section \ref{info:VENT}      &                             &                 \\ \hline
{\ct CRITERION}\footnote{Input for {\ct CRITERION} must be {\ct FLUX}, {\ct TEMPERATURE}, or {\ct TIME}. An associated {\ct SETPOINT} is required. For {\ct FLUX} or {\ct TEMPERATURE}, an associated ignition target must be specified by {\ct DEVC\_ID}.}
                            				         & Selection List              & Section \ref{info:VENT}      &                             &                 \\ \hline
{\ct DEVC\_ID}           					 & Character  		  & Section \ref{info:VENT}      &                             &                 \\ \hline
{\ct FACE}\footnote{Input for {\ct FACE} must be {\ct RIGHT}, {\ct FRONT}, {\ct LEFT}, or {\ct REAR}. Both {\ct FACE} and {\ct OFFSET} positioning refer to the first compartment specified ({\ct COMP\_IDS(1)}).}      	  & Selection List   & Section \ref{info:VENT}                 &                             &                 \\ \hline
{\ct ID}*                                                         & Character  	          & Section \ref{info:VENT}      &                             &                 \\ \hline
{\ct OFFSET}       					 & Real 		  & Section \ref{info:VENT}      & m                           &      0        \\ \hline
{\ct PRE\_FRACTION}\footnote{{\ct PRE\_FRACTION} and {\ct POST\_FRACTION} specify the vent fraction before and after the {\ct SETPOINT} is reached when {\ct CRITERION} is either {\ct TEMPERATURE} or {\ct FLUX}. They cannot be used with {\ct VENT\_RAMP\_ID}.}       					 & Real 		  & Section \ref{info:VENT}      & m                           &      1        \\ \hline
{\ct POST\_FRACTION}       					 & Real 		  & Section \ref{info:VENT}      & m                           &      1        \\ \hline
{\ct SETPOINT}           					 & Real  	          & Section \ref{info:VENT}      & s $\mid$ $^\circ$C $\mid$ kW/m$^2$ &                 \\ \hline
{\ct TYPE}\footnote{Input for {\ct TYPE} must be {\ct WALL} to specify a wall vent. } *
                                                                 & Selection List         & Section \ref{info:VENT6}     &                             &                 \\ \hline
{\ct TOP}*                 & Real                   & Section \ref{info:VENT}      & m                        &                 \\ \hline
{\ct VENT\_RAMP\_ID}\footnote{Associated ramp specifies the fraction of vent width for wall vents as a function of time and is only applicable when {\ct CRITERION} is set to {\ct TIME}.}  					 & Character  		  & Section \ref{info:RAMP}      &                             &                 \\ \hline
{\ct WIDTH}*                                                      & Real                   & Section \ref{info:VENT}      & m                           &                 \\ \hline
\end{longtable}
\end{minipage}

\graybox{Wall vents ({\ct TYPE='WALL'}) are defined by {\ct BOTTOM}, {\ct TOP}, and {\ct WIDTH}. Location of the vent for visualization is defined by {\ct FACE}, and {\ct OFFSET}. \\
}

\noindent Example:
\begin{lstlisting}
&VENT TYPE = 'WALL', ID = 'HVENT 1', COMP_IDS = 'Comp 1', 'OUTSIDE',
      WIDTH = 1., TOP = 2., BOTTOM = 0.,
      OFFSET = 2., FACE = 'FRONT', CRITERION = 'TIME' /

\end{lstlisting}

\subsection{Ceiling / Floor Vents\texorpdfstring{{\tt ,TYPE='CEILING'} or {\tt TYPE='FLOOR'}}{{, TYPE='CEILING'} or {\tt TYPE='FLOOR'}}}

\begin{minipage}{6.5in}
\renewcommand\footnoterule{}
\begin{longtable}{@{\extracolsep{\fill}}|l|l|l|l|l|}
\caption[Ceiling/Floor Vent Parameters ({\ct VENT} namelist group)]{For more information see Section~\ref{info:VENT}.}
\label{tbl:VENT} \\
\hline
\multicolumn{5}{|c|}{{\ct VENT, TYPE='CEILING'} or {\ct TYPE='FLOOR'} (Ceiling/Floor Vent Parameters)} \\
\hline \hline
\endfirsthead
\caption[]{Continued} \\
\hline
\multicolumn{5}{|c|}{{\ct VENT, TYPE='CEILING'} or {\ct TYPE='FLOOR'} (Ceiling/Floor Vent Parameters)} \\
\hline \hline
\endhead
\parbox{1.5in}{\bf Parameter}    & \parbox{1in}{\bf Type}  & \parbox{1in}{\bf Reference}  & \parbox{1in}{\bf Units}  & \parbox{1in}{\bf Default Value} \\ \hline
{\ct AREA}\footnote{ * indicates a required input for each ceiling/floor {\ct VENT} input included in the input file.} *      	 & Real  	          & Section \ref{info:VENT2}     & m$^2$                    &                                 \\ \hline
{\ct COMP\_IDS}*     	   			         & Character Doublet        & Section \ref{info:VENT}      &                             &                 \\ \hline
{\ct CRITERION}\footnote{Input for {\ct CRITERION} must be {\ct FLUX}, {\ct TEMPERATURE}, or {\ct TIME}. An associated {\ct SETPOINT} is required. For {\ct FLUX} or {\ct TEMPERATURE}, an associated ignition target must be specified by {\ct DEVC\_ID}.}
                            				         & Selection List              & Section \ref{info:VENT}      &                             &                 \\ \hline
{\ct DEVC\_ID}           					 & Character  		  & Section \ref{info:VENT}      &                             &                 \\ \hline
{\ct ID}*                                                         & Character  	          & Section \ref{info:VENT}      &                             &                 \\ \hline
{\ct OFFSETS}       					 & Real Doublet 		  & Section \ref{info:VENT}      & m                           &      0,0        \\ \hline
{\ct PRE\_FRACTION}\footnote{{\ct PRE\_FRACTION} and {\ct POST\_FRACTION} specify the vent fraction before and after the {\ct SETPOINT} is reached when {\ct CRITERION} is either {\ct TEMPERATURE} or {\ct FLUX}. They cannot be used with {\ct VENT\_RAMP\_ID}.}       					 & Real 		  & Section \ref{info:VENT}      & m                           &      1        \\ \hline
{\ct POST\_FRACTION}       					 & Real 		  & Section \ref{info:VENT}      & m                           &      1        \\ \hline
{\ct SHAPE}*\footnote{Input for {\ct SHAPE} must be {\ct ROUND} or {\ct SQUARE}.}
                                                                 & Selection List         & Section \ref{info:VENT2}     &                             &                 \\ \hline
{\ct TYPE}\footnote{Input for {\ct TYPE} must be {\ct CEILING} or {\ct FLOOR} for ceiling and floor vents. These may be used interchangeably. The order of {\ct COMP\_IDS} specifies the location of the vent with {\ct COMP\_IDS(1)} specifying the top compartment and {\ct COMP\_IDS(2)} specifying the bottom compartment.} *
                                                                 & Selection List         & Section \ref{info:VENT6}     &                             &                 \\ \hline
{\ct VENT\_RAMP\_ID}\footnote{Associated ramp specifies the fraction of vent area or ceiling and floor vents and is only applicable when {\ct CRITERION} is set to {\ct TIME}.}  					 & Character  		  & Section \ref{info:RAMP}      &                             &                 \\ \hline
\end{longtable}
\end{minipage}

\graybox {Ceiling and floor vents ({\ct TYPE='CEILING'} or {\ct TYPE='FLOOR'}) are defined by {\ct AREA} and {\ct SHAPE}. Location of the vent for visualization is defined by {\ct OFFSETS} as the distance from the compartment origin in the X (width) and Y (depth) direction. It is visualized on the floor of {\ct COMP\_IDS(1)}. \\
}

\noindent Example:
\begin{lstlisting}
&VENT TYPE = 'CEILING', ID = 'VVENT 1', COMP_IDS = 'Comp 3', 'Comp 2',
      AREA = 1., SHAPE = 'ROUND', CRITERION= 'TIME',
      OPENING_RAMP_ID = 'RAMP_VVENT_1'/
&RAMP ID = 'RAMP_VVENT_1', TYPE = 'FRACTION',
      T = 0., 100., 500.,
      F = 0., 0.5, 1. /

\end{lstlisting}

\subsection{Mechanical Vents\texorpdfstring{{\tt ,TYPE='MECHANICAL'}}{, TYPE='MECHANICAL'}}

\begin{minipage}{6.5in}
\renewcommand\footnoterule{}
\begin{longtable}{@{\extracolsep{\fill}}|l|l|l|l|l|}
\caption[Mechanical Vent Parameters ({\ct VENT} namelist group)]{For more information see Section~\ref{info:VENT}.}
\label{tbl:VENT} \\
\hline
\multicolumn{5}{|c|}{{\ct VENT, TYPE='MECHANICAL'} (Mechanical Vent Parameters)} \\
\hline \hline
\endfirsthead
\caption[]{Continued} \\
\hline
\multicolumn{5}{|c|}{{\ct VENT, TYPE='MECHANICAL'} (Mechanical Vent Parameters)} \\
\hline \hline
\endhead
\parbox{1.5in}{\bf Parameter}    & \parbox{1in}{\bf Type}  & \parbox{1in}{\bf Reference}  & \parbox{1in}{\bf Units}  & \parbox{1in}{\bf Default Value} \\ \hline
{\ct AREAS}\footnote{ * indicates a required input for each mechanical {\ct VENT} input included in the input file.} *       & Real Doublet 	          & Section \ref{info:VENT3}     & m$^2$                    &                 \\ \hline
{\ct COMP\_IDS}*    	   			         & Character Doublet        & Section \ref{info:VENT}      &                             &                 \\ \hline
{\ct CRITERION}\footnote{Input for {\ct CRITERION} must be {\ct FLUX}, {\ct TEMPERATURE}, or {\ct TIME}. An associated {\ct SETPOINT} is required. For {\ct FLUX} or {\ct TEMPERATURE}, an associated ignition target must be specified by {\ct DEVC\_ID}.}
                            				         & Selection List              & Section \ref{info:VENT}      &                             &                 \\ \hline
{\ct CUTOFFS}       					 & Real Doublet 	  	  & Section \ref{info:VENT4}     & Pa                          &     200,300     \\ \hline
{\ct DEVC\_ID}           					 & Character  		  & Section \ref{info:VENT}      &                             &                 \\ \hline
{\ct FILTERING\_RAMP\_ID}    					 & Character  		  & Section \ref{info:RAMP}      &                             &                 \\ \hline
{\ct FLOW}*      	 					 & Real  		  & Section \ref{info:VENT4}     & m$^3$/s                     &                 \\ \hline
{\ct HEIGHTS}*      					 & Real Doublet  	  & Section \ref{info:VENT3}     & m                           &                 \\ \hline
{\ct ID}*                                                         & Character  	          & Section \ref{info:VENT}      &                             &                 \\ \hline
{\ct OFFSETS}       					 & Real Doublet 		  & Section \ref{info:VENT}      & m                           &      0,0        \\ \hline
{\ct PRE\_FRACTION}\footnote{{\ct PRE\_FRACTION} and {\ct POST\_FRACTION} specify the vent fraction before and after the {\ct SETPOINT} is reached when {\ct CRITERION} is either {\ct TEMPERATURE} or {\ct FLUX}. They cannot be used with {\ct VENT\_RAMP\_ID}.}       					 & Real 		  & Section \ref{info:VENT}      & m                           &      1        \\ \hline
{\ct POST\_FRACTION}       					 & Real 		  & Section \ref{info:VENT}      & m                           &      1        \\ \hline
{\ct ORIENTATION}\footnote{Input for {\ct ORIENTATION} must be {\ct HORIZONTAL} or {\ct VERTICAL}}  					 & Selection List  		  & Section \ref{info:RAMP}      &                             &                 \\ \hline
{\ct SETPOINT}           					 & Real  	          & Section \ref{info:VENT}      & s $\mid$ $^\circ$C $\mid$ kW/m$^2$ &                 \\ \hline
{\ct TYPE}\footnote{Input for {\ct TYPE} must be {\ct MECHANICAL} for mechanical vents. } *
                                                                 & Selection List         & Section \ref{info:VENT6}     &                             &                 \\ \hline
{\ct VENT\_RAMP\_ID}\footnote{Associated ramp specifies the fraction of flow for mechanical vents and is only applicable when {\ct CRITERION} is set to {\ct TIME}.}  					 & Character  		  & Section \ref{info:RAMP}      &                             &                 \\ \hline
\end{longtable}
\end{minipage}

\graybox{Mechanical vents {\ct TYPE='MECHANICAL'} are defined by {\ct AREAS}, {\ct CUTOFFS}, {\ct FLOW}, and {\ct HEIGHTS}. Location of the vent for visualization is defined by {\ct ORIENTATION} and {\ct OFFSETS} as the distance from the compartment origin in the X (width) and Y (depth) direction and by {\ct HEIGHTS(1)} in the z direction. \\
}

\noindent Example:
\begin{lstlisting}
&VENT TYPE = 'MECHANICAL', ID = 'MVENT_1',
      COMP_IDS = 'OUTSIDE', 'Comp 1', AREAS = 0.25, 0.25,
      HEIGHTS = 2.75, 2.75, CRITERION = 'TIME', FLOW = 0.02,
      CUTOFFS = 200., 300., ORIENTATION='VERTICAL', OFFSETS = 0., 4. /

\end{lstlisting}




\clearpage
\section{Fires, Namelist Group \texorpdfstring{{\tt FIRE}}{FIRE}}
\label{info:FIRE3}

\begin{minipage}{6.5in}
\renewcommand\footnoterule{}
\begin{longtable}{@{\extracolsep{\fill}}|l|l|l|l|l|}
\caption[Fire Parameters ({\ct FIRE} namelist group)]{For more information see Section~\ref{info:FIRE}.}
\label{tbl:FIRE} \\
\hline
\multicolumn{5}{|c|}{{\ct FIRE} (Fire Parameters)} \\
\hline \hline
\endfirsthead
\caption[]{Continued} \\
\hline
\multicolumn{5}{|c|}{{\ct FIRE} (Fire Parameters)} \\
\hline \hline
\endhead
\parbox{1.5in}{\bf Parameter}    & \parbox{1in}{\bf Type}  & \parbox{1in}{\bf Reference}  & \parbox{1in}{\bf Units}  & \parbox{1in}{\bf Default Value} \\ \hline
{\ct AREA}\footnote{For AREA, HRR, and species yields, either a constant value (e.g., AREA or CO\_YIELD) or ramp (e.g., AREA\_RAMP\_ID or CO\_YIELD\_RAMP\_ID) may be used. Do not use both.}                & Real        & Section \ref{info:FIRE}                 & m$^2$                       &     0.3         \\ \hline
{\ct AREA\_RAMP\_ID}                 & Character        & Section \ref{info:FIRE}                 &                        &              \\ \hline
{\ct CARBON}               & Real     & Section \ref{info:FIRE}                 &                             &  1               \\ \hline
{\ct CHLORINE}             & Real     & Section \ref{info:FIRE}                 &                             &  0               \\ \hline
{\ct COMP\_ID}*\footnote{ * indicates a required input for each {\ct FIRE} input included in the input file. For heat release rate, either {\ct HRR} or {\ct HRR\_RAMP\_ID} must be included but not both.}             & Character   & Section \ref{info:FIRE}                 &                             &                 \\ \hline
{\ct CO\_YIELD}            & Real        & Section \ref{info:FIRE}                 &                             &  0               \\ \hline
{\ct CO\_YIELD\_RAMP\_ID}            & Character        & Section \ref{info:FIRE}                 &                             &                 \\ \hline
{\ct DEVC\_ID}             & Character   & Section \ref{info:FIRE}                 &                             &                 \\ \hline
{\ct HCL\_YIELD}           & Real   & Section \ref{info:FIRE}                 &                             &  0               \\ \hline
{\ct HCL\_YIELD\_RAMP\_ID}            & Character        & Section \ref{info:FIRE}                 &                             &                 \\ \hline
{\ct HCN\_YIELD}           & Real   & Section \ref{info:FIRE}                 &                             &  0               \\ \hline
{\ct HCN\_YIELD\_RAMP\_ID}            & Character        & Section \ref{info:FIRE}                 &                             &                 \\ \hline
{\ct HEAT\_OF\_COMBUSTION} & Real        & Section \ref{info:FIRE}                 & kJ/kg                       &     50000       \\ \hline
{\ct HRR}*        & Real   &  Section \ref{info:RAMP}                &  kW                           &                 \\ \hline
{\ct HRR\_RAMP\_ID}*        & Character   &  Section \ref{info:RAMP}                &                             &                 \\ \hline
{\ct HYDROGEN}             & Real     & Section \ref{info:FIRE}                 &                             &  4               \\ \hline
{\ct ID}*                   & Character   & Section \ref{info:FIRE}                 &                             &                 \\ \hline
{\ct IGNITION\_CRITERION}\footnote{Input for {\ct IGNITION\_CRITERION} must be {\ct FLUX}, {\ct TEMPERATURE}, or {\ct TIME}. An associated {\ct SETPOINT} is required. For {\ct FLUX} or {\ct TEMPERATURE}, an associated ignition target must be specified by {\ct DEVC\_ID}.}
                           & Selection List   & Section \ref{info:FIRE}                 &                             & TIME                \\ \hline
{\ct LOCATION}*             & Real Triplet        & Section \ref{info:FIRE}                 & m                           &                 \\ \hline
{\ct NITROGEN}             & Real     & Section \ref{info:FIRE}                 &                             & 0                \\ \hline
{\ct OXYGEN}               & Real     & Section \ref{info:FIRE}                 &                             & 0                \\ \hline
%{\ct PF\_CO\_YIELD}        & Real        & Section \ref{info:FIRE}                 &                             &                 \\ \hline
%{\ct PF\_SOOT\_YIELD}      & Real        & Section \ref{info:FIRE}                 &                             &                 \\ \hline
%{\ct PF\_TRACE\_YIELD}     & Real        & Section \ref{info:FIRE}                 &                             &                 \\ \hline
%{\ct POST\_FLASHOVER}      & Logical     & Section \ref{info:FIRE}                 &                             &  {\ct .FALSE.}  \\ \hline
{\ct RADIATIVE\_FRACTION}  & Real        & Section \ref{info:FIRE}                 &                             &     0.35        \\ \hline
{\ct SETPOINT}             & Real        & Section \ref{info:FIRE}                 & s $\mid$ $^\circ$C $\mid$ kW/m$^2$  & 0 s        \\ \hline
{\ct SOOT\_YIELD}          & Real        & Section \ref{info:FIRE}                 &                             & 0                \\ \hline
{\ct SOOT\_YIELD\_RAMP\_ID}            & Character        & Section \ref{info:FIRE}                 &                             &                 \\ \hline
{\ct TRACE\_YIELD}         & Real        & Section \ref{info:FIRE}                 &                             &  0               \\ \hline
{\ct TRACE\_YIELD\_RAMP\_ID}            & Character        & Section \ref{info:FIRE}                 &                             &                 \\ \hline
\end{longtable}
\end{minipage}


\vspace{\baselineskip}
\noindent Example:
\begin{lstlisting}
&FIRE ID = 'Cushion', COMP_ID = 'Comp 1',
      LOCATION = 2.5, 2.5, 0.,
      CARBON = 9, HYDROGEN = 6, OXYGEN = 2, NITROGEN = 2,
      IGNITION_CRITERION = 'TIME',
      HEAT_OF_COMBUSTION = 50000., RADIATIVE_FRACTION = 0.33,
      SOOT_YIELD = 0.227, CO_YIELD = 0.0845557,
      AREA = 0.1, HEIGHT = 0., HRR_RAMP_ID = 'Cushion Fire' /
&RAMP ID = 'Cushion Fire', TYPE = 'HEAT_RELEASE_RATE',
      T = 0., 60., 120., 180., 240., 300.,
      HRR = 0., 100000., 150000., 200000., 150000., 125000. /
\end{lstlisting}




\clearpage
\section{Devices, Namelist Group \texorpdfstring{{\tt DEVC}}{DEVC}}

\label{info:DEVC3}

\begin{minipage}{6.5in}
\renewcommand\footnoterule{}
\begin{longtable}{@{\extracolsep{\fill}}|l|l|l|l|l|}
\caption[Device Parameters ({\ct DEVC} namelist group)]{For more information see Section~\ref{info:DEVC}.}
\label{tbl:DEVC} \\
\hline
\multicolumn{5}{|c|}{{\ct DEVC} (Device Parameters)} \\
\hline \hline
\endfirsthead
\caption[]{Continued} \\
\hline
\multicolumn{5}{|c|}{{\ct DEVC} (Device Parameters)} \\
\hline \hline
\endhead
\parbox{1.5in}{\bf Parameter}    & \parbox{1in}{\bf Type}  & \parbox{1in}{\bf Reference}  & \parbox{1in}{\bf Units}  & \parbox{1in}{\bf Default Value} \\ \hline
{\ct COMP\_ID}*\footnote{ * indicates a required input for each {\ct DEVC} input included in the input file. Additional inputs may be required depending on the type of device.}            & Character   & Section \ref{info:DEVC}     &                   &                 \\ \hline
{\ct ID}*      		  & Character   & Section \ref{info:DEVC}     &                   &                 \\ \hline
{\ct TEMPERATURE\_DEPTH}  & Real        & Section \ref{info:DEVC}     &                   &       0.5       \\ \hline
{\ct LOCATION}*       & Real Triplet  & Section \ref{info:DEVC}     & m                 &                 \\ \hline
{\ct MATL\_ID}            & Character   & Section \ref{info:DEVC}     &                   &                 \\ \hline
{\ct NORMAL}         & Real Triplet  & Section \ref{info:DEVC}     &                   & 0,0,1                \\ \hline
{\ct RTI}                 & Real        & Section \ref{info:DEVC2}    & $\sqrt{\hbox{m}\cdot\hbox{s}}$   & 130  \\ \hline
{\ct SETPOINT}
	                  & Real        & Section \ref{info:DEVC}     & \degc or \%/m &     see footnote\footnote{For smoke detectors, the input is obscuration with a default value of 23.93 \%/m (8 \%/ft); for heat detectors, temperature with a default value of 57 \degc (135 \degf); and for sprinklers, temperature with a default value of 74 \degc (165 \degf).}  \\ \hline
{\ct SPRAY\_DENSITY}      & Real        & Section \ref{info:DEVC2}    & m/s               &                 \\ \hline
{\ct TYPE}* \footnote{Input for {\ct TYPE} must be {\ct PLATE}, {\ct CYLINDER}, {\ct SPRINKLER}, {\ct HEAT\_DETECTOR}, or {\ct SMOKE\_DETECTOR}}
                          & Selection List   & Section \ref{info:DEVC}     &                   &                 \\ \hline
\end{longtable}
\end{minipage}

\graybox{{\ct PLATE} and {\ct CYLINDER} targets are defined by {\ct COMP\_ID}, {\ct TYPE}, {\ct ID}, {\ct LOCATION}, {\ct TEMPERATURE\_DEPTH}, and {\ct NORMAL}. \\

A sprinkler is defined by {\ct COMP\_ID}, {\ct TYPE}, {\ct ID}, {\ct LOCATION}, {\ct RTI}, {\ct SETPOINT}, and {\ct SPRAY\_DENSITY}.  \\

A smoke detector is defined by {\ct COMP\_ID}, {\ct TYPE}, {\ct ID}, {\ct LOCATION}, and {\ct SETPOINT}. \\

A heat detector is defined by {\ct COMP\_ID}, {\ct TYPE}, {\ct ID}, {\ct LOCATION}, {\ct RTI}, and {\ct SETPOINT}.
}

\vspace{\baselineskip}
\noindent Examples:
\begin{lstlisting}
&DEVC ID = 'Targ 1', COMP_ID = 'Comp 1', TYPE = 'PLATE',
      LOCATION = 2.2, 1.88, 2.34., NORMAL = 0., 0., 1.,
      MATL_ID = 'CONCRETE', INTERNAL_LOCATION = 0.5 /

&DEVC ID = 'Sprinkler 1', COMP_ID = 'Comp 1',
      TYPE = 'SPRINKLER',
      LOCATION = 3., 3., 2.97,
      SETPOINT = 73.8889, RTI = 100., SPRAY_DENSITY = 7.E-5 /
\end{lstlisting}



\clearpage
\section{Compartment Connections, Namelist Group \texorpdfstring{{\tt CONN}}{CONN}}

\begin{minipage}{6.5in}
\renewcommand\footnoterule{}
\begin{longtable}{@{\extracolsep{\fill}}|l|l|l|l|l|}
\caption[Connection Parameters ({\ct CONN} namelist group)]{For more information see Section~\ref{info:CONN}.}
\label{tbl:CONN} \\
\hline
\multicolumn{5}{|c|}{{\ct CONN} (Connection Parameters)} \\
\hline \hline
\endfirsthead
\caption[]{Continued} \\
\hline
\multicolumn{5}{|c|}{{\ct CONN} (Connection Parameters)} \\
\hline \hline
\endhead
\parbox{1.5in}{\bf Parameter}    & \parbox{1in}{\bf Type}  & \parbox{1in}{\bf Reference}  & \parbox{1in}{\bf Units}  & \parbox{1in}{\bf Default Value} \\ \hline
{\ct COMP\_ID}*\footnote{ * indicates a required input for each {\ct CONN} input included in the input file.}             & Character           & Section \ref{info:CONN}                 &           &  	      \\ \hline
{\ct COMP\_IDS}*            & Character Array     & Section \ref{info:CONN}                 &           &  	      \\ \hline
{\ct F}                    & Real Array          & Section \ref{info:CONN}                 &           &              \\ \hline
{\ct TYPE}* \footnote{Input for {\ct TYPE} must be {\ct CEILING}, {\ct FLOOR}, or {\ct WALL}}        	
                           & Selection List           & Section \ref{info:CONN}                 &           &              \\ \hline
\end{longtable}
\end{minipage}

\vspace{\baselineskip}
\noindent Example:
\begin{lstlisting}
&CONN TYPE = 'FLOOR', COMP_ID = 'Comp 3', COMP_IDS = 'Comp 2' /
\end{lstlisting}




\clearpage
\section{Visualization, Namelist Groups \texorpdfstring{{\tt ISOF}}{ISOF}, \texorpdfstring{{\tt SLCF}}{SLCF}}

\subsection{\texorpdfstring{{\tt ISOF}}{ISOF} (Isosurface Parameters)}

\begin{minipage}{6.5in}
\renewcommand\footnoterule{}
\begin{longtable}{@{\extracolsep{\fill}}|l|l|l|l|l|}
\caption[Isosurface parameters ({\ct ISOF} namelist group)]{For more information see Section~\ref{info:ISOF}.}
\label{tbl:ISOF} \\
\hline
\multicolumn{5}{|c|}{{\ct ISOF} (Isosurface Parameters)} \\
\hline \hline
\endfirsthead
\caption[]{Continued} \\
\hline
\multicolumn{5}{|c|}{{\ct ISOF} (Isosurface Parameters)} \\
\hline \hline
\endhead
\parbox{1.5in}{\bf Parameter}    & \parbox{1in}{\bf Type}  & \parbox{1in}{\bf Reference}  & \parbox{1in}{\bf Units}  & \parbox{1in}{\bf Default Value} \\ \hline
{\ct COMP\_ID}*\footnote{ * indicates a required input for each {\ct ISOF} input included in the input file.}          & Character   & Section \ref{info:ISOF}                 &           &                 \\ \hline
{\ct VALUE}*             & Real        & Section \ref{info:ISOF}                 & $^\circ$C &                 \\ \hline
\end{longtable}
\end{minipage}

\vspace{\baselineskip}

\noindent Example:
\begin{lstlisting}
&ISOF COMP_ID = 'COMP_ID', VALUE = 100. /
\end{lstlisting}




\subsection{\texorpdfstring{{\tt SLCF}}{SLCF} (Slice File Parameters)}

\begin{minipage}{6.5in}
\renewcommand\footnoterule{}
\begin{longtable}{@{\extracolsep{\fill}}|l|l|l|l|l|}
\caption[Slice File parameters ({\ct SLCF} namelist group)]{For more information see Section~\ref{info:SLCF}.}
\label{tbl:SLCF} \\
\hline
\multicolumn{5}{|c|}{{\ct SLCF} (Slice File Parameters)} \\
\hline \hline
\endfirsthead
\caption[]{Continued} \\
\hline
\multicolumn{5}{|c|}{{\ct SLCF} (Slice File Parameters)} \\
\hline \hline
\endhead
\parbox{1.5in}{\bf Parameter}    & \parbox{1in}{\bf Type}  & \parbox{1in}{\bf Reference}  & \parbox{1in}{\bf Units}  & \parbox{1in}{\bf Default Value} \\ \hline
{\ct COMP\_ID}*\footnote{ * indicates a required input for each {\ct SLCF} input included in the input file. All inputs are required for a 2-D domain slice file.}          & Character   & Section \ref{info:SLCF}                 &           &                 \\ \hline
{\ct DOMAIN}*\footnote{DOMAIN must be 2-D or 3-D. If 2-D, {\ct PLANE} specifies the orientation of the slice in the X, Y, or Z direction and {\ct POSITION} specifies the offset from the origin of the specified plane.}            & Selection List   & Section \ref{info:SLCF}                 &           &                 \\ \hline
{\ct PLANE}             & Selection List   & Section \ref{info:SLCF}                 &           &                 \\ \hline
{\ct POSITION}          & Real        & Section \ref{info:SLCF}                 &           &                 \\ \hline
\end{longtable}
\end{minipage}

\vspace{\baselineskip}
\noindent Example:
\begin{lstlisting}
&SLCF  DOMAIN = '2-D', PLANE = 'X', POSITION = 2.5 /
\end{lstlisting}




\clearpage
\section{Ramp Functions, Namelist Group \texorpdfstring{{\tt RAMP}}{RAMP}}
\label{info:RAMP}
\label{info:RAMP2}

\begin{minipage}{6.5in}
\renewcommand\footnoterule{}
\begin{longtable}{@{\extracolsep{\fill}}|l|l|l|l|l|}
\caption[Ramp Function Parameters ({\ct RAMP} namelist group)]{For more information see Section~\ref{info:RAMP}.}
\label{tbl:RAMP} \\
\hline
\multicolumn{5}{|c|}{{\ct RAMP} (Ramp Function Parameters)} \\
\hline \hline
\endfirsthead
\caption[]{Continued} \\
\hline
\multicolumn{5}{|c|}{{\ct RAMP} (Ramp Function Parameters)} \\
\hline \hline
\endhead
\parbox{1.5in}{\bf Parameter}    & \parbox{1in}{\bf Type}  & \parbox{1in}{\bf Reference}  & \parbox{1in}{\bf Units}  & \parbox{1in}{\bf Default Value} \\ \hline
{\ct F}*\footnote{ * indicates a required input for each {\ct RAMP} input included in the input file. Either {\ct T} or {\ct Z} (but not both) are required depending on the type associated with the ramp.}               & Real Array            & Section \ref{info:RAMP2}                 &  use dependent        &                 \\ \hline
{\ct ID}*      		  & Character   & Section \ref{info:RAMP2}     &                   &                 \\ \hline
{\ct T}*               & Real Array            & Section \ref{info:RAMP2}                 &  s       &                 \\ \hline
{\ct TYPE}*\footnote{{\ct TYPE} must be either AREA, FRACTION, or HRR}            & Selection List        & Section \ref{info:RAMP2}                 &          &                 \\ \hline
{\ct Z}*               & Real Array            & Section \ref{info:RAMP2}                 &  m       &                 \\ \hline
\end{longtable}
\end{minipage}

\vspace{\baselineskip}

\graybox{
Vent opening is specified as a function of time with {\ct T} as time and {\ct F} as the fraction of the vent width (for wall vents), fraction of the vent area (for ceiling / floor vents), or the flowrate for mechanical vents). Refer to Section~\ref{info:VENT2}. \\

Filter efficiency is specified as a function of time with {\ct T} as time and {\ct F} as the fraction of soot and trace species flow removed in the associated mechanical vent. Refer to Section~\ref{info:VENT4}. \\

Compartment area in XY-plane is specified as a function of compartment height with {\ct Z} as the height and {\ct F} as the cross-sectional area in m$^2$. Refer to Section~\ref{info:COMP4}. \\

Heat release rate of a fire is specified as a function of time with {\ct T} as time and {\ct F} as the heat release rate of the fire in kW. Refer to Section~\ref{info:FIRE2}. \\

Fire species yields are specified as a function of time using {\ct T} as time and {\ct F} as the species yield fraction in kg of species produced per kg of fuel burned.  Refer to Section~\ref{info:FIRE2}.
}





\chapter{Running CFAST from a Command Prompt}

The model CFAST can also be run from a Windows command prompt.  CFAST can be run from any folder, and refer to a data file in any other folder. The fires and thermophysical properties have to be in either the data folder, or the executable folder. The data folder is checked first and then the executable folder.

\begin{lstlisting}
[drive1:\][folder1\]cfast [drive2:\][folder2\]project
\end{lstlisting}

The project name will have extensions appended as needed (see below). For example, to run a test case when the CFAST executable is located in c:$\backslash$firemodels$\backslash$cfast7 and the input data file is located in c:$\backslash$data, the following command could be used:

\begin{lstlisting}
c:\firemodels\cfast7\cfast c:\data\testfire0   <<< note that the extension is optional.
\end{lstlisting}

Command line options

\begin{itemize}
\item -k - no interactive keyboard access
\item -i - initialization only
\item -c - compact output
\item -f - full output (c and f are exclusive). Note the interaction of the f and c option. The default for the console output is /c. The default for the file output is /f. This default action can be overwritten by explicitly including the /f or /c option.
\item -n - net heat flux option
\item -v - validation output (outputs a modified set of spreadsheet files with different column headers designed to facilitate automated analysis of the output)
\end{itemize}


\label{last_page}




