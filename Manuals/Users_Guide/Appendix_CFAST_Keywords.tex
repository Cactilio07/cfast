\chapter{Alphabetical List of Input Parameters} \label{sec:CFAST_Keywords}


This appendix provides a list of all input parameters for CFAST. These parameters are grouped by name-list in separate tables. The name-list groups and their associated parameters are organized in alphabetical order. Inputs for the parameters can be either integers, real numbers, or text. For text inputs, characters are case sensitive. 

Short descriptions are provided as quick references below. Users should refer to the main body of this guide for detailed explanations.


\section{\texorpdfstring{{\tt COMP}}{COMP} (Compartment Parameters)}

COMP defines dimensions, locations, and conditions of all compartments in a simulation. Compartment parameters must be defined before they are referenced by other commands.

\begin{adjustwidth}{1cm}{0cm}
\begin{description}
  \item[ID] User-specified name for the compartment.
  \item[WIDTH, DEPTH, HEIGHT] Dimensions of the compartment. WIDTH, DEPTH, and HEIGHT are in X, Y, and Z direction, respectively.
  \item[ORIGIN] Location at the lower, left, front corner of the compartment.
  \item[CEILING/FLOOR/WALL\_MATL\_ID] User-specified name for surface materials assigned to the compartment. The name must match the ID for one of the defined material properties.
  \item[GRID] Grid spacing for slice and iso-surface files for the compartment.
  \item[HALL] Logical parameter invokes the corridor ceiling jet algorithm for this compartment.
  \item[SHAFT] Logical parameter triggers one-zone approximate for simluation in this compartment.
  \item[ROOM\_AREA\_RAMP] Cross-sectional area in XY plane as a function of height.
\end{description}
\end{adjustwidth}

\vspace{\baselineskip}

\begin{longtable}{@{\extracolsep{\fill}}|l|l|l|l|l|}
\caption[Boundary file parameters ({\ct COMP} namelist group)]{For more information see Section~\ref{info:COMP}.}
\label{tbl:COMP} \\
\hline
\multicolumn{5}{|c|}{{\ct COMP} (Compartment Parameters)} \\
\hline \hline
\endfirsthead
\caption[]{Continued} \\
\hline
\multicolumn{5}{|c|}{{\ct COMP} (Compartment Parameters)} \\
\hline \hline
\endhead
{\ct CEILING\_MATL\_ID}     & Character & Section \ref{info:BNDF}                 &           &                 \\ \hline
{\ct DEPTH}                 & Real      & Section \ref{info:BNDF}                 & m         &  		\\ \hline
{\ct FLOOR\_MATL\_ID}       & Character & Section \ref{info:BNDF}                 &           &                 \\ \hline
{\ct GRID}             & Integer   & Section \ref{info:BNDF}                 &           &                 \\ \hline
{\ct HALL}                  & Logical   & Section \ref{info:outputquantities}     &           & {\ct .FALSE.}   \\ \hline
{\ct HEIGHT}                & Real      & Section \ref{info:outputquantities}     & m         &                 \\ \hline
{\ct ID}                    & Character & Section \ref{info:outputquantities}     &           &                 \\ \hline
{\ct ORIGIN}           & Real      & Section \ref{info:BNDF}                 & m         &                 \\ \hline
{\ct ROOM\_AREA\_RAMP}      & Character & Section \ref{info:BNDF}                 &           &                 \\ \hline
{\ct SHAFT}                 & Logical   & Section \ref{info:BNDF}                 &           & {\ct .FALSE.}   \\ \hline
{\ct WALL\_MATL\_ID}        & Character & Section \ref{info:BNDF}                 &           &                 \\ \hline
{\ct WIDTH}                 & Real      & Section \ref{info:outputquantities}     & m         &                 \\ \hline
\end{longtable}

\noindent Example:
\begin{lstlisting}
&COMP 
	ID='Comp 1', WIDTH=5., DEPTH=5., HEIGHT=3., ORIGIN=0.,0.,0., 
	CEILING_MATL_ID='CONCRETE', WALL_MATL_ID='CONCRETE', 
	FLOOR_MATL_ID='CONCRETE', GRID=50,50,50 /
\end{lstlisting}

\vspace{\baselineskip}


\section{\texorpdfstring{{\tt CONN}}{CONN} (Connection Parameters)}

CONN is used to incorporate conduction heat transfer on surfaces between compartments.

\begin{adjustwidth}{1cm}{0cm}
\begin{description}
  \item[COMP\_ID] Name of the first compartment in which the heat transfer is coming from.
  \item[COMP\_IDS] Name of the adjacent compartment in which the heat transfer is going to. Multiple compartments can be considered.
  \item[F] Surface fraction associated to each pair of connection. Fractions should add to unity.
  \item[TYPE] Three options are available in the selection list: 1) CEILING, 2) FLOOR, and 3) WALL. Be mindful with the heat transfer direction.
\end{description}
\end{adjustwidth}

\vspace{\baselineskip}

\begin{longtable}{@{\extracolsep{\fill}}|l|l|l|l|l|}
\caption[Boundary file parameters ({\ct CONN} namelist group)]{For more information see Section~\ref{info:BNDF}.}
\label{tbl:CONN} \\
\hline
\multicolumn{5}{|c|}{{\ct CONN} (Connection Parameters)} \\
\hline \hline
\endfirsthead
\caption[]{Continued} \\
\hline
\multicolumn{5}{|c|}{{\ct CONN} (Connection Parameters)} \\
\hline \hline
\endhead
{\ct COMP\_ID}              & Character & Section \ref{info:BNDF}                 &           &  		\\ \hline
{\ct COMP\_IDS}              & Character & Section \ref{info:BNDF}                 &           &  		\\ \hline
{\ct F}             & Real      & Section \ref{info:outputquantities}     &           &                 \\ \hline
{\ct TYPE}        	    & {\ct SELECTION LIST} & Section \ref{info:BNDF}                 &           &                 \\ \hline
\end{longtable}

\noindent Example:
\begin{lstlisting}
&CONN 
	TYPE='WALL', COMP_ID='Comp 1', COMP_IDS='Comp 2', 'Comp 3', F=0.5, 0.5 /
\end{lstlisting}

\vspace{\baselineskip}


\section{\texorpdfstring{{\tt DEVC}}{DEVC} (Device Parameters)}

DEVC specifies targets, sprinklers, and detectors. Please refer to cooresponding section for input details associated to each type of device. 

\begin{adjustwidth}{1cm}{0cm}
\begin{description}
  \item[COMP\_ID] Name of the compartment where device is located in.
  \item[DEVC\_TYPE] Five devices are available in the selection list:: 1) PLATE, 2) CYLINDER, 3) SPRINKLER, 4) HEAT\_DETECTOR, and 5) SMOKE\_DETECTOR. PLATE and CYLINDER are two different target types.
  \item[ID] User-specified name of the device.
  \item[INTERNAL\_TEMPERATURE\_LOCATION] Fraction of depth into material at which internal temperature is reported. For center of material, a value of 0.5 is specified.
  \item[LOCATION] Device location relative to the lower, left, front corner of the compartment.
  \item[MATERIAL\_ID] Material name associated to the device. It must match a name for one of the defined material properties.
  \item[NORMAL] Normal vectors on the front surface of target.
  \item[RTI] Response time index value for a sprinkler.
  \item[SETPOINT] Activation value for detectors. For smoke detectors, this is obscuration; for heat detectors or sprinklers, this is temperature.
  \item[SPRAY\_DENSITY] Amount of water dispersed by the sprinkler.
\end{description}
\end{adjustwidth}

\vspace{\baselineskip}

\begin{longtable}{@{\extracolsep{\fill}}|l|l|l|l|l|}
\caption[Boundary file parameters ({\ct DEVC} namelist group)]{For more information see Section~\ref{info:BNDF}.}
\label{tbl:DEVC} \\
\hline
\multicolumn{5}{|c|}{{\ct DEVC} (Device Parameters)} \\
\hline \hline
\endfirsthead
\caption[]{Continued} \\
\hline
\multicolumn{5}{|c|}{{\ct DEVC} (Device Parameters)} \\
\hline \hline
\endhead
{\ct COMP\_ID}            & Character   & Section \ref{info:outputquantities}     &                   &                 \\ \hline
{\ct DEVC\_TYPE}      & {\ct SELECTION LIST}   & Section \ref{info:outputquantities}     &                   &                 \\ \hline
{\ct ID}      		      & Character   & Section \ref{info:BNDF}                 &                   &                 \\ \hline
{\ct INTERNAL\_TEMPERATURE\_LOCATION}     & Real        & Section \ref{info:outputquantities}     &                   &                 \\ \hline
{\ct LOCATION}       & Real        & Section \ref{info:BNDF}                 & m                 &                 \\ \hline
{\ct MATERIAL\_ID}            & Character   & Section \ref{info:BNDF}                 &                   &                 \\ \hline
{\ct NORMAL}         & Real        & Section \ref{info:BNDF}                 &                   &                 \\ \hline
{\ct RTI}                 & Real        & Section \ref{info:BNDF}                 & m-s$^{0.5}$       &                 \\ \hline
{\ct SETPOINT}             & Real        & Section \ref{info:outputquantities}     & $^\circ$C $\mid$ 1/m  &                 \\ \hline
{\ct SPRAY\_DENSITY}      & Real        & Section \ref{info:outputquantities}     & m/s               &                 \\ \hline
\end{longtable}

\noindent Example:
\begin{lstlisting}
&DEVC 
	ID='Targ 1', COMP_ID='Comp 1', DEVC_TYPE='PLATE', 
	LOCATION=2.5,2.5,0., NORMAL=0.,0.,1.,
	MATL_ID='Concrete', INTERNAL_TEMPERATURE_LOCATION=0.5 /

&DEVC 
	ID='Sprinkler 1', COMP_ID='Comp 1', DEVC_TYPE='SPRINKLER',
	LOCATION=3.,3.,2.97, SETPOINT=73.88892, RTI=100., SPRAY_DENSITY=7.E-5 /
\end{lstlisting}

\vspace{\baselineskip}


\section{\texorpdfstring{{\tt FIRE}}{FIRE} (Fire Parameters)}

FIRE places a fire into a compartment. The conditions and properties of a fire can be specified. A target can be potentially ignited.

\begin{adjustwidth}{1cm}{0cm}
\begin{description}
  \item[AREA] Cross-sectional area of the base of the fire.
  \item[CARBON] Stoichiometry of the fuel (carbon).
  \item[CHLORINE] Stoichiometry of the fuel (chlorine).
  \item[COMP\_ID] The compartment in which the fire is located.
  \item[CO\_YIELD] Carbon monoxide yields (kg of CO/kg fuel burned).
  \item[DEVC\_ID] User-specified device (target) associated with this fire.
  \item[HEAT\_OF\_COMBUSTION] Heat of combustion of the fuel.
  \item[HRR\_RAMP\_ID] Name of a ramp that describes the heat release rate as a function of time.
  \item[HYDROGEN] Stoichiometry of the fuel (hydrogen).
  \item[ID] User-specified name for the fire.
  \item[IGNITION\_CRITERION] Mean of ignition to the target. Three options are available in the selection list: 1) FLUX, 2) TIME, and 3) TEMPERATURE. 
  \item[LOCATION] Location associated to the center of the base of the fire relative to the lower left front corner of the compartment.
  \item[NITROGEN] Stoichiometry of the fuel (nitrogen).
  \item[OXYGEN] Stoichiometry of the fuel (oxygen).
  \item[PF\_CO\_YIELD] Carbon monoxide yields after flashover.
  \item[PF\_SOOT\_YIELD] Soot monoxide yields after flashover.
  \item[PF\_TRACE\_YIELD] Trace species yields after flashover.
  \item[POST\_FLASHOVER] Logical statement to indication potential existence of flashover condition in a simulation.
  \item[RADIATIVE\_FRACTION] Radiation fraction for the fuel. Sum of radiation fraction and convection fraction is added to unity. 
  \item[SETPOINT] Ignition value for target associated to either FLUX, TIME, or TEMPERATURE.
  \item[SOOT\_YIELD] Soot yields (kg of soot/kg fuel burned).
  \item[TRACE\_YIELD] Trace species yields (kg of trace species/kg fuel burned).
\end{description}
\end{adjustwidth}

\vspace{\baselineskip}

\begin{longtable}{@{\extracolsep{\fill}}|l|l|l|l|l|}
\caption[Boundary file parameters ({\ct FIRE} namelist group)]{For more information see Section~\ref{info:BNDF}.}
\label{tbl:FIRE} \\
\hline
\multicolumn{5}{|c|}{{\ct FIRE} (Fire Parameters)} \\
\hline \hline
\endfirsthead
\caption[]{Continued} \\
\hline
\multicolumn{5}{|c|}{{\ct FIRE} (Fire Parameters)} \\
\hline \hline
\endhead
{\ct AREA}                 & Real        & Section \ref{info:outputquantities}     & m$^2$                       &                 \\ \hline
{\ct CARBON}               & Integer     & Section \ref{info:BNDF}                 &                             &                 \\ \hline
{\ct CHLORINE}             & Integer     & Section \ref{info:BNDF}                 &                             &                 \\ \hline
{\ct COMP\_ID}             & Character   & Section \ref{info:BNDF}                 &                             &                 \\ \hline
{\ct CO\_YIELD}            & Real        & Section \ref{info:BNDF}                 &                             &                 \\ \hline
{\ct DEVC\_ID}             & Character   & Section \ref{info:BNDF}                 &                             &                 \\ \hline
{\ct HEAT\_OF\_COMBUSTION} & Real        & Section \ref{info:BNDF}                 & kJ/kg                       &                 \\ \hline
{\ct HRR\_RAMP\_ID}            & Character   & Section \ref{info:BNDF}                 &                             &                 \\ \hline
{\ct HYDROGEN}             & Integer     & Section \ref{info:BNDF}                 &                             &                 \\ \hline
{\ct ID}                   & Character   & Section \ref{info:BNDF}                 &                             &                 \\ \hline
{\ct IGNITION\_CRITERION}  & {\ct SELECTION LIST}        & Section \ref{info:BNDF}                 &                             &                 \\ \hline
{\ct LOCATION}        & Real        & Section \ref{info:BNDF}                 & m                           &                 \\ \hline
{\ct NITROGEN}             & Integer     & Section \ref{info:BNDF}                 &                             &                 \\ \hline
{\ct OXYGEN}               & Integer     & Section \ref{info:BNDF}                 &                             &                 \\ \hline
{\ct PF\_CO\_YIELD}        & Real        & Section \ref{info:BNDF}                 &                             &                 \\ \hline
{\ct PF\_SOOT\_YIELD}      & Real        & Section \ref{info:BNDF}                 &                             &                 \\ \hline
{\ct PF\_TRACE\_YIELD}     & Real        & Section \ref{info:BNDF}                 &                             &                 \\ \hline
{\ct POST\_FLASHOVER}      & Logical     & Section \ref{info:BNDF}                 &                             &                 \\ \hline
{\ct RADIATIVE\_FRACTION}  & Real        & Section \ref{info:BNDF}                 &                             &                 \\ \hline
{\ct SETPOINT}             & Real        & Section \ref{info:BNDF}                 & s $\mid$ $^\circ$C $\mid$ kW/m$^2$  &                 \\ \hline
{\ct SOOT\_YIELD}          & Real        & Section \ref{info:BNDF}                 &                             &                 \\ \hline
{\ct TRACE\_YIELD}         & Real        & Section \ref{info:BNDF}                 &                             &                 \\ \hline
\end{longtable}

\begin{lstlisting}
&FIRE 
	ID='Cushion', COMP_ID='Comp 1', LOCATION='2.5,2.5,0.,
	CARBON=9, HYDROGEN=6, OXYGEN=2, NITROGEN=2,
	HEAT_OF_COMBUSTION=50000, RADIATIVE_FRACTION=0.33,
	SOOT_YIELD=0.227, CO_YIELD=0.0846, HRR_RAMP_ID='Cushion Fire',
	DEVC_ID='Curtain',  IGNITION_CRITERION='TIME', SETPOINT=60 /
\end{lstlisting}

\vspace{\baselineskip}



\section{\texorpdfstring{{\tt HEAD}}{HEAD} (Header Parameters)}

HEAD is used to set up an input file for a simulation.

\begin{adjustwidth}{1cm}{0cm}
\begin{description}
  \item[VERSION] Version of CFAST being used.
  \item[TITLE] Job title assigned to the simulation.
\end{description}
\end{adjustwidth}

\vspace{\baselineskip}

\begin{longtable}{@{\extracolsep{\fill}}|l|l|l|l|l|}
\caption[Boundary file parameters ({\ct SLCF} namelist group)]{For more information see Section~\ref{info:BNDF}.}
\label{tbl:HEAD} \\
\hline
\multicolumn{5}{|c|}{{\ct HEAD} (Header Parameters)} \\
\hline \hline
\endfirsthead
\caption[]{Continued} \\
\hline
\multicolumn{5}{|c|}{{\ct HEAD} (Header Parameters)} \\
\hline \hline
\endhead
{\ct VERSION}         & Integer     & Section \ref{info:BNDF}                 &           &                 \\ \hline
{\ct TITLE}           & Character   & Section \ref{info:BNDF}                 &           &                 \\ \hline
\end{longtable}

\begin{lstlisting}
&HEAD 
	VERSION=7300', TITLE='Example input file in NAMELIST format' /
\end{lstlisting}

\vspace{\baselineskip}


\section{\texorpdfstring{{\tt INIT}}{INIT} (Initial Conditions)}

INIT defines initial conditions for a simulation.

\begin{adjustwidth}{1cm}{0cm}
\begin{description}
  \item[PRESSURE] Ambient pressure outside the building.
  \item[RELATIVE\_HUMIDITY] Ambient relative humidity outside the building.
  \item[TEMPERATURE] Ambient temperature of the building. Two components are required: 1) outside temperature and 2) inside temperature.
\end{description}
\end{adjustwidth}

\vspace{\baselineskip}

\begin{longtable}{@{\extracolsep{\fill}}|l|l|l|l|l|}
\caption[Boundary file parameters ({\ct INIT} namelist group)]{For more information see Section~\ref{info:BNDF}.}
\label{tbl:INIT} \\
\hline
\multicolumn{5}{|c|}{{\ct INIT} (Initial Conditions)} \\
\hline \hline
\endfirsthead
\caption[]{Continued} \\
\hline
\multicolumn{5}{|c|}{{\ct INIT} (Initial Conditions)} \\
\hline \hline
\endhead
{\ct PRESSURE}        & Real   & Section \ref{info:BNDF}                 & Pa        &                 \\ \hline
{\ct RELATIVE\_HUMIDITY}   & Real   & Section \ref{info:BNDF}                 & \%        &                 \\ \hline
{\ct TEMPERATURE}     & Real   & Section \ref{info:BNDF}                 & $^\circ$C &                 \\ \hline
\end{longtable}

\begin{lstlisting}
&INIT 
	PRESSURE=101325, RELATIVE_HUMIDITY=50, TEMPERATURE=20.,20. /
\end{lstlisting}

\vspace{\baselineskip}


\section{\texorpdfstring{{\tt ISOF}}{ISOF} (Isosurface Parameters)}

\begin{adjustwidth}{1cm}{0cm}
\begin{description}
  \item[COMP\_ID] Name of the compartment.
  \item[VALUE] Gas temperature to be displayed as an iso-surface.
\end{description}
\end{adjustwidth}

\vspace{\baselineskip}

\begin{longtable}{@{\extracolsep{\fill}}|l|l|l|l|l|}
\caption[Boundary file parameters ({\ct ISOF} namelist group)]{For more information see Section~\ref{info:BNDF}.}
\label{tbl:ISOF } \\
\hline
\multicolumn{5}{|c|}{{\ct ISOF} (Isosurface Parameters)} \\
\hline \hline
\endfirsthead
\caption[]{Continued} \\
\hline
\multicolumn{5}{|c|}{{\ct ISOF} (Isosurface Parameters)} \\
\hline \hline
\endhead
{\ct COMP\_ID}        & Character   & Section \ref{info:BNDF}                 &           &                 \\ \hline
{\ct VALUE}             & Real        & Section \ref{info:BNDF}                 & $^\circ$C &                 \\ \hline
\end{longtable}

\begin{lstlisting}
&INIT 
	COMP\_ID='COMP_ID', VALUE=100. /
\end{lstlisting}

\vspace{\baselineskip}


\section{\texorpdfstring{{\tt MATL}}{MATL} (Material Properties)}

MATL defines thermal and phyiscal properties for a single material that may be referenced as a compartment surface material, target material, or fire object. Each name must be unique within a single input file.

\begin{adjustwidth}{1cm}{0cm}
\begin{description}
  \item[CONDUCTIVITY] Thermal conductivity of the material.
  \item[DENSITY] Density of the material.
  \item[EMISSIVITY] Surface emissivity of the material.
  \item[SPECIFIC\_HEAT] Specific heat of the material.
  \item[ID] User-specified name for the material.
  \item[THICKNESS] Thickness of the material.
\end{description}
\end{adjustwidth}

\vspace{\baselineskip}

\begin{longtable}{@{\extracolsep{\fill}}|l|l|l|l|l|}
\caption[Boundary file parameters ({\ct MATL} namelist group)]{For more information see Section~\ref{info:BNDF}.}
\label{tbl:MATL} \\
\hline
\multicolumn{5}{|c|}{{\ct MATL} (Material Properties)} \\
\hline \hline
\endfirsthead
\caption[]{Continued} \\
\hline
\multicolumn{5}{|c|}{{\ct MATL} (Material Properties)} \\
\hline \hline
\endhead
{\ct CONDUCTIVITY}        & Real 	 & Section \ref{info:BNDF}                 & kW/(m-K)  &                 \\ \hline
{\ct DENSITY}             & Real 	 & Section \ref{info:BNDF}                 & kg/m$^3$  &                 \\ \hline
{\ct EMISSIVITY}          & Real	 & Section \ref{info:BNDF}                 &           &                 \\ \hline
{\ct SPECIFIC\_HEAT}      & Real	 & Section \ref{info:BNDF}                 & kJ/(kg-K) &                 \\ \hline
{\ct ID}                  & Character    & Section \ref{info:BNDF}                 &           &                 \\ \hline
{\ct THICKNESS}           & Real  	 & Section \ref{info:BNDF}                 & m         &                 \\ \hline
\end{longtable}

\begin{lstlisting}
&MATL 
	ID='CONCRETE', CONDUCTIVITY=1.75, SPECIFIC_HEAT=1000., DENSITY=2200., EMISSIVITY=0.94, THICKNESS=0.15 /
\end{lstlisting}

\vspace{\baselineskip}


\section{\texorpdfstring{{\tt MISC}}{MISC} (Miscellaneous Parameters)}

MISC defines global miscellaneous input parameters. It contains parameters that do not logically fit into any other category.

\begin{adjustwidth}{1cm}{0cm}
\begin{description}
  \item[ADIABATIC] Logical statement to set all surfaces to be adiabatic conditions.
  \item[MAX\_TIME\_STEP] Maximum time step.
  \item[LOWER\_OXYGEN\_LIMIT] Lower oxygen limit for combustion.
\end{description}
\end{adjustwidth}

\vspace{\baselineskip}

\begin{longtable}{@{\extracolsep{\fill}}|l|l|l|l|l|}
\caption[Boundary file parameters ({\ct MISC} namelist group)]{For more information see Section~\ref{info:BNDF}.}
\label{tbl:MISC} \\
\hline
\multicolumn{5}{|c|}{{\ct MISC} (Miscellaneous Parameters)} \\
\hline \hline
\endfirsthead
\caption[]{Continued} \\
\hline
\multicolumn{5}{|c|}{{\ct MISC} (Miscellaneous Parameters)} \\
\hline \hline
\endhead
{\ct ADIABATIC}            & Logical     & Section \ref{info:BNDF}                 &           & {\ct .FALSE.}   \\ \hline
{\ct MAX\_TIME\_STEP}      & Real        & Section \ref{info:BNDF}                 & s         & 0.2            \\ \hline
{\ct LOWER\_OXYGEN\_LIMIT} & Real        & Section \ref{info:BNDF}                 &           & 0.15            \\ \hline
\end{longtable}

\begin{lstlisting}
&MISC 
	ADIABATIC='FALSE'', MAX\_TIME\_STEP=0.2, LOWER\_OXYGEN\_LIMIT=0.15 /
\end{lstlisting}

\vspace{\baselineskip}


\section{\texorpdfstring{{\tt RAMP}}{RAMP} (Ramp Function Parameters)}

RAMP specifies a time-varying or a height-varying variable for a vent or a fire object.

\begin{adjustwidth}{1cm}{0cm}
\begin{description}
  \item[F] Fraction value [0-1] either varying with time or height.
  \item[H] Height.
  \item[HRR] Heat release rate varying with time.
  \item[ID] User-specified name for the ramp.
  \item[T] Time.
\end{description}
\end{adjustwidth}

\begin{longtable}{@{\extracolsep{\fill}}|l|l|l|l|l|}
\caption[Boundary file parameters ({\ct RAMP} namelist group)]{For more information see Section~\ref{info:BNDF}.}
\label{tbl:RAMP} \\
\hline
\multicolumn{5}{|c|}{{\ct RAMP} (Ramp Function Parameters)} \\
\hline \hline
\endfirsthead
\caption[]{Continued} \\
\hline
\multicolumn{5}{|c|}{{\ct RAMP} (Ramp Function Parameters)} \\
\hline \hline
\endhead
{\ct F}               & Real        & Section \ref{info:BNDF}                 &           &                 \\ \hline
{\ct H}               & Real        & Section \ref{info:BNDF}                 &  m         &                 \\ \hline
{\ct HRR}             & Real        & Section \ref{info:BNDF}                 & W         &                 \\ \hline
{\ct ID}       	      & Character   & Section \ref{info:BNDF}                 &           &                 \\ \hline
{\ct T}               & Real        & Section \ref{info:BNDF}                 &  s        &                 \\ \hline
\end{longtable}

\begin{lstlisting}
&RAMP 
	ID='Cushion Fire', 
	T  =0.,60.,120.,80.,240.,300.,360.,420.,480.,540.,1800.,
	HRR=0.,100000.,150000.,200000.,150000.,125000.,100000.,90000.,80000.,75000.,75000. /  
\end{lstlisting}

\vspace{\baselineskip}


\section{\texorpdfstring{{\tt SLCF}}{SLCF} (Slice File Parameters)}

SLCF specifies display animations of various gas phase quantities in a plane or volume in a compartment. The output frequency of the slice files is controlled by the SMOKEVIEW input on the TIMES input line.

\begin{adjustwidth}{1cm}{0cm}
\begin{description}
  \item[COMP\_ID] Name of the comparment.
  \item[DOMAIN] Either 2-D for a single slice of temperature data or 3-D for all three dimensions.
  \item[PLANE] For 2-D slices, either X, Y, or Z specifying a slice perpendicular to the chosen plane.
  \item[POSITION] Distance along the selected axis relative to the compartment origin.
\end{description}
\end{adjustwidth}

\vspace{\baselineskip}

\begin{longtable}{@{\extracolsep{\fill}}|l|l|l|l|l|}
\caption[Boundary file parameters ({\ct SLCF} namelist group)]{For more information see Section~\ref{info:BNDF}.}
\label{tbl:SLCF} \\
\hline
\multicolumn{5}{|c|}{{\ct SLCF} (Slice File Parameters)} \\
\hline \hline
\endfirsthead
\caption[]{Continued} \\
\hline
\multicolumn{5}{|c|}{{\ct SLCF} (Slice File Parameters)} \\
\hline \hline
\endhead
{\ct COMP\_ID}        & Character   & Section \ref{info:BNDF}                 &           &                 \\ \hline
{\ct DOMAIN}            & Character   & Section \ref{info:BNDF}                 &           &                 \\ \hline
{\ct PLANE}             & Character   & Section \ref{info:BNDF}                 &           &                 \\ \hline
{\ct POSITION}          & Real        & Section \ref{info:BNDF}                 &           &                 \\ \hline
\end{longtable}

\begin{lstlisting}
&SLCF 
	COMP_ID='Comp 1', DOMAIN='2-D', PLANE='X', POSITION=2.5 /
\end{lstlisting}

\vspace{\baselineskip}


\section{\texorpdfstring{{\tt TIME}}{TIME} (Time Parameters)}

\begin{adjustwidth}{1cm}{0cm}
\begin{description}
  \item[PRINT] Time interval for printed output.
  \item[SIMULATION] Total simulation time.
  \item[SMOKEVIEW] Time interval for smokeview output.
  \item[SPREADSHEET] Time interval for spreadsheet output.
\end{description}
\end{adjustwidth}

\vspace{\baselineskip}

\begin{longtable}{@{\extracolsep{\fill}}|l|l|l|l|l|}
\caption[Boundary file parameters ({\ct TIME} namelist group)]{For more information see Section~\ref{info:BNDF}.}
\label{tbl:TIME} \\
\hline
\multicolumn{5}{|c|}{{\ct TIME} (Time Parameters)} \\
\hline \hline
\endfirsthead
\caption[]{Continued} \\
\hline
\multicolumn{5}{|c|}{{\ct TIME} (Time Parameters)} \\
\hline \hline
\endhead
{\ct PRINT}             & Integer   & Section \ref{info:BNDF}                 & s         & 50              \\ \hline
{\ct SIMULATION}        & Integer   & Section \ref{info:BNDF}                 & s         &                 \\ \hline
{\ct SMOKEVIEW}         & Integer   & Section \ref{info:BNDF}                 & s         & 10              \\ \hline
{\ct SPREADSHEET}       & Integer   & Section \ref{info:BNDF}                 & s         & 10              \\ \hline
\end{longtable}

\begin{lstlisting}
&TIME 
	SIMULATION=1800, PRINT=60, SPREADSHEET=10, SMOKEVIEW=30 /
\end{lstlisting}

\vspace{\baselineskip}



\section{\texorpdfstring{{\tt VENT}}{VENT} (Vent Parameters)}

VENT specfies ceiling, floor, wall, and mechanical vents. Please refer to cooresponding section for input details associated to each type of vent.

\begin{adjustwidth}{1cm}{0cm}
\begin{description}
  \item[AREA] Cross-sectional area of the vent (ceiling/floor vent).
  \item[AREAS] Cross-sectional area of the vent (mechanical vent). Two components are required.
  \item[BOTTOM] Location of the bottom of the vent messaured from compartment floor.
  \item[COMP\_IDS] Name of compartments. Two names are required: 1) a first compartment and 2) an adjacent compartment. 
  \item[CRITERION] Mean of activate to a vent opening. Three options are available in the selection list: 1) FLUX, 2) TIME, and 3) TEMPERATURE.
  \item[CUTOFFS] Cutoff pressure. Two components are requiared: 1) first pressure indicates a decrease to fan flowrate and 2) second pressure indicates zero fan flowrate. For pressure in between the given values, fan flowrate decreases linearly.
  \item[DEVC\_ID] Name of device associated with this vent and used to judge vent opening or closing based on  incident heat flux or temperature.
  \item[FACE] Surface within first compartment containing the wall vent. Four faces are available in the selection list: 1) FRONT, 2) RIGHT, 3) BACK, and 4) LEFT.
  \item[FILTERING\_RAMP\_ID] A variable only for mechanical vent. Name of a ramp that specify the filtering efficiency as a function of time.
  \item[FLOW] Fan flowrate from first compartment to the adjacent compartment.
  \item[HEIGHTS] A variable only for mechanical vent. Two components are required: 1) height of the vent relative to the floor in the first compartment and 2) height of the vent relative to the floor in the adjacent compartment.
  \item[ID] User-specified name of the vent.
  \item[OFFSETS] A variable only for mechanical vent. Two components are required. Distances to the center of the vent relative to the lower left front corner of the compartment in x (width) direction and y (depth) direction, respectively.
  \item[OPENING\_RAMP\_ID] Name of a ramp that specify the opening fraction to a vent as a function of time.
  \item[SETPOINT] Value that triggers the opening of the vent and the value can be in heat flux, time, or temperature.
  \item[SHAPE] Shape of a ceiling/floor vent. Two options are availble in the selection list: 1) ROUND and 2) SQUARE.
  \item[TOP] Location of the top of the vent measured from the compartment floor.
  \item[TYPE] Vent type. Four options are available in the selection list: 1) CEILING, 2) FLOOR, 3) MECHANICAL, and 4) WALL.
  \item[WIDTH] Width of the vent.
\end{description}
\end{adjustwidth}

\vspace{\baselineskip}

\begin{longtable}{@{\extracolsep{\fill}}|l|l|l|l|l|}
\caption[Boundary file parameters ({\ct VENT} namelist group)]{For more information see Section~\ref{info:BNDF}.}
\label{tbl:VENT} \\
\hline
\multicolumn{5}{|c|}{{\ct VENT} (Vent Parameters)} \\
\hline \hline
\endfirsthead
\caption[]{Continued} \\
\hline
\multicolumn{5}{|c|}{{\ct VENT} (Vent Parameters)} \\
\hline \hline
\endhead
{\ct AREA}      	  & Real  	& Section \ref{info:BNDF}                 & m$^2$                       &                 \\ \hline
{\ct AREAS}      	  & Real  	& Section \ref{info:BNDF}                 & m$^2$                       &                 \\ \hline
{\ct BOTTOM}       	  & Real  	& Section \ref{info:BNDF}                 & m                           &                 \\ \hline
{\ct COMP\_IDS}    	  & Character   & Section \ref{info:BNDF}                 &                             &                 \\ \hline
{\ct CRITERION}           & {\ct SELECTION LIST}   & Section \ref{info:BNDF}                 &                             &                 \\ \hline
{\ct CUTOFFS}         & Real  	& Section \ref{info:BNDF}                 & Pa                          &                 \\ \hline
{\ct DEVC\_ID}            & Character   & Section \ref{info:BNDF}                 &                             &                 \\ \hline
{\ct FACE}       	  & {\ct SELECTION LIST}   & Section \ref{info:BNDF}                 &                             &                 \\ \hline
{\ct FILTERING\_RAMP\_ID}     	  & Character   & Section \ref{info:BNDF}                 &                             &                 \\ \hline
{\ct FLOW}      	  & Real  	& Section \ref{info:BNDF}                 & m$^3$/s                     &                 \\ \hline
{\ct HEIGHTS}         & Real  	& Section \ref{info:BNDF}                 & m                           &                 \\ \hline
{\ct ID}       		  & Character   & Section \ref{info:BNDF}                 &                             &                 \\ \hline
{\ct OFFSETS}         & Real  	& Section \ref{info:BNDF}                 & m                           &                 \\ \hline
{\ct OPENING\_RAMP\_ID}   & Real  	& Section \ref{info:BNDF}                 &                             &                 \\ \hline
{\ct SETPOINT}            & Real  	& Section \ref{info:BNDF}                 & s $\mid$ $^\circ$C $\mid$ kW/m$^2$ &                 \\ \hline
{\ct SHAPE}     	  & Character   & Section \ref{info:BNDF}                 &                             &                 \\ \hline
{\ct TOP}                 & Real  	& Section \ref{info:BNDF}                 & m                           &                 \\ \hline
{\ct TYPE}                & {\ct SELECTION LIST}   & Section \ref{info:BNDF}                 &                             &                 \\ \hline
{\ct WIDTH}               & Real  	& Section \ref{info:BNDF}                 & m                           &                 \\ \hline
\end{longtable}

\vspace{\baselineskip}

\begin{lstlisting}
&VENT 
	ID='Door Outside', TYPE='WALL', COMP_IDS='Comp 1','OUTSIDE', BOTTOM=0., TOP=2., WIDTH=1., FACE='FRONT' / 

&VENT 
	ID='Floor Hole', TYPE='FLOOR', COMP_IDS='Comp 3’,Comp 2', AREA=1., SHAPE='ROUND', OPENING_RAMP_ID='Floor Hole Opening' /

&VENT 
	ID='HVAC in', TYPE='MECHANICAL', COMP_IDS='Comp 2','OUTSIDE', AREAS=0.25,0.25, HEIGHTs=2.75,2.75, FLOW=0.02, OFFSETS=0.,4. /
\end{lstlisting}



\chapter{Running CFAST from a Command Prompt}

The model CFAST can also be run from a Windows command prompt.  CFAST can be run from any folder, and refer to a data file in any other folder. The fires and thermophysical properties have to be in either the data folder, or the executable folder. The data folder is checked first and then the executable folder.

\begin{lstlisting}
[drive1:\][folder1\]cfast [drive2:\][folder2\]project
\end{lstlisting}

The project name will have extensions appended as needed (see below). For example, to run a test case when the CFAST executable is located in c:$\backslash$firemodels$\backslash$cfast7 and the input data file is located in c:$\backslash$data, the following command could be used:

\begin{lstlisting}
c:\firemodels\cfast7\cfast c:\data\testfire0   <<< note that the extension is optional.
\end{lstlisting}

Command line options

\begin{itemize}
\item -k - no interactive keyboard access
\item -i - initialization only
\item -c - compact output
\item -f - full output (c and f are exclusive). Note the interaction of the f and c option. The default for the console output is /c. The default for the file output is /f. This default action can be overwritten by explicitly including the /f or /c option.
\item -n - net heat flux option
\item -v - validation output (outputs a modified set of spreadsheet files with different column headers designed to facilitate automated analysis of the output)
\end{itemize}


\label{last_page}


