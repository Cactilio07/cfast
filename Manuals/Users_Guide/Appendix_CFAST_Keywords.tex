\chapter{CFAST Text-based Input File}

CFAST is a Fortran program that reads a text-based input file that uses Fortran {\em namelist} records. The graphical user interface, CEdit, writes this file, but you can manually write this file as well. This appendix lists the names of the parameters, which are organized into groups that roughly coincide with the tabs in the graphical user interface. The namelist group names are listed in Table~\ref{tbl:namelistgroups}.

\begin{table}[ht]
\begin{center}
\caption{CFAST Input File Keywords}
\label{tbl:namelistgroups}
\begin{tabular}{|c|l|c|c|}
\hline
Group Name  & Namelist Group Description& Reference Section & Parameter Table  \\ \hline
{\ct COMP}   & Compartment Parameters         & \ref{info:COMP} & \ref{tbl:COMP}  \\ \hline
{\ct CONN}   & Surface Connection Parameters          & \ref{info:CONN} & \ref{tbl:CONN}  \\ \hline
{\ct DEVC}    & Device Parameters          & \ref{info:DEVC} & \ref{tbl:DEVC}  \\ \hline
{\ct FIRE}     & Fire Parameters  & \ref{info:FIRE} & \ref{tbl:FIRE}  \\ \hline
{\ct HEAD}    & Input File Header            & \ref{info:HEAD} & \ref{tbl:HEAD}  \\ \hline
{\ct INIT}      & Initial Conditions            & \ref{info:INIT} & \ref{tbl:INIT}  \\ \hline
{\ct ISOF}     & Isosurface File Outputs            & \ref{info:ISOF} & \ref{tbl:ISOF}  \\ \hline
{\ct MATL}    & Material Properties           & \ref{info:MATL} & \ref{tbl:MATL}  \\ \hline
{\ct MISC}     & Miscellaneous    & \ref{info:MISC} & \ref{tbl:MISC}  \\ \hline
{\ct RAMP}    & Ramp Profiles            & \ref{info:RAMP} & \ref{tbl:RAMP}  \\ \hline
{\ct SLCF}     & Slice File Outputs       & \ref{info:SLCF} & \ref{tbl:SLCF}  \\ \hline
{\ct TIME}     & Simulation Time            & \ref{info:TIME} & \ref{tbl:TIME}  \\ \hline
{\ct VENT}     & Vent Parameters              & \ref{info:VENT} & \ref{tbl:VENT}  \\ \hline
\end{tabular}
\end{center}
\end{table}

Examples of each of the inputs are included in the sections that follow.  All examples are taken from the sample input file {\ct Users\_Guide\_Example.cfast} included with the CFAST distribution. Note that the namelist formatted input file has the suffix {\ct .cfast} as opposed to the older {\ct .in} files. Following are some general rules about the CFAST input file:
\begin{itemize}
\item Many of the listed keywords are mutually exclusive. Repeated entry of some keywords can cause the program to either fail or run in an unpredictable manner.
\item Use of some keywords triggers the code to operate in a certain mode/condition. For example, specifying {\ct ADIABATIC} to be {\ct TRUE} triggers the code to treat all compartment surfaces to be perfectly insulated.
\item Multple inputs are required whenever the keyword is in plural form --- keywords ending with an \textbf{s}. For example, the {\ct TEMPERATURES} input in the {\ct INIT} keyword requires that two temperatures are required (in this case, one for exterior ambient temperature and one for interior ambient temperature). In the case of missing inputs, an error message will be generated to assist users for trouble-shooting.
\item Default values to inputs are assigned to some of the keywords to facilitate the set up of an input file. For instance, Table \ref{tbl:MISC} shows that the {\ct LOWER\_OXYGEN\_LIMIT} has a default value of {\ct 0.15}. This value is taken from the SFPE handbook \cite{SFPE:2003} for general use and it implies that the burning rate will be limited/reduced when the oxygen level is reaches 15\%. Users should review the applicability of any default values for their simulation.
\end{itemize}


\clearpage

\section{Simulation Environment, Namelist Groups \texorpdfstring{{\tt HEAD}}{HEAD}, \texorpdfstring{{\tt TIME}}{TIME}, \texorpdfstring{{\tt INIT}}{INIT}, \texorpdfstring{{\tt MISC}}{MISC}}

\renewcommand{\tabcolsep}{.1in}
\begin{longtable}{|l|l|l|l|l|}
\caption[Header Parameters ({\ct HEAD} namelist group)]{For more information see Section~\ref{info:HEAD}.}
\label{tbl:HEAD} \\
\hline
\multicolumn{5}{|c|}{{\ct HEAD} (Header Parameters)} \\
\hline \hline
\endfirsthead
\caption[]{Continued} \\
\hline
\multicolumn{5}{|c|}{{\ct HEAD} (Header Parameters)} \\
\hline \hline
\endhead
\parbox{1.5in}{\bf Parameter}    & \parbox{1in}{\bf Type}  & \parbox{1in}{\bf Reference}  & \parbox{1in}{\bf Units}  & \parbox{1in}{\bf Default Value} \\ \hline
{\ct VERSION}         & Integer     & Section \ref{info:HEAD}                 &           &                 \\ \hline
{\ct TITLE}           & Character   & Section \ref{info:HEAD}                 &           &                 \\ \hline
\end{longtable}



\begin{longtable}{@{\extracolsep{\fill}}|l|l|l|l|l|}
\caption[Time Parameters ({\ct TIME} namelist group)]{For more information see Section~\ref{info:TIME}.}
\label{tbl:TIME} \\
\hline
\multicolumn{5}{|c|}{{\ct TIME} (Time Parameters)} \\
\hline \hline
\endfirsthead
\caption[]{Continued} \\
\hline
\multicolumn{5}{|c|}{{\ct TIME} (Time Parameters)} \\
\hline \hline
\endhead
\parbox{1.5in}{\bf Parameter}    & \parbox{1in}{\bf Type}  & \parbox{1in}{\bf Reference}  & \parbox{1in}{\bf Units}  & \parbox{1in}{\bf Default Value} \\ \hline
{\ct PRINT}             & Integer   & Section \ref{info:TIME}                 & s         & 60              \\ \hline
{\ct SIMULATION}        & Integer   & Section \ref{info:TIME}                 & s         &                 \\ \hline
{\ct SMOKEVIEW}         & Integer   & Section \ref{info:TIME}                 & s         & 15              \\ \hline
{\ct SPREADSHEET}       & Integer   & Section \ref{info:TIME}                 & s         & 15              \\ \hline
\end{longtable}



\begin{longtable}{@{\extracolsep{\fill}}|l|l|l|l|l|}
\caption[Initial Conditions ({\ct INIT} namelist group)]{For more information see Section~\ref{info:INIT}.}
\label{tbl:INIT} \\
\hline
\multicolumn{5}{|c|}{{\ct INIT} (Initial Conditions)} \\
\hline \hline
\endfirsthead
\caption[]{Continued} \\
\hline
\multicolumn{5}{|c|}{{\ct INIT} (Initial Conditions)} \\
\hline \hline
\endhead
\parbox{1.5in}{\bf Parameter}    & \parbox{1in}{\bf Type}  & \parbox{1in}{\bf Reference}  & \parbox{1in}{\bf Units}  & \parbox{1in}{\bf Default Value} \\ \hline
{\ct PRESSURE}        & Real   & Section \ref{info:INIT}                 & Pa        & 101325         \\ \hline
{\ct RELATIVE\_HUMIDITY}   & Real   & Section \ref{info:INIT}                 & \%        &   50          \\ \hline
{\ct TEMPERATURES(1:2)}     & Real   & Section \ref{info:INIT}                 & $^\circ$C &    20, 20          \\ \hline
\end{longtable}



\begin{longtable}{@{\extracolsep{\fill}}|l|l|l|l|l|}
\caption[Miscellaneous Parameters ({\ct MISC} namelist group)]{For more information see Section~\ref{info:MISC}.}
\label{tbl:MISC} \\
\hline
\multicolumn{5}{|c|}{{\ct MISC} (Miscellaneous Parameters)} \\
\hline \hline
\endfirsthead
\caption[]{Continued} \\
\hline
\multicolumn{5}{|c|}{{\ct MISC} (Miscellaneous Parameters)} \\
\hline \hline
\endhead
\parbox{1.5in}{\bf Parameter}    & \parbox{1in}{\bf Type}  & \parbox{1in}{\bf Reference}  & \parbox{1in}{\bf Units}  & \parbox{1in}{\bf Default Value} \\ \hline
{\ct ADIABATIC}            & Logical     & Section \ref{info:MISC}                 &           & {\ct .FALSE.}  \\ \hline
{\ct MAX\_TIME\_STEP}      & Real        & Section \ref{info:MISC}                 & s         &   2            \\ \hline
{\ct LOWER\_OXYGEN\_LIMIT} & Real        & Section \ref{info:MISC}                 &           &   0.15          \\ \hline
\end{longtable}

\noindent Examples:
\begin{lstlisting}
&HEAD  VERSION = 7300, TITLE = 'Users Guide Example Case' /
&TIME  SIMULATION = 3600., PRINT = 50., SMOKEVIEW = 50., SPREADSHEET = 50. /
&INIT  PRESSURE = 101325., RELATIVE_HUMIDITY = 50., TEMPERATURE = 20., 20. /
&MISC  LOWER_OXYGEN_LIMIT = 0.10 /
\end{lstlisting}


\clearpage

\section{Thermal Properties, Namelist Group \texorpdfstring{{\tt MATL}}{MATL}}


\begin{longtable}{@{\extracolsep{\fill}}|l|l|l|l|l|}
\caption[Thermal Properties ({\ct MATL} namelist group)]{For more information see Section~\ref{info:MATL}.}
\label{tbl:MATL} \\
\hline
\multicolumn{5}{|c|}{{\ct MATL} (Material Properties)} \\
\hline \hline
\endfirsthead
\caption[]{Continued} \\
\hline
\multicolumn{5}{|c|}{{\ct MATL} (Material Properties)} \\
\hline \hline
\endhead
\parbox{1.5in}{\bf Parameter}    & \parbox{1in}{\bf Type}  & \parbox{1in}{\bf Reference}  & \parbox{1in}{\bf Units}  & \parbox{1in}{\bf Default Value} \\ \hline
{\ct CONDUCTIVITY}        & Real 	 & Section \ref{info:MATL}                 & kW/(m$\cdot$K)  &                 \\ \hline
{\ct DENSITY}             & Real 	 & Section \ref{info:MATL}                 & kg/m$^3$  &                 \\ \hline
{\ct EMISSIVITY}          & Real	 & Section \ref{info:MATL}                 &           &   0.9          \\ \hline
{\ct SPECIFIC\_HEAT}      & Real	 & Section \ref{info:MATL}                 & kJ/(kg$\cdot$K) &                 \\ \hline
{\ct ID}                  & Character    & Section \ref{info:MATL}                 &           &                 \\ \hline
{\ct THICKNESS}           & Real  	 & Section \ref{info:MATL}                 & m         &                 \\ \hline
\end{longtable}

\noindent Example:
\begin{lstlisting}
&MATL ID = 'CONCRETE', CONDUCTIVITY = 1.75, SPECIFIC_HEAT = 1., 
      DENSITY = 2200., EMISSIVITY = 0.94, THICKNESS = 0.15 /
\end{lstlisting}


\clearpage
\section{Compartments, Namelist Group \texorpdfstring{{\tt COMP}}{COMP}}

\begin{minipage}{6.5in}
%\begin{longtable}{@{\extracolsep{\fill}}|l|l|l|l|l|}
\begin{longtable}{|l|l|l|l|l@{\extracolsep{\fill}}|}
\caption[Compartment parameters ({\ct COMP} namelist group)]{For more information see Section~\ref{info:COMP}.}
\label{tbl:COMP} \\
\hline
\multicolumn{5}{|c|}{{\ct COMP} (Compartment Parameters)} \\
\hline \hline
\endfirsthead
\caption[]{Continued} \\
\hline
\multicolumn{5}{|c|}{{\ct COMP} (Compartment Parameters)} \\
\hline \hline
\endhead
\parbox{1.5in}{\bf Parameter}    & \parbox{1in}{\bf Type}  & \parbox{1in}{\bf Reference}  & \parbox{1in}{\bf Units}  & \parbox{1in}{\bf Default Value} \\ \hline
{\ct CEILING\_MATL\_ID}     & Character & Section \ref{info:COMP2}                 &           &                 \\ \hline
{\ct DEPTH}                 & Real      & Section \ref{info:COMP}                 & m         &  		\\ \hline
{\ct FLOOR\_MATL\_ID}       & Character & Section \ref{info:COMP2}                 &           &                 \\ \hline
{\ct GRID(1:3)}             & Integer Array   & Section \ref{info:SLCF2}                 &           &                 \\ \hline
{\ct HALL}                  & Logical   & Section \ref{info:COMP3}     &           & {\ct .FALSE.}   \\ \hline
{\ct HEIGHT}                & Real      & Section \ref{info:COMP}     & m         &                 \\ \hline
{\ct ID}                    & Character & Section \ref{info:COMP}     &           &                 \\ \hline
{\ct ORIGIN(1:3)}           & Real Array     & Section \ref{info:COMP}                 & m         &                 \\ \hline
{\ct ROOM\_AREA\_RAMP}
\footnote{For compartments where the cross-sectional area varies with height, the namelist group, {\ct RAMP} can be used. Keywords associated with this feature can be found in Section~\ref{tbl:RAMP}.}  
                            & Character & Section \ref{info:COMP4}                 &           &                 \\ \hline
{\ct SHAFT}                 & Logical   & Section \ref{info:COMP3}                 &           & {\ct .FALSE.}   \\ \hline
{\ct WALL\_MATL\_ID}        & Character & Section \ref{info:COMP2}                 &           &                 \\ \hline
{\ct WIDTH}                 & Real      & Section \ref{info:COMP}     & m         &                 \\ \hline
\end{longtable}
\end{minipage}

\vspace{\baselineskip}
\noindent Example:
\begin{lstlisting}
&COMP ID = 'Comp 1', WIDTH = 5., DEPTH = 5., HEIGHT = 3., 
      ORIGIN(1:3) = 0., 0., 0., CEILING_MATL_ID = 'CONCRETE', 
      WALL_MATL_ID = 'CONCRETE', FLOOR_MATL_ID = 'CONCRETE',
      GRID(1:3) = 50, 50, 50 /
\end{lstlisting}




\clearpage
\section{Vents, Namelist Group \texorpdfstring{{\tt VENT}}{VENT}}

\begin{minipage}{6.5in}
\begin{longtable}{@{\extracolsep{\fill}}|l|l|l|l|l|}
\caption[Vent Parameters ({\ct VENT} namelist group)]{For more information see Section~\ref{info:VENT}.}
\label{tbl:VENT} \\
\hline
\multicolumn{5}{|c|}{{\ct VENT} (Vent Parameters)} \\
\hline \hline
\endfirsthead
\caption[]{Continued} \\
\hline
\multicolumn{5}{|c|}{{\ct VENT} (Vent Parameters)} \\
\hline \hline
\endhead
\parbox{1.5in}{\bf Parameter}    & \parbox{1in}{\bf Type}  & \parbox{1in}{\bf Reference}  & \parbox{1in}{\bf Units}  & \parbox{1in}{\bf Default Value} \\ \hline
{\ct AREA}\footnote{Appropriate for ceiling/floor vent}      	             & Real  	               & Section \ref{info:VENT2}     & m$^2$                    &                                 \\ \hline
{\ct AREAS(1:2)}\footnote{Appropriate for mechanical vent}      	         & Real Array 	           & Section \ref{info:VENT3}     & m$^2$                    &                 \\ \hline
{\ct BOTTOM}\footnote{Appropriate for wall vent}       	             & Real                    & Section \ref{info:VENT}      & m                        &                 \\ \hline
{\ct COMP\_IDS(1:2)}     	     & Character Array         & Section \ref{info:VENT}                 &                             &                 \\ \hline
{\ct CRITERION}\footnote{Input for {\ct CRITERION} must be {\ct FLUX}, {\ct TEMPERATURE}, or {\ct TIME}. An associated {\ct SETPOINT} is required. For {\ct FLUX} or {\ct TEMPERATURE}, an associated ignition target must be specified by {\ct DEVC\_ID}.}           
                                 & Character               & Section \ref{info:VENT}                 &                             &                 \\ \hline
{\ct CUTOFFS(1:2)}         & Real Array 	& Section \ref{info:VENT4}                 & Pa                          &     200, 300     \\ \hline
{\ct DEVC\_ID}            & Character   & Section \ref{info:VENT}                 &                             &                 \\ \hline
{\ct FACE}\footnote{Input for {\ct FACE} must be {\ct RIGHT}, {\ct FRONT}, {\ct LEFT}, or {\ct REAR}.}      	  & Selection List   & Section \ref{info:VENT}                 &                             &                 \\ \hline
{\ct FILTERING\_RAMP\_ID}    	  & Character   & Section \ref{info:RAMP}                 &                             &                 \\ \hline
{\ct FLOW}      	  & Real  	& Section \ref{info:VENT4}                 & m$^3$/s                     &                 \\ \hline
{\ct HEIGHTS(1:2)}         & Real Array 	& Section \ref{info:VENT3}                 & m                           &                 \\ \hline
{\ct ID}       		  & Character   & Section \ref{info:VENT}                 &                             &                 \\ \hline
{\ct OFFSETS(1:2)}         & Real Array 	& Section \ref{info:VENT}                 & m                           &      0, 0        \\ \hline
{\ct OPENING\_RAMP\_ID}   & Character  	& Section \ref{info:RAMP}                 &                             &                 \\ \hline
{\ct SETPOINT}            & Real  	& Section \ref{info:VENT}                 & s $\mid$ $^\circ$C $\mid$ kW/m$^2$ &                 \\ \hline
{\ct SHAPE}\footnote{Input for {\ct SHAPE} must be {\ct ROUND} or {\ct SQUARE}.}     	  & Selection List   & Section \ref{info:VENT2}                 &                             &                 \\ \hline
{\ct TOP}\footnote{Appropriate for wall vent}                 & Real  	& Section \ref{info:VENT}                 & m                           &                 \\ \hline
{\ct TYPE}\footnote{Input for {\ct TYPE} must be {\ct CEILING}, {\ct FLOOR}, {\ct MECHANICAL}, or {\ct WALL}. }                & Selection List    & Section \ref{info:VENT6}                 &                             &                 \\ \hline
{\ct WIDTH}               & Real  	& Section \ref{info:VENT}                 & m                           &                 \\ \hline
\end{longtable}
\end{minipage}


\clearpage
\noindent Examples:
\begin{lstlisting}
&VENT TYPE = 'WALL', ID = 'HVENT 1', COMP_IDS(1:2) = 'Comp 1', 'OUTSIDE',
      WIDTH = 1., TOP = 2., BOTTOM = 0., OFFSETS(1:2) = 2., 0., 
      FACE = 'FRONT', CRITERION = 'TIME' /

&VENT TYPE = 'CEILING', ID = 'VVENT 1', COMP_IDS(1:2) = 'Comp 3', 'Comp 2',
      AREA = 1., SHAPE = 'ROUND', CRITERION= 'TIME', 
      OPENING_RAMP_ID = 'RAMP_VVENT_1'/
&RAMP ID = 'RAMP_VVENT_1', TYPE = 'FRACTION',
      T = 0., 100., 500.,
      F = 0., 0.5, 1. /

&VENT TYPE = 'MECHANICAL', ID = 'MVENT_1', 
      COMP_IDS(1:2) = 'OUTSIDE', 'Comp 1', AREAS(1:2) = 0.25, 0.25, 
      HEIGHTS(1:2) = 2.75, 2.75, CRITERION = 'TIME', FLOW = 0.02, 
      CUTOFFS(1:2) = 200., 300., OFFSETS = 0., 4. /

\end{lstlisting}




\clearpage
\section{Fires, Namelist Group \texorpdfstring{{\tt FIRE}}{FIRE}}
\label{info:FIRE3}

\begin{minipage}{6.5in}
\begin{longtable}{@{\extracolsep{\fill}}|l|l|l|l|l|}
\caption[Fire Parameters ({\ct FIRE} namelist group)]{For more information see Section~\ref{info:FIRE}.}
\label{tbl:FIRE} \\
\hline
\multicolumn{5}{|c|}{{\ct FIRE} (Fire Parameters)} \\
\hline \hline
\endfirsthead
\caption[]{Continued} \\
\hline
\multicolumn{5}{|c|}{{\ct FIRE} (Fire Parameters)} \\
\hline \hline
\endhead
\parbox{1.5in}{\bf Parameter}    & \parbox{1in}{\bf Type}  & \parbox{1in}{\bf Reference}  & \parbox{1in}{\bf Units}  & \parbox{1in}{\bf Default Value} \\ \hline
{\ct AREA}                 & Real        & Section \ref{info:FIRE}     & m$^2$                       &      0.3       \\ \hline
{\ct CARBON}               & Integer     & Section \ref{info:FIRE}                 &                             &                 \\ \hline
{\ct CHLORINE}             & Integer     & Section \ref{info:FIRE}                 &                             &                 \\ \hline
{\ct COMP\_ID}             & Character   & Section \ref{info:FIRE}                 &                             &                 \\ \hline
{\ct CO\_YIELD}            & Real        & Section \ref{info:FIRE}                 &                             &                 \\ \hline
{\ct DEVC\_ID}             & Character   & Section \ref{info:FIRE}                 &                             &                 \\ \hline
{\ct HEAT\_OF\_COMBUSTION} & Real        & Section \ref{info:FIRE}                 & kJ/kg                       &     50000    \\ \hline
{\ct HRR\_RAMP\_ID}        & Character   &  Section \ref{info:RAMP}                &                             &                 \\ \hline
{\ct HYDROGEN}             & Integer     & Section \ref{info:FIRE}                 &                             &                 \\ \hline
{\ct ID}                   & Character   & Section \ref{info:FIRE}                 &                             &                 \\ \hline
{\ct IGNITION\_CRITERION}\footnote{Must be {\ct FLUX}, {\ct TEMPERATURE}, or {\ct TIME}. An associated {\ct SETPOINT} is required. For {\ct FLUX} or {\ct TEMPERATURE}, an associated ignition target must be specified by {\ct DEVC\_ID}.}
                           & Character   & Section \ref{info:FIRE}                 &                             &                 \\ \hline
{\ct LOCATION}             & Real        & Section \ref{info:FIRE}                 & m                           &                 \\ \hline
{\ct NITROGEN}             & Integer     & Section \ref{info:FIRE}                 &                             &                 \\ \hline
{\ct OXYGEN}               & Integer     & Section \ref{info:FIRE}                 &                             &                 \\ \hline
{\ct PF\_CO\_YIELD}        & Real        & Section \ref{}                 &                             &                 \\ \hline
{\ct PF\_SOOT\_YIELD}      & Real        & Section \ref{}                 &                             &                 \\ \hline
{\ct PF\_TRACE\_YIELD}     & Real        & Section \ref{}                 &                             &                 \\ \hline
{\ct POST\_FLASHOVER}      & Logical     & Section \ref{}                 &                             &    {\ct .FALSE.}    \\ \hline
{\ct RADIATIVE\_FRACTION}  & Real        & Section \ref{info:FIRE}                 &                             &     0.35        \\ \hline
{\ct SETPOINT}             & Real        & Section \ref{info:FIRE}                 & s $\mid$ $^\circ$C $\mid$ kW/m$^2$  &                 \\ \hline
{\ct SOOT\_YIELD}          & Real        & Section \ref{info:FIRE}                 &                             &                 \\ \hline
{\ct TRACE\_YIELD}         & Real        & Section \ref{info:FIRE}                 &                             &                 \\ \hline
\end{longtable}
\end{minipage}


\vspace{\baselineskip}
\noindent Example:
\begin{lstlisting}
&FIRE ID = 'Cushion', COMP_ID = 'Comp 1',
      LOCATION(1:3) = 2.5, 2.5, 0.,
      CARBON = 9, HYDROGEN = 6, OXYGEN = 2, NITROGEN = 2,
      IGNITION_CRITERION = 'TIME',
      HEAT_OF_COMBUSTION = 50000., RADIATIVE_FRACTION = 0.33,
      SOOT_YIELD = 0.227, CO_YIELD = 0.0845557,
      AREA = 0.1, HEIGHT = 0., HRR_RAMP_ID = 'Cushion Fire' /

&RAMP ID = 'Cushion Fire', TYPE = 'HEAT_RELEASE_RATE',
      T = 0., 60., 120., 180., 240., 300.,
      HRR = 0., 100000., 150000., 200000., 150000., 125000. /
\end{lstlisting}




\clearpage
\section{Devices, Namelist Group \texorpdfstring{{\tt DEVC}}{DEVC}}

\label{info:DEVC3}

\begin{minipage}{6.5in}
\begin{longtable}{@{\extracolsep{\fill}}|l|l|l|l|l|}
\caption[Device Parameters ({\ct DEVC} namelist group)]{For more information see Section~\ref{info:DEVC}.}
\label{tbl:DEVC} \\
\hline
\multicolumn{5}{|c|}{{\ct DEVC} (Device Parameters)} \\
\hline \hline
\endfirsthead
\caption[]{Continued} \\
\hline
\multicolumn{5}{|c|}{{\ct DEVC} (Device Parameters)} \\
\hline \hline
\endhead
\parbox{1.5in}{\bf Parameter}    & \parbox{1in}{\bf Type}  & \parbox{1in}{\bf Reference}  & \parbox{1in}{\bf Units}  & \parbox{1in}{\bf Default Value} \\ \hline
{\ct COMP\_ID}            & Character   & Section \ref{info:DEVC}     &                   &                 \\ \hline
{\ct TYPE}\footnote{{\ct PLATE}, {\ct CYLINDER}, {\ct SPRINKLER}, {\ct HEAT\_DETECTOR}, or {\ct SMOKE\_DETECTOR}}      
                          & Character   & Section \ref{info:DEVC}     &                   &                 \\ \hline
{\ct ID}      		      & Character   & Section \ref{info:DEVC}                 &                   &                 \\ \hline
{\ct INTERNAL\_LOCATION}  & Real        & Section \ref{info:DEVC}     &                   &       0.5       \\ \hline
{\ct LOCATION(1:3)}       & Real Array  & Section \ref{info:DEVC}                 & m                 &                 \\ \hline
{\ct MATL\_ID}            & Character   & Section \ref{info:DEVC}                 &                   &                 \\ \hline
{\ct NORMAL(1:3)}         & Real Array  & Section \ref{info:DEVC}                 &                   &                 \\ \hline
{\ct RTI}                 & Real        & Section \ref{info:DEVC2}                 & $\sqrt{\hbox{m}\cdot\hbox{s}}$       &                 \\ \hline
{\ct SETPOINT}            & Real        & Section \ref{info:DEVC}     & $^\circ$C or \%/m  &     23.93 \%/m      \\ \hline
{\ct SPRAY\_DENSITY}      & Real        & Section \ref{info:DEVC2}     & m/s               &                 \\ \hline
\end{longtable}
\end{minipage}

\graybox{{\ct PLATE} and {\ct CYLINDER} targets are defined by {\ct COMP\_ID}, {\ct DEVC\_TYPE}, {\ct ID}, {\ct LOCATION}, {\ct INTERNAL\_TEMPERATURE\_LOCATION}, and {\ct NORMAL}. \\

A sprinkler is defined by {\ct COMP\_ID}, {\ct DEVC\_TYPE}, {\ct ID}, {\ct LOCATION}, {\ct RTI}, {\ct SETPOINT}, and {\ct SPRAY\_DENSITY}.  \\

A smoke detector is defined by {\ct COMP\_ID}, {\ct DEVC\_TYPE}, {\ct ID}, {\ct LOCATION}, and {\ct SETPOINT}. \\

Input for {\ct SETPOINT} for smoke detectors is obscuration; for heat detectors or sprinklers, the input is temperature.}

\vspace{\baselineskip}
\noindent Examples:
\begin{lstlisting}
&DEVC ID = 'Targ 1', COMP_ID = 'Comp 1', TYPE = 'PLATE',
      LOCATION(1:3) = 2.2, 1.88, 2.34., NORMAL(1:3) = 0., 0., 1.,
      MATL_ID = 'CONCRETE', INTERNAL_LOCATION = 0.5 /

&DEVC ID = 'Sprinkler 1', COMP_ID = 'Comp 1',
      TYPE = 'SPRINKLER',
      LOCATION(1:3) = 3., 3., 2.97,
      SETPOINT = 73.8889, RTI = 100., SPRAY_DENSITY = 7.E-5 /
\end{lstlisting}



\clearpage
\section{Compartment Connections, Namelist Group \texorpdfstring{{\tt CONN}}{CONN}}

\begin{minipage}{6.5in}
\begin{longtable}{@{\extracolsep{\fill}}|l|l|l|l|l|}
\caption[Connection Parameters ({\ct CONN} namelist group)]{For more information see Section~\ref{info:CONN}.}
\label{tbl:CONN} \\
\hline
\multicolumn{5}{|c|}{{\ct CONN} (Connection Parameters)} \\
\hline \hline
\endfirsthead
\caption[]{Continued} \\
\hline
\multicolumn{5}{|c|}{{\ct CONN} (Connection Parameters)} \\
\hline \hline
\endhead
\parbox{1.5in}{\bf Parameter}    & \parbox{1in}{\bf Type}  & \parbox{1in}{\bf Reference}  & \parbox{1in}{\bf Units}  & \parbox{1in}{\bf Default Value} \\ \hline
{\ct COMP\_ID}              & Character & Section \ref{info:CONN}                 &           &  		\\ \hline
{\ct COMP\_IDS}              & Character & Section \ref{info:CONN}                 &           &  		\\ \hline
{\ct F}             & Real      & Section \ref{info:CONN}     &           &                 \\ \hline
{\ct TYPE}\footnote{{\ct 'CEILING'}, {\ct 'FLOOR'}, or {\ct 'WALL'}}        	    & Character     & Section \ref{info:CONN}                 &           &                 \\ \hline
\end{longtable}
\end{minipage}

\vspace{\baselineskip}
\noindent Example:
\begin{lstlisting}
&CONN TYPE = 'FLOOR', COMP_ID = 'Comp 3', COMP_IDS = 'Comp 2' /
\end{lstlisting}




\clearpage
\section{Visualization, Namelist Groups \texorpdfstring{{\tt ISOF}}{ISOF}, \texorpdfstring{{\tt SLCF}}{SLCF}}

\subsection{\texorpdfstring{{\tt ISOF}}{ISOF} (Isosurface Parameters)}

\begin{longtable}{@{\extracolsep{\fill}}|l|l|l|l|l|}
\caption[Isosurface parameters ({\ct ISOF} namelist group)]{For more information see Section~\ref{info:ISOF}.}
\label{tbl:ISOF} \\
\hline
\multicolumn{5}{|c|}{{\ct ISOF} (Isosurface Parameters)} \\
\hline \hline
\endfirsthead
\caption[]{Continued} \\
\hline
\multicolumn{5}{|c|}{{\ct ISOF} (Isosurface Parameters)} \\
\hline \hline
\endhead
\parbox{1.5in}{\bf Parameter}    & \parbox{1in}{\bf Type}  & \parbox{1in}{\bf Reference}  & \parbox{1in}{\bf Units}  & \parbox{1in}{\bf Default Value} \\ \hline
{\ct COMP\_ID}        & Character   & Section \ref{info:ISOF}                 &           &                 \\ \hline
{\ct VALUE}             & Real        & Section \ref{info:ISOF}                 & $^\circ$C &                 \\ \hline
\end{longtable}

\noindent Example:
\begin{lstlisting}
&ISOF COMP_ID = 'COMP_ID', VALUE = 100. /
\end{lstlisting}




\subsection{\texorpdfstring{{\tt SLCF}}{SLCF} (Slice File Parameters)}

\begin{longtable}{@{\extracolsep{\fill}}|l|l|l|l|l|}
\caption[Slice File parameters ({\ct SLCF} namelist group)]{For more information see Section~\ref{info:SLCF}.}
\label{tbl:SLCF} \\
\hline
\multicolumn{5}{|c|}{{\ct SLCF} (Slice File Parameters)} \\
\hline \hline
\endfirsthead
\caption[]{Continued} \\
\hline
\multicolumn{5}{|c|}{{\ct SLCF} (Slice File Parameters)} \\
\hline \hline
\endhead
\parbox{1.5in}{\bf Parameter}    & \parbox{1in}{\bf Type}  & \parbox{1in}{\bf Reference}  & \parbox{1in}{\bf Units}  & \parbox{1in}{\bf Default Value} \\ \hline
{\ct COMP\_ID}        & Character   & Section \ref{info:SLCF}                 &           &                 \\ \hline
{\ct DOMAIN}            & Character   & Section \ref{info:SLCF}                 &           &                 \\ \hline
{\ct PLANE}             & Character   & Section \ref{info:SLCF}                 &           &                 \\ \hline
{\ct POSITION}          & Real        & Section \ref{info:SLCF}                 &           &                 \\ \hline
\end{longtable}

\noindent Example:
\begin{lstlisting}
&SLCF  DOMAIN = '2-D', PLANE = 'X', POSITION = 2.5 /
\end{lstlisting}




\clearpage
\section{Ramp Functions, Namelist Group \texorpdfstring{{\tt RAMP}}{RAMP}}
\label{info:RAMP}
\label{info:RAMP2}

\vspace{\baselineskip}
\begin{longtable}{@{\extracolsep{\fill}}|l|l|l|l|l|}
\caption[Ramp Function Parameters ({\ct RAMP} namelist group)]{For more information see Section~\ref{info:RAMP}.}
\label{tbl:RAMP} \\
\hline
\multicolumn{5}{|c|}{{\ct RAMP} (Ramp Function Parameters)} \\
\hline \hline
\endfirsthead
\caption[]{Continued} \\
\hline
\multicolumn{5}{|c|}{{\ct RAMP} (Ramp Function Parameters)} \\
\hline \hline
\endhead
\parbox{1.5in}{\bf Parameter}    & \parbox{1in}{\bf Type}  & \parbox{1in}{\bf Reference}  & \parbox{1in}{\bf Units}  & \parbox{1in}{\bf Default Value} \\ \hline
{\ct F}               & Real        & Section \ref{info:RAMP2}                 &           &                 \\ \hline
{\ct A}               & Real        & Section \ref{info:RAMP2}                 &  m$^2$         &                 \\ \hline
{\ct H}               & Real        & Section \ref{info:RAMP2}                 &  m         &                 \\ \hline
{\ct HRR}             & Real        & Section \ref{info:RAMP2}                 & kW         &                 \\ \hline
{\ct ID}      	      & Character   & Section \ref{info:RAMP2}                 &           &                 \\ \hline
{\ct T}               & Real        & Section \ref{info:RAMP2}                 &  s        &                 \\ \hline
{\ct TYPE}                & Real        & Section \ref{info:RAMP2}                 &          &                 \\ \hline
\end{longtable}

\graybox{
Input for {\ct TYPE} must be {\ct AREA}, {\ct EFFICIENCY}, {\ct FRACTION}, or {\ct HEAT\_RELEASE\_RATE}.  \\

Vent opening is specified as a function of time with {\ct T} and {\ct F}. {\ct F} is the fraction of the vent width (for wall vents) or the vent area (for ceiling, floor, and mechanical vents). Refer to Section~\ref{info:VENT2}. \\

Filter efficiency is specified as a function of time with {\ct T} and {\ct F}. {\ct F} is the fraction of soot and trace species removed in the associated mechanical vent. Refer to Section~\ref{info:VENT4}. \\

Compartment area in XY-plane is specified as  function of compartment height with {\ct H} and {\ct A}. Refer to Section~\ref{info:COMP4}. \\

Heat release rate of a fire is specified as  function of time using {\ct T} abd {\ct HRR}. Refer to Section~\ref{info:FIRE2}.
}





\chapter{Running CFAST from a Command Prompt}

The model CFAST can also be run from a Windows command prompt.  CFAST can be run from any folder, and refer to a data file in any other folder. The fires and thermophysical properties have to be in either the data folder, or the executable folder. The data folder is checked first and then the executable folder.

\begin{lstlisting}
[drive1:\][folder1\]cfast [drive2:\][folder2\]project
\end{lstlisting}

The project name will have extensions appended as needed (see below). For example, to run a test case when the CFAST executable is located in c:$\backslash$firemodels$\backslash$cfast7 and the input data file is located in c:$\backslash$data, the following command could be used:

\begin{lstlisting}
c:\firemodels\cfast7\cfast c:\data\testfire0   <<< note that the extension is optional.
\end{lstlisting}

Command line options

\begin{itemize}
\item -k - no interactive keyboard access
\item -i - initialization only
\item -c - compact output
\item -f - full output (c and f are exclusive). Note the interaction of the f and c option. The default for the console output is /c. The default for the file output is /f. This default action can be overwritten by explicitly including the /f or /c option.
\item -n - net heat flux option
\item -v - validation output (outputs a modified set of spreadsheet files with different column headers designed to facilitate automated analysis of the output)
\end{itemize}


\label{last_page}


