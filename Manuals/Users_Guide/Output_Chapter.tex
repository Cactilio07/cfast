\chapter{Output from CFAST}
\label{Output_Chapter}

The output of CFAST includes the temperatures of the upper and lower gas layers within each compartment, the ceiling/wall/floor temperatures within each compartment, the visible smoke and gas species concentrations within each layer, target temperatures and sprinkler activation time.  The amount of information can be very large, especially for complex geometries and long simulations.

\section{Compact Output}

The default output to the console is called the compact form, and shows the basic information about a scenario, including layer temperatures and the size of fires. Default text output provides a simple overview for the user to make sure the case runs as expected.
\begin{lstlisting}[basicstyle=\scriptsize]
 Time =   1800.0 seconds.

 Compartment   Upper   Lower   Inter.  Pyrol     Fire      Pressure  Ambient
               Temp.   Temp.   Height  Rate      Size                Target
               (C)     (C)     (m)     (kg/s)    (W)       (Pa)      (W/m^2)
 -----------------------------------------------------------------------------
    1          113.4    33.3    1.4     1.393E-02 3.000E+05-0.790      523.
  Outside                                         0.00
\end{lstlisting}
The first column contains the compartment number.  On each row with its compartment number from left to right is the upper layer temperature, lower layer temperature, the height of the interface between the two layers, the total pyrolysis rate, and finally the total fire size.  The only value given for the outside is the total heat release rate of fires venting to the outside.

\section{Detailed Outputs}

The following sections describe each of the outputs from the model.  Each section refers to a specific part of the print out and appears in the order the output appears. A description of each option follows.

\subsection{Output for Initialization}

This option prints the initial conditions to the output before the actual run starts.  This merely mimics the inputs specified by the user in the input data file  The initial conditions break down into seven sections.  Each is described below with the section name. The following explanation uses the output from the case STANDARD.IN. STANDARD.IN is included in the distribution. Please note, there are no mechanical ventilation, horizontal vents or detectors in this example, so the section discussing these phenomena are from additional data files.

\subsubsection{Overview}

The overview gives a general description of the case.  The output is fairly self explanatory. ``Doors, ...'' is the total number of horizontal natural flow vent connections in all compartments of the simulation.  ``Ceil. Vents, ...'' gives the total number of vertical natural flow vent connections in all compartments of the simulation.  The last header on the line ``MV Connections'' has the total number mechanical flow connections to all compartments in the simulation. Times in these outputs come from the TIMES input. All times are in s.
\begin{lstlisting}[basicstyle=\tiny]
CFAST

Version          : CFAST 7.1.0
Revision         : Gitv7.0.1-132-gf03a791
Revision Date    : Tue Feb 2 12:54:53 2016 -0500
Compilation Date : Tue 02/02/2016  12:55 PM

Data file: C:\Users\rpeacoc\Documents\Visual Studio 2013\Projects\cfast\Bin\Data\standard.in
Title: Users Guide Example Case


OVERVIEW


Compartments    Doors, ...    Ceil. Vents, ...    MV Connects
-------------------------------------------------------------
   3               3             1                    4

Simulation     Output         Smokeview      Spreadsheet
Time           Interval       Interval       Interval
   (s)            (s)            (s)            (s)
--------------------------------------------------------
  3600          50             10             10
\end{lstlisting}

\subsubsection{Ambient Conditions}

This section, like the overview section, needs little elaboration.  It gives the starting atmospheric conditions for the simulation both for outside and inside the structure. Data for these outputs come from the TAMB and EAMB inputs. Temperatures are in K, pressure in Pa, elevations in m, and wind speed in m/s. Wind Power is the dimensionless power law coefficient from the WIND input.
\begin{lstlisting}[basicstyle=\tiny]
AMBIENT CONDITIONS

Interior       Interior       Exterior       Exterior
Temperature    Pressure       Temperature    Pressure
  (C)            (Pa)           (C)            (Pa)
-----------------------------------------------------
    20.          101325.          20.          101325.
\end{lstlisting}

\subsubsection{Compartments}

The compartments section gives a summary of the geometry for the simulation.  A simple table summarizes the geometry with compartments running down the page in numerical order.  The various dimensions for each compartment are on the row with its compartment number.  Two columns need explanation.  The second to last column ``Ceiling Height'' gives the height of the ceiling relative to the station height in the Ambient Conditions section.  Similarly the ``Floor Height'' refers to the height of the floor above the station height.

\begin{lstlisting}[basicstyle=\tiny]
COMPARTMENTS

Compartment  Name                Width        Depth        Height       Floor        Ceiling
                                                                        Height       Height
                                 (m)          (m)          (m)          (m)          (m)
------------------------------------------------------------------------------------------------
    1               Comp 1        5.00         5.00         3.00         0.00         3.00
    2               Comp 2        5.00         5.00         3.00         0.00         3.00
    3               Comp 3        5.00         5.00         3.00         3.00         6.00
\end{lstlisting}


\subsubsection{Horizontal Natural Ventilation}

This is the first table in the vent connections section.  Each row in the table characterizes one vent.  The first two columns contain the two compartments connected by the vent.  Each vent is ordered first by the lower number of the two compartments and then the numeric order of the second compartment.  The third column gives the vent number.  Column four is the width of the vent.  The next two columns report the sill and soffit height for the vent relative to the floor of the first compartment.  The seventh and eighth columns have a second listing of the sill and soffit height, this time relative to the station height.

\begin{lstlisting}[basicstyle=\tiny]
VENT CONNECTIONS

Horizontal Natural Flow Connections (Doors, Windows, ...)

From           To             Vent       Width       Sill        Soffit      Abs.        Abs.
Compartment    Compartment    Number                 Height      Height      Sill        Soffit
                                         (m)         (m)         (m)         (m)         (m)
----------------------------------------------------------------------------------------------------
Comp 1         Comp 2          1          1.00        0.00        2.00        0.00        2.00
Comp 1         Outside         1          1.00        0.00        2.00        0.00        2.00
Comp 3         Outside         1          1.00        1.00        2.00        4.00        5.00
\end{lstlisting}
From compartment, to compartment, vent number, width, sill height, and soffit height all come directly from the HVENT specifications in the input data file. Absolute sill height is the station elevation + compartment floor height + sill height. Absolute soffit height is the station elevation + compartment floor height + soffit height.

\subsubsection{Vertical Natural Ventilation}

The first column is the upper compartment.  The upper compartment is the compartment where the vent opens into the floor.  The second column is the lower compartment where the vent is in the ceiling.  The third column describes the shape of the vent, which can be either round or square.  The fourth column gives the area of the vent.  The last two columns are the height of the vent, relative to the floor of the lower room and relative to the station height respectively.
\begin{lstlisting}[basicstyle=\tiny]
Vertical Natural Flow Connections (Ceiling, ...)

Top            Bottom         Shape     Area      Relative  Absolute
Compartment    Compartment                        Height    Height
                                        (m^2)     (m)       (m)
------------------------------------------------------------------------
    3              2          Round      0.25      3.00      3.00
\end{lstlisting}
Top compartment, bottom compartment, shape, and area come from the VVENT specifications in the input data file. Relative height is the height of the vent above the floor of the bottom compartment and absolute height is the height of the vent above the station elevation.

\subsubsection{Mechanical Flow Connections}

This section lists all connections to compartments and fans that connect between compartments. The table lists, in order, the number of the system the fan is a part, the ``from'' node and its height, the ``to'' node and its height.  A fan actually draws air from the first or ``from'' node and pushes it to the second or ``to'' node. The sixth column is the cross-sectional area of the duct connection to the chosen compartment. The seventh column is the fan number as defined in CEdit.  The next two columns are the minimum and maximum pressures at which the fan curve is defined.  The rest of the row is made up of the one to five fan curve coefficients in the input file.

\begin{lstlisting}[basicstyle=\tiny]
FANS

System    From           From      To             To        Area      Fan         Minimum       Maximum    Flowrate
                         Elev.                    Elev.               Number
                         (m)                      (m)       (m^2)                 (Pa)          (Pa)       (m^3/s)
-------------------------------------------------------------------------------------------------------------------
   1      Outside        2.75      Node  1        2.75      0.25
          Node  1        2.75      Node  2        2.75                  1         2.00E+02      3.00E+02     2.00E-02
          Node  2        2.75      Comp  1        2.75      0.25
   2      Comp  2        2.75      Node  3        2.75      0.25
          Node  3        2.75      Node  4        2.75                  2         2.00E+02      3.00E+02     2.00E-02
          Node  4        2.75      Outside        2.75      0.25
\end{lstlisting}

\subsubsection{Thermal Properties}

The thermal properties section is broken into two parts.  The first part is a table that lists the material for each surface of each compartment.  The compartments appear as rows down the page in numerical order.  From left to right next to the compartment number comes the material for the ceiling, wall and floor.  The second part lists the properties of each material used in the simulation. For each listing of a material, the name is followed by the conductivity, specific heat, density, thickness and emissivity. In addition to materials for compartment surfaces, any thermal properties specified for targets are also listed (this may include thermal properties for gaseous materials specified as fire sources in a simulation.

\begin{lstlisting}[basicstyle=\tiny]
THERMAL PROPERTIES

Compartment    Ceiling      Wall         Floor
-----------------------------------------------
Comp 1       CONCRETE     CONCRETE     CONCRETE
Comp 2       CONCRETE     CONCRETE     CONCRETE
Comp 3       CONCRETE     CONCRETE     CONCRETE


Name    Conductivity      Specific Heat     Density        Thickness     Emissivity
-----------------------------------------------------------------------------------
CONCRETE     1.75           1.000E+03       2.200E+03       0.150           0.940
DEFAULT     0.120            900.            800.           1.200E-02       0.900
\end{lstlisting}
Material choices of the ceiling, walls, and floors come from the CEILI, WALLS, and FLOOR specifications in the input data file. Units for thermal properties are standard S.I. units.  For thermal conductivity, W/m K; for specific heat, J/kg K; for density, kg/m$^3$; for thickness, m; emissivity is dimensionless.


\subsubsection{Fires}

The fire section lists all the information about the main fire and any object fires that might exist.  All the information for each fire is listed separately.  If there is a main fire, it comes first.  Each fire listing has the same form.  First is the name of the fire followed by a list of general information.  Listed left to right is the compartment the fire is in, the type of fire, the initial x (width), y (depth), z (height) position of the fire, the relative humidity, the lower oxygen limit, and finally the radiative fraction for the fire.

A table of time history curves for the fire follows.  The table contains all the time history curves for the fire.  Each row on the table is a specific time given in the left most column.  The rest of the columns give the values at that particular time.  The column headers indicate each input quantity and correspond to specific keywords in the fire definition. The headings are defined as follows: `Mdot' is pyrolysis rate; `Hcomb' is the heat of combustion; `Qdot' is the heat release rate; `Zoffset' is height of the fire above the base z-position; `Soot' is the fraction of the fuel mass converted to soot during combustion; `CO' is the fraction of the fuel mass converted to carbon monoxide during combustion; `HCN' is the fraction of the fuel mass converted to hydrogen cyanide during combustion; `HCl' is the fraction of the fuel mass converted to hydrogen chloride during combustion; `CT' is the concentration-time product; and `TS' is the fraction of fuel mass converted to trace species during combustion.

\begin{lstlisting}[basicstyle=\tiny]
FIRES


Name: bunsen   Referenced as object #  1 Normal fire

Compartment    Fire Type       Position (x,y,z)     Relative    Lower O2    Radiative
                                                    Humidity    Limit       Fraction
-------------------------------------------------------------------------------------
Comp 1         Constrained     2.50   2.50   0.00   50.0        15.00        0.33


Chemical formula of the fuel
Carbon     Hydrogen  Oxygen    Nitrogen  Chlorine
--------------------------------------------------
  1.000     4.000     0.000     0.000     0.000


  Time      Mdot      Hcomb     Qdot      Zoffset   Soot      CO        HCN       HCl       TS
  (s)       (kg/s)    (J/kg)    (W)       (m)       (kg/kg)   (kg/kg)   (kg/kg)   (kg/kg)   (kg/kg)
------------------------------------------------------------------------------------------------------
     0.      0.0      5.00E+07   0.0       0.0       0.0      1.05E-03   0.0       0.0       0.0
    60.     2.00E-03  5.00E+07  1.00E+05   0.0       0.0      1.05E-03   0.0       0.0       0.0
   120.     3.00E-03  5.00E+07  1.50E+05   0.0       0.0      1.05E-03   0.0       0.0       0.0
   180.     4.00E-03  5.00E+07  2.00E+05   0.0       0.0      1.05E-03   0.0       0.0       0.0
   240.     3.00E-03  5.00E+07  1.50E+05   0.0       0.0      1.05E-03   0.0       0.0       0.0
   300.     2.50E-03  5.00E+07  1.25E+05   0.0       0.0      1.05E-03   0.0       0.0       0.0
   360.     2.00E-03  5.00E+07  1.00E+05   0.0       0.0      1.05E-03   0.0       0.0       0.0
   420.     1.80E-03  5.00E+07  9.00E+04   0.0       0.0      1.05E-03   0.0       0.0       0.0
   480.     1.60E-03  5.00E+07  8.00E+04   0.0       0.0      1.05E-03   0.0       0.0       0.0
   540.     1.50E-03  5.00E+07  7.50E+04   0.0       0.0      1.05E-03   0.0       0.0       0.0
  1800.     1.50E-03  5.00E+07  7.50E+04   0.0       0.0      1.05E-03   0.0       0.0       0.0


Name: Wood_Wall   Referenced as object #  2 Wall   fire

Compartment    Fire Type       Position (x,y,z)     Relative    Lower O2    Radiative
                                                    Humidity    Limit       Fraction
-------------------------------------------------------------------------------------
Comp 2         Constrained     2.50   5.00   0.00   50.0        15.00        0.33


Chemical formula of the fuel
Carbon     Hydrogen  Oxygen    Nitrogen  Chlorine
--------------------------------------------------
  6.000    10.000     5.000     0.000     0.000


  Time      Mdot      Hcomb     Qdot      Zoffset   Soot      CO        HCN       HCl       TS
  (s)       (kg/s)    (J/kg)    (W)       (m)       (kg/kg)   (kg/kg)   (kg/kg)   (kg/kg)   (kg/kg)
------------------------------------------------------------------------------------------------------
     0.      0.0      1.81E+07   0.0       0.0      2.00E-02  2.00E-02   0.0       0.0       0.0
  8000.     5.52E-02  1.81E+07  1.00E+06   3.0      2.00E-02  2.00E-02   0.0       0.0       0.0
\end{lstlisting}
All of the inputs for the main fire come from the fire specifications in the input data file. Data for the object fire comes from the object data file included with the CFAST software. Units for most values are included in the output.  Fire position is in m, relative humidity is in \%, lower oxygen limit is in volume percent, and pyrolysis temperature is in K.


\subsubsection{Targets}

The entry for targets shows the orientation of additional targets specified in the data file. Targets explicitly specified in the data file are listed first in the order they are included in the data file.  Each target is numbered based on the order of the target specifications in the input data file.  The compartment number, position of the target within the compartment, direction of the front face of the target object expressed as a normal unit vector to the surface, and object material.

\begin{lstlisting}[basicstyle=\tiny]
TARGETS

Target                      Compartment    Position (x, y, z)         Direction (x, y, z)      Material
------------------------------------------------------------------------------------------------------
    1   Targ 1                Comp 1       2.20     1.88     2.34     0.00     0.00     1.00   CONCRETE
\end{lstlisting}
All of the inputs for targets come from the TARGE command in the input data file. Direction is specified as a unit vector as described in the section on target input. Units for position and direction are all in m.

\subsubsection{Detectors and Sprinklers}

The entry for each detector or sprinkler shows the compartment and position of the device and its activation characteristics. For smoke detectors, activation is based on the smoke obscuration as the position of the detector; for heat detectors and sprinkler, the temperature of the detector.


\begin{lstlisting}[basicstyle=\tiny]
DETECTORS/ALARMS/SPRINKLERS
Target  Compartment        Type           Position (x, y, z)            Activation
                                                                        Obscuration    Temperature   RTI           Spray Density
                                         (m)      (m)      (m)          (%/m)         (C)           (m s)^1/2     (m/s)
--------------------------------------------------------------------------------------------------------------------------------
  1     Comp 1             HEAT         2.50     2.50     2.97                         73.89         100.00
\end{lstlisting}
All of the inputs for detectors and sprinklers come from the DETEC command in the input data file. Units for position are all in m.

\subsection{Output for Main Variables}

The normal print out is the first information printed at each time interval.  This information includes the layer temperatures, interface height, volume of the upper layer, layer absorption coefficients, and compartment pressure (relative to ambient).

\begin{lstlisting}[basicstyle=\tiny]
Time =   3600.0 seconds.

Compartment    Upper     Lower      Inter.      Upper           Upper      Lower       Pressure
               Temp.     Temp       Height      Vol             Absor      Absorb
               (C)       (C)        (m)         (m^3)           (m^-1)     (m^-1)      (Pa)
----------------------------------------------------------------------------------------------------
Comp 1         78.70      23.01      1.404       40.    ( 53%)  0.235      0.100       -0.430
Comp 2         176.5      44.97      1.495       38.    ( 50%)  0.240      9.153E-02    -1.49
Comp 3         104.3      27.58      1.248       44.    ( 58%)  0.241      9.556E-02   -0.336
\end{lstlisting}
The second table of the normal print out has information about the fires.  In essence it is two tables joined.  The first part lists information by fire.  It starts with the main fire, if there is one, and then the object fires down the page.  The fires are listed in the second column followed by the plume flow rate, the pyrolysis rate, the fire size, and flame height.  The next three columns are then skipped.  The next column with information is the amount of heat given off by each fire convectively, followed by the amount of heat given off radiantly. The last two columns give the total mass pyrolyzed and the amount of trace species produced.  The second part starts after all the fires have been individually listed.  It gives the totals for all fires in each compartment.  The first column has the compartment number.  The compartments start at one and are listed down the page in order.  The third to fifth columns are the same as the first part except the values are totals for the compartment and not just for one fire.  The sixth column has the total heat release rate that occurs in the upper layer.  The next column has the same total in the lower layer.  The eighth column has the total size of vent fires in the compartment.

\begin{lstlisting}[basicstyle=\tiny]
FIRES

Compartment    Fire    Plume     Pyrol     Fire      Flame   Fire in  Fire in   Vent   Convec.  Radiat.   Pyrolysate Trace
                       Flow      Rate      Size      Height  Upper    Lower     Fire
                       (kg/s)    (kg/s)    (W)       (m)     (W)      (W)       (W)    (W)      (W)       (kg)       (kg)
 -------------------------------------------------------------------------------------------------------------------------
                bunsen 8.571E-05 1.440E-07  7.20      0.00                              4.82      2.37     1.02       0.00
              Wood_Wal 0.974     2.486E-02 4.500E+05 0.382                             3.015E+05 1.485E+05 44.8       0.00

Comp 1                 8.571E-05 1.440E-07  7.20              0.00     7.20      0.00
Comp 2                 0.974     2.486E-02 4.500E+05          0.00     4.500E+05 0.00
\end{lstlisting}
Flame height is calculated from the work of Heskestad~\cite{Heskestad:2002}. The average flame height is defined as the distance from the fuel source to the top of the visible flame where the intermittency is 0.5.  A flame intermittency of 0.5 means that the visible flame is above the mean 50~\% of the time and below the mean 50~\% of the time.


\subsection{Output for Wall Surfaces, Targets, and Detectors/Sprinklers}

The printed output provides two tables displaying information about wall surface or target temperatures and fluxes, and heat detectors or sprinklers. The left most column specifies the compartment number; followed by four columns providing the temperatures of the bounding surfaces of the compartment in contact with the ceiling, upper wall surface (in contact with the upper layer gases), lower wall surface (in contact with the lower layer gases), and floor, in that order. Next comes information about targets in the compartment, with each target listed on a separate line.  Information in the columns includes the surface temperature of the target, net heat flux to the target, and the percentage of that net flux that is due to radiation from the fire, radiation from compartment surfaces, radiation from the gas layers, and convection from the gas surrounding the target.  CFAST includes one target in the center of the floor for all compartments. Information on additional targets specified by the user in the input data file are also included, in the order specified in the input file.

For smoke detectors, heat detectors, and sprinklers, the temperature of the device, its current state (activated or not), and the nearby gas temperature and velocity are included.

\begin{lstlisting}[basicstyle=\tiny]
SURFACES AND TARGETS

Compartment  Ceiling  Up wall  Low wall  Floor   Target  Gas     Surface  Internal Flux To  Fire    Surface  Gas
             Temp.    Temp.    Temp.     Temp.           Temp.   Temp.    Temp.    Target   Rad.    Rad.     Rad.     Convect.
             (C)      (C)      (C)       (C)             (C)     (C)      (C)      (W/m^2)  (W/m^2) (W/m^2)  (W/m^2)  (W/m^2)
------------------------------------------------------------------------------------------------------------------------------
Comp 1        31.8     30.8     22.5     23.4
                                                 Targ 1  78.7    30.4     24.5      503.     0.00    341.     209.     347.
Comp 2        133.     61.8     86.2     51.3
Comp 3        34.2     34.2     22.7     23.1
\end{lstlisting}

\begin{lstlisting}[basicstyle=\tiny]
DETECTORS/ALARMS/SPRINKLERS
                                      Sensor         Smoke

Number  Compartment        Type       Temp (C)       Temp (C)      Vel (m/s)     Obs (1/m)          Activated
--------------------------------------------------------------------------------------------------------------
  1     Comp 1             HEAT       7.757E+01      7.870E+01                                      YES
\end{lstlisting}
In all cases, the flux to/from a target is net radiation or net convection. That is, it is the incoming minus the outgoing. So while a target or object is heating, the flux will be positive, and once it starts to cool, the flux will be negative. Values for radiation from fires (fire rad.), radiation from surfaces (surface rad.), radiation from the gas layers (gas rad.), and convection from surfaces (convect) are expressed as the net flux to target (flux to target). Positive values indicate heat gains by the target and negative values indicate heat losses.


\subsection{Output for Gas Species}

The output has two tables displaying information about the amounts of species in each layer. The species information follows the normal print out.  The first table gives species volume fractions for the upper layers of all the compartments and the second reports the same for the lower layers of all the compartments.  Again the compartments are listed down the page and the information across the page.  The species are each given in one of several different terms.  Below each header are the units for the given value.  Most of the headers are simply the chemical formula for the species being tracked.  However, several are not obvious.  ``TUHC'' is the total unburned hydrocarbons or the pyrolyzed fuel that hasn't burned yet.  ``OD'' is the optical density, which is a measure of the amount of smoke. ``TS'' is trace species.

\begin{lstlisting}[basicstyle=\tiny]
UPPER LAYER SPECIES

Compartment    N2         O2         CO2        CO         HCN        HCL        TUHC       H2O        OD          TS
               (%)        (%)        (%)        (%)        (%)        (%)        (%)        (%)        (1/m)       kg
----------------------------------------------------------------------------------------------------------------------
Comp 1         76.7       17.8       2.24      4.617E-02   0.00       0.00       0.00       3.13       1.69       0.00
Comp 2         76.4       17.4       2.58      5.320E-02   0.00       0.00       0.00       3.43       1.53       0.00
Comp 3         76.5       17.4       2.52      5.205E-02   0.00       0.00       0.00       3.38       1.78       0.00


LOWER LAYER SPECIES

Compartment    N2         O2         CO2        CO         HCN        HCL        TUHC       H2O        OD          TS
               (%)        (%)        (%)        (%)        (%)        (%)        (%)        (%)        (1/m)       kg
----------------------------------------------------------------------------------------------------------------------
Comp 1         78.3       20.4      7.400E-02  1.526E-03   0.00       0.00       0.00       1.23      6.656E-02   0.00
Comp 2         78.3       20.4      8.491E-02  1.751E-03   0.00       0.00       0.00       1.24      7.108E-02   0.00
Comp 3         78.4       20.5      4.208E-10  8.678E-12   0.00       0.00       0.00       1.16      3.726E-10   0.00
\end{lstlisting}
The report by species for nitrogen, oxygen, hydrogen chloride and the total unburned hydrocarbons (fuel vapor in the layer) are percent by volume. Carbon dioxide, carbon monoxide and hydrogen cyanide are in parts per million, which is proportional to the volume fraction.  Optical depth per meter is a measure of the visibility in the smoke. This is covered in detail in the comment on visibility in the section on fires and species specification. The concentration-time (CT) calculation is an integration of the species input for type CT (See section \ref{sec:fire_inputs} for the input of CT) and is intended to represent a relative dose of toxic gas species. Trace species (TS) is the total kilograms of the trace species that is present in the compartment. It is an absolute measure and not percent or density.


\subsection{Output for Vent Flows}

Information about vent flow is obtained by using this option.  It includes a section detailing mass flow through horizontal, vertical, and mechanical vents. There are two forms for the vent flow. The first is flow through the vents as mass per second. The alternative, obtained with the total mass flow output option on the `Run!,' `Output Options' menu, gives the total mass which has flowed through the vent(s) and the relative mass of trace species divided by the total mass of trace species produced up to the current time.

The section for vent flow is titled ``FLOW THROUGH VENTS (kg/s).''  Because flow is always given in positive values, each vent is listed twice, once for flow going from compartment A to compartment B (labelled as ``Flow relative to `from''') and a second time for flow from B to A (labelled as ``Flow relative to `To'''.  As the example below shows, the first column specifies the vent, including the type of vent (an ``H'' in this column stands for horizontal flow, such as through a doorway or window; a ``V'' here would mean vertical flow, such as through an opening in the ceiling, and an ``M'' stands for a mechanical ventilation connection) and the compartment from which the flow comes. The second column lists the name of the compartment. Up to six additional columns detail the flow at this vent. Flow into and out of the compartment through the vent in the upper and lower layers are included.

\begin{lstlisting}[basicstyle=\tiny]
FLOW THROUGH VENTS (kg/s)

                            Flow relative to 'From'                        Flow Relative to 'To'
                            Upper Layer             Lower Layer            Upper Layer            Lower Layer
Vent   From/Bottom  To/Top  Inflow      Outflow     Inflow     Outflow     Inflow      Outflow    Inflow      Outflow
------------------------------------------------------------------------------------------------------------------------
H  1   Comp 1       Comp 2  0.2996E+00  0.5729E-02  0.1406E-01 0.9832E+00  0.1469E-02  0.2904E+00 0.9874E+00  0.2322E-01
H  2   Comp 1       Outside             0.2891E+00  0.9410E+00             0.2891E+00                         0.9410E+00
H  3   Comp 3       Outside             0.6894E+00  0.1667E-01 0.1752E-01  0.7069E+00                         0.1667E-01
V  1   Comp 2       Comp 3              0.6855E+00             0.1217E-02  0.6868E+00
M  1   Outside      Comp 1                                     0.2391E-01  0.2391E-01
M  3   Comp 2       Outside             0.1559E-01                                                0.1559E-01


TOTAL MASS FLOW THROUGH MECHANICAL VENTS (kg)

To             Through              Upper Layer               Lower Layer              Trace Species
Compartment    Vent             Inflow       Outflow      Inflow       Outflow       Vented    Filtered
--------------------------------------------------------------------------------------------------------

Comp 1         M Node  2       0.8585E+02                0.2200E+00

Comp 2         M Node  3                    0.6754E+02                0.4443E+00
Outside        M Node  1                                              0.8607E+02
               M Node  4                                 0.6798E+02
\end{lstlisting}

An alternative printout is provided that shows total (mass) flow through vents. At present this is confined to mechanical ventilation (It applies only to vents which can be filtered, in this case mechanical ventilation). The last column is obtained by summing the outflow/inflow for each vent and then dividing that sum by the total trace species produced by all fires. For details on this value, see the section on output listing for fires.


\section{Spreadsheet Output}

CFAST can generate a number of output files in a plain text spreadsheet format.  These files capture a snap shot of the modeling data at an instant of time. This instance is determined by the fourth entry on the TIMES line of the data file. \emph{However}, there are events which can occur in between these reporting periods. Examples are the ignition of objects and the activation of detectors or sprinklers. These are \emph{not} reported in these output files.

\subsection{Primary Output Variables (project\_n.csv)}

There are two sets of information in this file. The first includes compartment information such as layer temperature. This is output by compartment and there are eight entries for each compartment plus a column that indicates the current simulation time:

\begin{description}
\item[Time] (s)
\item[Upper Layer Temperature] (\degc)
\item[Lower Layer Temperature] (\degc)
\item[Layer Height]  (m)
\item[ Upper Layer Volume] (m$^3$): total volume of the upper layer. This is just the floor area times the difference between the ceiling height and the layer height.
\item[ Pressure] (Pa): pressure at compartment floor relative to the outside at the absolute height of the floor.
\item[Ambient Temp Target Flux] (W/m$^2$): net heat flux to the center of the floor assuming the floor is at ambient temperature.  This is largely useful to estimate the tenability of the compartment.
\item[Floor Temp Target Flux] (W/m$^2$): net heat flux to the center of the floor.
\item[HRR Door Jet Fires] (W): total heat release rate of all door jet fires \emph{adding} heat to this compartment.
\end{description}
The second set of information is for fires. There are seven entries per fire.  This information is displayed for each fire :
\begin{description}
\item[Plume Entrainment Rate] (kg/s): current mass entrained from the lower layer into the plume for this fire.
\item[Pyrolysis Rate] (kg/s): current rate of mass loss for this fire.
\item[HRR] (W): current total heat release rate for this fire. This is just the sum of the heat release rate for the lower layer and upper layer for this fire.
\item[HRR Lower] (W): current heat release rate for burning in the lower layer for this fire.
\item[HRR Upper] (W):  current heat release rate for burning in the upper layer for this fire.
\item[Flame Height] (m): current calculated flame height for this fire.
\item[Convective HRR] (W): current rate of heat release by convection for this fire.  The remainder is released by radiation to the surroundings.
\item[Total Pyrolysate Released] (kg): total mass released by the fire up to the current time.
\item[Total Trace Species Released] (kg): total mass of trace species released by the fire up to the current time.
\end{description}

\subsection{Species Output (project\_s.csv)}

At present, there are nine outputs in this file: oxygen (O2), carbon dioxide (CO2), carbon monoxide (CO),  hydrogen cyanide (HCN), hydrogen chloride (HCL), water vapor (H2O), optical depth (OD), concentration-time dose (CT) and trace species (TS). This set of nine is enumerated for the upper and lower layer, and are done sequentially by compartment.
\begin{description}
\item[Time] (s)
\item[O2 Upper/Lower Layer] (mol \%): oxygen concentration in the upper (or lower) layer in the current compartment
\item[CO2 Upper/Lower Layer] (mol \%):  carbon dioxide concentration in the upper (or lower) layer in the current compartment
\item[CO Upper/Lower Layer] (ppm by volume):  carbon monoxide concentration in the upper (or lower) layer in the current compartment
\item[HCN Upper/Lower Layer] (ppm by volume):  HCN concentration in the upper (or lower) layer in the current compartment
\item[HCL Upper/Lower Layer] (ppm by volume):  HCl concentration in the upper (or lower) layer in the current compartment
\item[H2O Upper/Lower Layer] (mol \%):  water vapor concentration in the upper (or lower) layer in the current compartment
\item[Optical Density Uppe/Lowerr Layer] (m$^{-1}$):  optical density in the upper (or lower) layer in the current compartment
\item[C-T Product Upper/Lower Layer] (g-min/m$^3$):  integrated concentration-time product in the upper (or lower) layer in the current compartment
\item[Trace Species Upper/Lower Layer] (kg):  total mass of trace species in the upper (or lower) layer in the current compartment
\end{description}

\subsection{Vent Flow (project\_f.csv)}

The first columns in this file pertain to the horizontal flow through vertical vents such as windows and doors. There are two types of output, first to and from the outside, and second for interior compartments. The flow is broken down to flow in and out of the compartments. For flow to and from the outside (compartment N) there are two entries.
\begin{description}
\item[Time] (s)
\item[HVENT Inflow Vent \# 1 Outside-1] (kg/s): mass flow into the current compartment through the current horizontal flow (door/windows) vent connected to the current compartment
\item[HVENT Outflow Vent \# 1 Outside-1] (kg/s): mass flow out of the current compartment through the current horizontal flow (door/windows) vent connected to the current compartment
\end{description}
The second set of columns pertain to the vertical flow through horizontal vents. There are two entries for each vent in each compartment, showing the total flow into or out of each vent in each compartment.
\begin{description}
\item[VVENT Inflow Vent \# 1-Outside] (kg/s):  mass flow into the current compartment through the current vertical flow (ceiling/floor) vent connected to the current compartment
\item[VVENT Outflow Vent \# 1-Outside Vent Connection at Node 1-2] (kg/s):  mass flow out of the current  compartment through the current vertical flow (ceiling/floor) vent connected to the current compartment
\end{description}
The third set of columns pertain to the mechanical ventilation. Once again, there will be an entry for each node/compartment pair, showing the total flow into or out of the compartment through this node. In addition, the total amount of trace species through filters and captured on filters in mechanical ventilation connected to the compartment is included.
\begin{description}
\item[MVENT Inflow Vent Connection at Node 1-2] (kg/s): mass flow into the current compartment through the current mechanical flow (HVAC) vent connected to the current compartment
\item[MVENT Outflow] (kg/s): current mass flow out of the compartment through the current mechanical flow (HVAC) vent connected to the current compartment
\item[MVENT Trace Species Flow Fan at Node 2] (kg/s): current total mass of trace species flowing through the filter at the fan in the current mechanical vent connected to the current compartment
\item[MVENT Trace Species Filtered Fan at Node 2] (kg/s): total mass of trace species captured on the filter at the fan in the current mechanical vent connected to the current compartment
\end{description}

\subsection{Surface and Target Temperature and Heat Flux (project\_w.csv)}

This file provides information on surface and target temperatures and flux, and reports on the current state of detectors and sprinklers (as a sub-set of detectors). The output is in three sections, one for wall surface temperatures, one for target temperature and heat flux, and one for detector/sprinkler temperature and activation. The first set of columns pertain to the temperature of compartment surfaces.
\begin{description}
\item[Time] (s)
\item[Ceiling Temperature] (\degc): temperature of the ceiling surface in the current compartment
\item[Upper Wall Temperature] (\degc): temperature of the wall surface adjacent to the upper layer in the current compartment
\item[Lower Wall Temperature] (\degc): temperature of the  wall surface adjacent to the lower layer in the current compartment
\item[Floor Temperature] (\degc): temperature of the floor surface in the current compartment
\end{description}
The second set of columns pertain to the user-defined targets included in the simulation.
\begin{description}
\item[Target Surrounding Gas Temperature] (\degc): gas temperature nearby the current target
\item[Target Surface Temperature] (\degc): temperature of the surface of the current target
\item[Target Center Temperature] (\degc): interior temperature of the current target
\item[Target Total Flux] (kW/m$2$): total net heat flux to the front surface of the current target
\item[Target Convective Flux] (kW/m$^2$): convective heat flux to the  front surface of the current target
\item[Target Radiative Flux] (kW/m$^2$): total net radiative heat flux to the front surface of the current target
\item[Target Fire Radiative Flux] (kW/m$^2$): radiative heat flux from fires to the front surface of the current target
\item[Target Surface Radiative Flux] (kW/m$^2$): radiative heat flux from compartment surfaces to the front surface of the current target
\item[Target Gas Radiative Flux] (kW/m$^2$):   radiative heat flux from the upper and lower gas layers to the front surface of the current target
\item[Target Radiative Loss Flux] (kW/m$^2$):   radiative heat flux from the current target to surroundings at the calculated temperature of the target
\item[Target Total Gauge Flux] (kW/m$2$): total net heat flux to the front surface of the current target assuming the target radiative losses are at ambient temperature
\item[Target Radiative Gauge Flux] (kW/m$^2$): total net radiative heat flux to the front surface of the current target assuming the target radiative losses are at ambient temperature
\item[Target Radiative Loss Gauge Flux] (kW/m$^2$):   radiative heat flux from the current target to surroundings assuming the target radiative losses are at ambient temperature
\end{description}
The third set of columns pertain to the detector/sprinkler output.
\begin{description}
\item[Sensor Temperature] (\degc): temperature of the current detector / sprinkler
\item[Sensor Activation] (none): indicator of activation of the current detector / sprinkler; takes a value of zero if the sensor has not activated and one if it has
\item[Sensor Surrounding Gas Temperature] (\degc): gas temperature nearby the current detector / sprinkler. This is the ceiling jet temperature at the device location if the device is in the ceiling jet or the appropriate gas layer temperature if the device is lower in the compartment
\item[Sensor Surrounding Gas Velocity] (m/s): gas velocity nearby the current detector / sprinkler. This is the velocity of the ceiling jet at the device location if the device is in the ceiling jet or a default value of 0.1 m/s if the device is lower in the compartment
\end{description}

\newpage

\section{Error Messages}

In some (hopefully rare) cases, a simulation will fail to complete. In those cases, an error message provides guidance to the user on possible reasons for the failure. The message will contain an error number which provides a reference to additional information from the table below. Most often, these errors result from improper information in the input data files.
During initialization of the program for a simulation, CFAST may stop with an error message if the simulation cannot be initialized due to a missing or incorrect file specification. The error codes are as follows:
\begin{description}
\item[100] program called with no arguments (no input file)
\item[101] internal error in fire input; code for a free burning fire should not be reachable
\item[102] project file does not exist
\item[103] total file name length including path  is more than 256 characters
\item[104] one of the output files is not accessible (for example, if a CFAST case with this name is already running)
\item[105] error writing to an output file (openoutputfiles)
\item[106] a system fault has occurred. Applies to all open/close pairs once the model is running
\item[107] incompatible options
\item[108] not currently used
\item[109] cannot find/open a file
\item[110] error in handling the status input/output
\end{description}
Error codes from 1 to 99 are from the routine which parses the input and will be reported in the .log file.  The first set indicates a command with the wrong number of arguments. These errors indicate an error in a particular input command as follows:
\begin{description}
\item[1] TIMES command
\item[2] TAMB command
\item[3] EAMB command
\item[4] LIMO2 command
\item[5] THERMAL or FIRE commands
\item[7] MAINF command
\item[8] COMPA command
\item[10] HVENT command
\item[11] EVENT command
\item[12] MVENT command
\item[23] VVENT command
\item[24] WIND command
\item[25] INTER command
\item[26] MVOPN command
\item[28] MVDCT command
\item[29] MVFAN command
\item[32] OBJECT command
\item[34] CJET and DETEC command
\item[35] STPMAX command
\item[37] VHEAT command
\item[39] ONEZ command
\item[41] TARGE command
\item[46] HALL command
\item[47] ROOMA command
\item[51] ROOMH command
\item[55] DTCHE command
\item[56] SETP command
\item[58] HHEAT command
\item[65] HEATF command
\end{description}
The second set of errors related to parsing the input indicate specific errors with a command as follows:
\begin{description}
\item[9, compa] Compartment out of range
\item[26, inter] Not a defined compartment
\item[27, mvopn] Specified node number too large for this system
\item[30, mvfan] Fan curve has incorrect specification
\item[31, mvfan] Exceeded allowed number of fans
\item[33, object] Object must be assigned to an existing compartment
\item[35, detect] Invalid DETECTOR specification
\item[36, detect] A referenced compartment is not yet defined
\item[38, vheat] VHEAT has specified a non-existent compartment
\item[42, target] Too many targets are being defined
\item[43, target] The compartment specified by TARGET does not exist
\item[44, target] Invalid TARGET solution method specified
\item[45,  target] Invalid equation type specified in TARGET
\item[49, rooma] Compartment specified by ROOMA does not exist
\item[52, roomh] Compartment specified by ROOMH is not defined
\item[53, roomh] ROOMH error on data line
\item[54, roomh] Data on the ROOMA (or H) line must be positive
\item[57, setp] Trying to reset the SETP parameters
\item[61, hheat] HHEAT specification error in compartment pairs
\item[62, hheat] Error in fraction for HHEAT
\item[63, object] Fire type out of range
\item[64, object] The fire must be assigned to an existing compartment
\item[66, heatf] The heat source must be assigned to an existing compartment
\item[67, mvent] Compartment has not been defined
\item[68, mvent] Exceed one of the array bounds, ierror=68 (external), 69 (internal)  and 70 (fan)
\item[71, event] Undefined vent type
\item[72, inter] Specification for interface height is outside of allowable range
\item[73, inter] Compartments must be defined in pairs
\item[74, setp] The requested “SETP” command does not exists
\item[75, setp] Incorrect file reference
\item[76, setp] Cannot read the parameter file
\item[77, setp] Unsupported parameter
\end{description}
Errors 400 and above are failures while the model is running. 610 through 685 are failures in the numerical routines; these are rarely seen, but typically result from an internal error in the model.


